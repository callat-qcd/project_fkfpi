\documentclass[prd,11pt,tightenlines,preprintnumbers,showpacs,superscriptaddress,notitlepage,nofootinbib,eqsecnum,floatfix,longbibliography]{revtex4-1}
% LOAD PREAMBLE
\usepackage{graphicx}  % needed for figures
\usepackage{dcolumn}   % needed for some tables
\usepackage{bm}        % for math
\usepackage{amssymb}   % for math
\usepackage{standalone}
\usepackage{enumitem}
\usepackage[pdftex]{color}
\usepackage{xcolor}
\usepackage{slashed}
\usepackage{booktabs}
\usepackage{multirow}
\usepackage{amsmath}
\usepackage{stackrel}
\usepackage{rotating}
\usepackage{CJKutf8}
\usepackage{pifont}
\usepackage{mathtools}
\usepackage{hyperref}
\hypersetup{
    colorlinks=true,       % false: boxed links; true: colored links
    linkcolor=blue,          % color of internal links
    citecolor=blue,        % color of links to bibliography
    filecolor=blue,      % color of file links
    urlcolor=blue           % color of external links
}
\usepackage{simplewick}
% NEW COMMANDS
% NEW COMMANDS

% define colors
\definecolor{kngrey}{HTML}{A6AAA9}
\definecolor{knred}{HTML}{EC5D57}
\definecolor{knorange}{HTML}{F39019}
\definecolor{knyellow}{HTML}{F5D328}
\definecolor{kngreen}{HTML}{70BF41}
\definecolor{knblue}{HTML}{51A7F9}
\definecolor{knpurple}{HTML}{B36AE2}

% moved from FH paper
\newcommand*\dstrike[2][thin]{\tikz[baseline] \node [strike out,draw,anchor=text,inner sep=0pt,text=black,#1]{#2};}  

\DeclareMathOperator{\st}{str}
\DeclareMathOperator{\tr}{tr}
\DeclareMathOperator{\Erfc}{Erfc}
\DeclareMathOperator{\Erf}{Erf}
\DeclareMathOperator{\Tr}{Tr}


\def\gs{\bar{g}_0}
\def\gv{\bar{g}_1}
\def\cp{\textrm{\dstrike{CP}}}
\def\mc#1{{\mathcal #1}}


\def\a{{\alpha}}
\def\b{{\beta}}
\def\d{{\delta}}
\def\D{{\Delta}}
\def\t{\tau}
\def\e{{\varepsilon}}
\def\g{{\gamma}}
\def\G{{\Gamma}}
\def\k{{\kappa}}
\def\l{{\lambda}}
\def\L{{\Lambda}}
\def\m{{\mu}}
\def\n{{\nu}}
\def\o{{\omega}}
\def\O{{\Omega}}
\def\S{{\Sigma}}
\def\s{{\sigma}}
\def\th{{\theta}}

\def\ol#1{{\overline{#1}}}


\def\Dslash{D\hskip-0.65em /}
\def\Dtslash{\tilde{D} \hskip-0.65em /}

\def\mbfx{\mathbf{x}}

\def\CPT{{$\chi$PT}}
\def\QCPT{{Q$\chi$PT}}
\def\PQCPT{{PQ$\chi$PT}}
\def\tr{\text{tr}}
\def\str{\text{str}}
\def\diag{\text{diag}}
\def\order{{\mathcal O}}


\def\cC{{\mathcal C}}
\def\cB{{\mathcal B}}
\def\cT{{\mathcal T}}
\def\cQ{{\mathcal Q}}
\def\cL{{\mathcal L}}
\def\cO{{\mathcal O}}
\def\cA{{\mathcal A}}
\def\cQ{{\mathcal Q}}
\def\cR{{\mathcal R}}
\def\cH{{\mathcal H}}
\def\cW{{\mathcal W}}
\def\cE{{\mathcal E}}
\def\cM{{\mathcal M}}
\def\cD{{\mathcal D}}
\def\cN{{\mathcal N}}
\def\cP{{\mathcal P}}
\def\cK{{\mathcal K}}
\def\Qt{{\tilde{Q}}}
\def\Dt{{\tilde{D}}}
\def\St{{\tilde{\Sigma}}}
\def\cBt{{\tilde{\mathcal{B}}}}
\def\cDt{{\tilde{\mathcal{D}}}}
\def\cTt{{\tilde{\mathcal{T}}}}
\def\cMt{{\tilde{\mathcal{M}}}}
\def\At{{\tilde{A}}}
\def\cNt{{\tilde{\mathcal{N}}}}
\def\cOt{{\tilde{\mathcal{O}}}}
\def\cPt{{\tilde{\mathcal{P}}}}
\def\cI{{\mathcal{I}}}
\def\cJ{{\mathcal{J}}}


\def\eqref#1{{(\ref{#1})}}

\newcommand{\new}[1]{{\color{red}{#1}}}
\definecolor{gray}{rgb}{0.6,0.6,0.6}
\newcommand{\old}[1]{{\color{gray}{#1}}}


\def\ip{{i^\prime}}
\def\jp{{j^\prime}}
\def\kp{{k^\prime}}
\def\ap{{\alpha^\prime}}
\def\bp{{\beta^\prime}}
\def\gp{{\gamma^\prime}}
\def\rp{{\rho^\prime}}


% Greek Letters
\def\a{{\alpha}}
\def\b{{\beta}}
\def\d{{\delta}}
\def\D{{\Delta}}
\def\e{{\epsilon}}
\def\g{{\gamma}}
\def\G{{\Gamma}}
\def\k{{\kappa}}
\def\l{{\lambda}}
\def\L{{\Lambda}}
\def\m{{\mu}}
\def\n{{\nu}}
\def\w{{\omega}}
\def\O{{\Omega}}
\def\S{{\Sigma}}
\def\s{{\sigma}}
\def\t{{\tau}}
\def\th{{\theta}}
\def\x{{\xi}}

\def\ol#1{{\overline{#1}}}

%slash's
\def\Dslash{D\hskip-0.65em /}
\def\dslash{{\partial\hskip-0.5em /}}
\def\vslash{{\rlap \slash v}}
\def\qbar{{\overline q}}

% Jargon
\def\CPT{{$\chi$PT}}
\def\QCPT{{Q$\chi$PT}}
\def\PQCPT{{PQ$\chi$PT}}
\def\tr{\text{tr}}
\def\str{\text{str}}
\def\diag{\text{diag}}
\def\order{{\mathcal O}}
\def\vit{{\it v}}
\def\vD{\vit\cdot D}
\def\am{\alpha_M}
\def\bm{\beta_M}
\def\gm{\gamma_M}
\def\smb{\sigma_M}
\def\smt{\overline{\sigma}_M}
\def\tb{{\tilde b}}

\def\mc#1{{\mathcal #1}}

% Fields
\def\Bbar{\overline{B}}
\def\Tbar{\overline{T}}
\def\cBbar{\overline{\cal B}}
\def\cTbar{\overline{\cal T}}
\def\pq{(PQ)}


\def\eqref#1{{(\ref{#1})}}

\newcommand{\glasgow}{
 School of Physics and Astronomy, 
    University of Glasgow, 
    Glasgow G12 8QQ, UK
 }
\newcommand{\jlabt}{
	Theory Center, 
	Thomas Jefferson National Accelerator Facility, 
	Newport News, VA 23606, USA
	}
\newcommand{\jlabc}{
	Scientific Computing Group, 
	Thomas Jefferson National Accelerator Facility, 
	Newport News, VA 23606, USA
	}
\newcommand{\julich}{
 Institut f\"{u}r Kernphysik and Institute for Advanced Simulation,
 Forschungszentrum J\"{u}lich, 54245 J\"{u}lich Germany
 }
\newcommand{\liverpool}{
    Theoretical Physics Division, Department of Mathematical Sciences, University of Liverpool, Liverpool L69 3BX, UK
    }
\newcommand{\lblnsd}{
    Nuclear Science Division,
    Lawrence Berkeley National Laboratory,
	Berkeley, CA 94720, USA
	}
\newcommand{\lblnersc}{
    NERSC,
    Lawrence Berkeley National Laboratory,
	Berkeley, CA 94720, USA
	}
\newcommand{\llnl}{
	Physics Division,
	Lawrence Livermore National Laboratory,
	Livermore, CA 94550, USA
	}
\newcommand{\nvidia}{
    NVIDIA Corporation,
    2701 San Tomas Expressway, Santa Clara, CA 95050, USA
    }
\newcommand{\rbrc}{
	RIKEN-BNL Research Center, 
	Brookhaven National Laboratory, 
	Upton, NY 11973, USA
	}
\newcommand{\rutgers}{
	Physics Department and Astronomy,
	Rutgers,
	The State University of New Jersey, Piscataway, NJ 08854, USA
	}
\newcommand{\ucb}{
	Department of Physics,
	University of California,
	Berkeley, CA 94720, USA
	}
\newcommand{\unc}{
	Department of Physics and Astronomy,
	University of North Carolina,
	Chapel Hill, NC 27516-3255, USA
	}
\newcommand{\intuw}{
    Institute for Nuclear Theory,
    University of Washington,
    Seattle, WA 98195, USA
    }
\newcommand{\wm}{
	Department of Physics,
	The College of William \& Mary,
	Williamsburg, VA 23187, USA
	}


\newcount\hour \newcount\hourminute \newcount\minute 
\hour=\time \divide \hour by 60
\hourminute=\hour \multiply \hourminute by 60
\minute=\time \advance \minute by -\hourminute
\newcommand{\mydate}{\ \today \ - \number\hour :\number\minute}


\begin{document}

\title{$F_K / F_\pi$ notes}

%\author{Evan~Berkowitz}
%\affiliation{\julich}

%\author{Chris~Bouchard}
%\affiliation{\glasgow}

%\author{David~A.~Brantley}
%\affiliation{\wm}
%\affiliation{\lblnsd}
%\affiliation{\llnl}

%\author{Chia-Cheng Chang}
%\affiliation{\lblnsd}
%
%\author{M.A.~Clark}
%\affiliation{\nvidia}
%
%\author{Arjun~Gambhir}
%\affiliation{\llnl}
%\affiliation{\lblnsd}
%\affiliation{\ucb}
%
%\author{Nicolas~Garron}
%\affiliation{\liverpool}
%
%\author{B\'{a}lint~Jo\'{o}}
%\affiliation{\jlabt}
%
%\author{Thorsten~Kurth}
%\affiliation{\lblnersc}
%
%\author{Chris~Monahan}
%\affiliation{\intuw}
%
%\author{Henry~Monge-Camacho}
%\affiliation{\wm}
%\affiliation{\lblnsd}
%
%\author{Amy~Nicholson}
%\affiliation{\unc}
%
%%\author{Kostas Orginos}
%%\email[]{kostas@wm.edu}
%%\affiliation{\wm}
%%\affiliation{\jlab}
%
%\author{Enrico~Rinaldi}
%\affiliation{\rbrc}
%
%\author{Pavlos~Vranas}
%\affiliation{\llnl}
%\affiliation{\lblnsd}

\author{Andr\'{e}~Walker-Loud}
%\email[]{awalker-loud@lbl.gov}
%\affiliation{\lblnsd}
%\affiliation{\llnl}
%\affiliation{\ucb}

\date{\mydate}

\begin{abstract}
Notes on $F_K/F_\pi$ extrapolation
\end{abstract}
\maketitle
%%%%%%%%%%%%%%%%%%%%%%%%%%%%%%%%%
%%%%%%%%%%%%%%%%%%%%%%%%%%%%%%%%%
%%%%%%%%%%%%%%%%%%%%%%%%%%%%%%%%%

%\tableofcontents

\section{Strategy}
We want to perform several extrapolations of our $F_K/F_\pi$ data and perform a weighted model average of these extrapolations.
The six extrapolations to be performed are:
\begin{enumerate}
\item NLO MA + NNLO c.t.: ratio of $F_K$ and $F_\pi$ fit functions
\begin{enumerate}
	\item Use full on-shell mixed meson mass as they appear
	\item Use average mixed-meson mass splitting, where average is taken over all flavors and quark masses
\end{enumerate}
\item NLO MA + NNLO c.t.: Taylor expanded ratio of $F_K/F_\pi$ extrapolation functions
\begin{enumerate}
	\item Use full on-shell mixed meson mass as they appear
	\item Use average mixed-meson mass splitting, where average is taken over all flavors and quark masses
\end{enumerate}



\item NLO $\chi$PT + discretization + NNLO c.t.: ratio of $F_K$ and $F_\pi$ fit functions

\item NLO $\chi$PT + discretization + NNLO c.t.: Taylor expanded ratio of $F_K/F_\pi$ extrapolation functions

\end{enumerate}
In all cases, the predicted NLO finite volume corrections should be used.
Also - we can change these 6 fits into 18 by treating the $F^2$ terms that appear in the NLO expressions and definitions of the $\e_\phi^2$ parameters as
\begin{equation}
F^2 \rightarrow \left\{
	\begin{matrix}
	F_\pi^2\\
	F_\pi F_K\\
	F_K^2
	\end{matrix}\right.
\end{equation}




\subsection{Non analytic NLO loop integral functions}
At NLO, the following functions arise in the extrapolation formulae (the $\ln$ terms arise from the infinite volume integrals and the sums over Bessel functions are the finite volume corrections):
\begin{equation}
\mc{I}(m) = \frac{m^2}{(4\pi)^2} \ln \left( \frac{m^2}{\mu^2} \right)
	+ \frac{m^2}{4\pi^2} \sum_{|\mathbf{n}|\neq0} \frac{c_n}{mL|\mathbf{n}|} K_1(mL|\mathbf{n}|)
\end{equation}
\begin{align}
d\mc{I}(m) &= \frac{1 + \ln \left( \frac{m^2}{\mu^2} \right)}{(4\pi)^2} 
	+\sum_{|\mathbf{n}|\neq0} \frac{c_n}{(4\pi)^2} \left[
		\frac{2K_1(mL|\mathbf{n}|)}{mL|\mathbf{n}|}
		-K_0(mL|\mathbf{n}|)
		-K_2(mL|\mathbf{n}|)
	\right]
\nonumber\\&=
	\frac{1}{(4\pi)^2} + \frac{\mc{I}(m)}{m^2}
	+\sum_{|\mathbf{n}|\neq0} \frac{c_n}{(4\pi)^2} \left[
		\frac{2K_1(mL|\mathbf{n}|)}{mL|\mathbf{n}|}
		-K_0(mL|\mathbf{n}|)
		-K_2(mL|\mathbf{n}|)
	\right]
\end{align}
\begin{equation}
\mc{K}(m,M) = \frac{1}{M^2 - m^2} \left[ \mc{I}(M) - \mc{I}(m) \right]
\end{equation}
\begin{align}
\mc{K}_{21}(m,M) &= \int_R \frac{d^d k}{(2\pi)^d} \frac{1}{(k^2 -m^2)^2} \frac{1}{k^2 - M^2}
\nonumber\\&=
	\frac{\partial}{\partial m^2} \int_R \frac{d^d k}{(2\pi)^d} \frac{1}{k^2 -m^2} \frac{1}{k^2 - M^2}
\nonumber\\&=
	\frac{\partial}{\partial m^2} \mc{K}(m,M)
\nonumber\\&=
	\frac{1}{(M^2 - m^2)^2} \left[ \mc{I}(M) - \mc{I}(m) \right]
	-\frac{1}{M^2 - m^2} d\mc{I}(m)
\end{align}
\begin{align}
\mc{K}(m_1,m_2,m_3) &= \frac{1}{m_1^2 - m_2^2}\frac{1}{m_1^2-m_3^2} \mc{I}(m_1)
	+\frac{1}{m_2^2 - m_1^2} \frac{1}{m_2^2 - m_3^2} \mc{I}(m_2)
\nonumber\\&\phantom{=}
	+\frac{1}{m_3^2 - m_1^2} \frac{1}{m_3^2 - m_2^2} \mc{I}(m_3)
\end{align}


The NNLO and NNNLO counter term expressions are 
\begin{equation}
\d_{c.t.}^{\rm NNLO} = \e_a^2 (\e_K^2 - \e_\pi^2) A_{2,2}
	+(\e_K^2 - \e_\pi^2)^2 M_{4,0,0}
	+\e_K^2 (\e_K^2 - \e_\pi^2) M_{2,2,0}
	+\e_\pi^2 (\e_K^2 - \e_\pi^2) M_{2,0,2}
\end{equation}
\begin{equation}
\d_{c.t.}^{\rm NNNLO} = \e_a^4 (\e_K^2 - \e_\pi^2) A_{4,2}
	+\e_a^2 (\e_K^2 - \e_\pi^2)^2 A_{2,4}
	+\e_a^2 \e_K^2 (\e_K^2 - \e_\pi^2) A_{2,2,2,0}
	+\e_a^2 \e_\pi^2 (\e_K^2 - \e_\pi^2) A_{2,2,0,2}
\end{equation}



\subsection{NLO chiral extrapolation formulae}

In these expressions, the $\eta$ meson mass can be approximated with the $SU(3)$ Gell-Mann--Okubo formula
\begin{equation}
m_\eta^2 = \frac{4}{3} m_K^2 - \frac{1}{3}m_\pi^2\, .
\end{equation}
The $X$ mass appearing in the MA formulae is
\begin{equation}
m_X^2 = m_\eta^2 + a^2 \D_{\rm I}
\end{equation}
where $a^2 \D_{\rm I}$ is the taste-identity mass splitting which MILC has determined~\cite{Bazavov:2012xda}.

Given the quark mass tuning we have done, at LO in MA EFT~\cite{Chen:2006wf}, all partial quenching parameters are given by the taste-identity splitting
\begin{equation}
\D_{rs}^2 = \D_{ju}^2 = a^2 \D_{\rm I}\, .
\end{equation}
Also, all the mixed meson masses are given by
\begin{equation}
m_{val,sea}^2 = \frac{1}{2}m_{val,val}^2 + \frac{1}{2}m_{sea,sea,5}^2 + a^2 \Delta_{\rm Mix}
	\simeq m_{val,val}^2 + a^2 \Delta_{\rm Mix}\, .
\end{equation}
We can plot the splitting over flavors and pion masses, and see how well this LO relation holds.  We will also fit using the full on-shell masses as they appear in the MA formula as well as use an average mixed-meson mass splitting, $a^2 \D_{\rm mix}$, averaged over the flavors and ensembles (possibly replace the average with extrapolation).







\subsubsection{NLO MA + NNLO c.t. ratio}
For these fits, we will do
\begin{equation}
\frac{F_K}{F_\pi} = \frac{F_K^{\rm nlo}}{F_\pi^{\rm nlo}} + \d_{c.t.}^{\rm NNLO}
\end{equation}
where
\begin{align}
\frac{F_\pi^{\rm nlo}}{F_0} &= 1
	- \frac{\mc{I}(m_{ju})}{F^2}
	-\frac{\mc{I}(m_{ru})}{2F^2}
	+4 \e_\pi^2 (4\pi)^2 (L_4 + L_5)
	+8 \e_K^2 (4\pi)^2 L_4
\end{align}

\begin{align}
\frac{F_K^{\rm nlo}}{F_0} &= 1
	-\frac{\mc{I}(m_{ju})}{2F^2}
	+\frac{\mc{I}(m_\pi)}{8F^2}
	-\frac{\mc{I}(m_{ru})}{4F^2}
	-\frac{\mc{I}(m_{sj})}{2F^2}
	-\frac{\mc{I}(m_{rs})}{4F^2}
	+\frac{\mc{I}(m_{ss})}{4F^2}
	-\frac{3\mc{I}(m_X)}{8F^2}
\nonumber\\&\phantom{=}
	+4\e_\pi^2 L_4 + 4\e_K^2(L_5 + 2L_4)
\nonumber\\&\phantom{=}
	+\D_{ju}^2 \left[ -\frac{d\mc{I}(m_\pi)}{8F^2} + \frac{\mc{K(}m_\pi,m_X)}{4F^2} \right]
	-\D_{ju}^4 \frac{\mc{K}_{21}(m_\pi,m_X)}{24 F^2}
\nonumber\\&\phantom{=}
	+\D_{ju}^2\D_{rs}^2 \left[ \frac{\mc{K}(m_\pi, m_{ss}, m_X)}{6F^2}
		+\frac{\mc{K}_{21}(m_{ss},m_X)}{12F^2} \right]
\nonumber\\&\phantom{=}
	+\D_{rs}^2 \left[
		\frac{\mc{K}(m_{ss},m_X)}{4F^2} 
		-\frac{\mc{K}_{21}(m_{ss},m_X) m_K^2}{6F^2}
		+\frac{\mc{K}_{21}(m_{ss},m_X) m_\pi^2}{6F^2}
		\right]
\end{align}


\subsubsection{NLO MA + NNLO c.t. Taylor expanded ratio}
\begin{align}\label{eq:fkfpi_ma}
\frac{F_K}{F_\pi} &= 1
	+\frac{\mc{I(m_{ju})}}{2F^2}
	+\frac{\mc{I}(m_\pi)}{8F^2}
	+\frac{\mc{I}(m_{ru})}{4F^2}
	-\frac{\mc{I}(m_{sj})}{2F^2}
	+\frac{\mc{I}(m_{ss})}{4F^2}
	-\frac{\mc{I}(m_{rs})}{4F^2}
	-\frac{3\mc{I}(m_X)}{8F^2}
\nonumber\\&\phantom{=}
	+\D_{ju}^2 \left[ -\frac{d\mc{I}(m_\pi)}{8F^2} + \frac{\mc{K(}m_\pi,m_X)}{4F^2} \right]
	-\D_{ju}^4 \frac{\mc{K}_{21}(m_\pi,m_X)}{24 F^2}
\nonumber\\&\phantom{=}
	+\D_{ju}^2\D_{rs}^2 \left[ \frac{\mc{K}(m_\pi, m_{ss}, m_X)}{6F^2}
		+\frac{\mc{K}_{21}(m_{ss},m_X)}{12F^2} \right]
\nonumber\\&\phantom{=}
	+\D_{rs}^2 \left[
		\frac{\mc{K}(m_{ss},m_X)}{4F^2} 
		-\frac{\mc{K}_{21}(m_{ss},m_X) m_K^2}{6F^2}
		+\frac{\mc{K}_{21}(m_{ss},m_X) m_\pi^2}{6F^2}
		\right]
\nonumber\\&\phantom{=}
	+ 4 (4\pi)^2 L_5(\mu) \frac{m_K^2 - m_\pi^2}{(4\pi F)^2}
\end{align}

\subsubsection{NLO $\chi$PT + NNLO c.t. ratio}
For these fits, we will do
\begin{equation}
\frac{F_K}{F_\pi} = \frac{F_K^{\rm nlo}}{F_\pi^{\rm nlo}} + \d_{c.t.}^{\rm NNLO}
\end{equation}
where
\begin{align}
\frac{F_\pi^{\rm nlo}}{F_0} &= 1
	- \frac{\mc{I}(m_\pi)}{F^2}
	-\frac{1}{2}\frac{\mc{I}(m_K)}{F^2}
	+4 \e_\pi^2 (4\pi)^2 (L_4 + L_5)
	+8 \e_K^2 (4\pi)^2 L_4
\end{align}
and
\begin{align}
\frac{F_K^{\rm nlo}}{F_0} &= 1
	-\frac{3}{8}\frac{\mc{I}(m_\pi)}{F^2}
	-\frac{3}{4}\frac{\mc{I}(m_K)}{F^2}
	-\frac{3\mc{I}(m_\eta)}{8F^2}
%\nonumber\\&\phantom{=}
	+4\e_\pi^2 L_4 + 4\e_K^2(L_5 + 2L_4)
\end{align}



\subsubsection{NLO $\chi$PT + NNLO c.t. Taylor expansion}
\begin{align}\label{eq:fkfpi_chpt}
\frac{F_K}{F_\pi} &= 1
	+\frac{5}{8} \frac{\mc{I}(m_\pi)}{F^2}
	-\frac{1}{4} \frac{\mc{I}(m_K)}{F^2}
	-\frac{3}{8} \frac{\mc{I}(m_\eta)}{F^2}
%\nonumber\\&\phantom{=}
	+4(\e_K^2 - \e_\pi^2) (4\pi)^2 L_5(\L_{\chi})
%\nonumber\\&\phantom{=}
	+ \d_{c.t.}^{\rm NNLO}
\end{align}



\section{OLD}
In this expression
We can estimate the FV corrections by tracking only the terms with $\pi$ and $ju$ meson loops.
Introducing shorthand notation, we find
\begin{align}
\d \frac{F_K}{F_\pi} &= 
	\frac{\d\mc{I(m_{ju})}}{2F^2}
	+\frac{\d\mc{I}(m_\pi)}{8F^2}
	+\D_{ju}^2 \left[ -\frac{\d d\mc{I}(m_\pi)}{8F^2} + \frac{\d\mc{K(}m_\pi,m_X)}{4F^2} \right]
\nonumber\\&\phantom{=}
	-\D_{ju}^4 \frac{\d\mc{K}_{21}(m_\pi,m_X)}{24 F^2}
	+\D_{ju}^2\D_{rs}^2 \frac{\d\mc{K}(m_\pi, m_{ss}, m_X)}{6F^2}
\nonumber\\&= \sum_{|\mathbf{n}|\neq0} c_n \bigg\{
	2\frac{\e_{ju}^2 K_1^{ju}}{x^{Ln}_{ju}}
	+\frac{1}{2}\frac{\e_{\pi}^2 K_1^\pi}{x^{Ln}_{\pi}}
%\nonumber\\&\qquad 
	-\frac{1}{8}\frac{\D_{ju}^2}{(4\pi F)^2} \left[
		\frac{2K_1^\pi}{x^{Ln}_{\pi}}
		-K_0^\pi
		-K_2^\pi
	\right]
\nonumber\\&\qquad 
	+\frac{\D_{ju}^2}{4F^2} \frac{-1}{(m_X^2-m_\pi^2)} \frac{m_\pi^2}{4\pi^2} \frac{K_1^\pi}{x^{Ln}_{\pi}}
\nonumber\\&\qquad 
	-\frac{\D_{ju}^4}{24F^2} \left[
	\frac{1}{(m_X^2-m_\pi^2)^2}\frac{-m_\pi^2}{4\pi^2} \frac{K_1^\pi}{x^{Ln}_{\pi}}
	-\frac{1}{m_X^2 -m_\pi^2}\frac{1}{(4\pi)^2} \left[
		\frac{2K_1^\pi}{x^{Ln}_{\pi}}
		-K_0^\pi
		-K_2^\pi
	\right]
	\right]
\nonumber\\&\qquad 
	+\frac{\D_{ju}^2\D_{rs}^2}{6F^2} \frac{1}{(m_\pi^2-m_X^2)(m_\pi^2-m_{ss}^2)}
	\frac{m_\pi^2}{4\pi^2} \frac{K_1^\pi}{x^{Ln}_{\pi}}
	\bigg\}
\nonumber\\&= \sum_{|\mathbf{n}|\neq0} c_n \bigg\{
	2\e_{ju}^2\frac{K_1^{ju}}{x^{Ln}_{ju}}
	+\frac{\e_{\pi}^2}{2}\frac{K_1^\pi}{x^{Ln}_{\pi}}
	-\frac{\e_{\D_{ju}}^2}{8} \left[
		\frac{2K_1^\pi}{x^{Ln}_{\pi}}
		-K_0^\pi
		-K_2^\pi
	\right]
	-\frac{\e_{\D_{ju}}^2 \e_\pi^2}{\e_X^2-\e_\pi^2} \frac{K_1^\pi}{x^{Ln}_{\pi}}
\nonumber\\&\qquad 
	+\frac{\e_{\D_{ju}}^4}{24}(4\pi)^4F^2 \left[
	\frac{4F^2\e_\pi^2 / (4\pi F)^4}{(\e_X^2-\e_\pi^2)^2} \frac{K_1^\pi}{x^{Ln}_{\pi}}
	+\frac{1/(4\pi F)^2}{\e_X^2 -\e_\pi^2}\frac{1}{(4\pi)^2} \left[
		\frac{2K_1^\pi}{x^{Ln}_{\pi}}
		-K_0^\pi
		-K_2^\pi
	\right]
	\right]
\nonumber\\&\qquad 
	+\frac{\e_{\D_{ju}}^2\e_{\D_{rs}}^2 4\e_\pi^2}{6(\e_\pi^2-\e_X^2)(\e_\pi^2-\e_{ss}^2)}
	 \frac{K_1^\pi}{x^{Ln}_{\pi}}
	\bigg\}
%FINAL SIMPLIFICATION
\nonumber\\&= \sum_{|\mathbf{n}|\neq0} c_n \bigg\{
	2\e_{ju}^2\frac{K_1^{ju}}{x^{Ln}_{ju}}
	+\frac{\e_{\pi}^2}{2}\frac{K_1^\pi}{x^{Ln}_{\pi}}
	-\frac{\e_{\D_{ju}}^2}{8} \left[
		\frac{2K_1^\pi}{x^{Ln}_{\pi}}
		-K_0^\pi
		-K_2^\pi
	\right]
	-\frac{\e_{\D_{ju}}^2 \e_\pi^2}{\e_X^2-\e_\pi^2} \frac{K_1^\pi}{x^{Ln}_{\pi}}
\nonumber\\&\qquad 
	+\frac{\e_{\D_{ju}}^4}{24} \left[
	\frac{4\e_\pi^2}{(\e_X^2-\e_\pi^2)^2} \frac{K_1^\pi}{x^{Ln}_{\pi}}
	+\frac{1}{\e_X^2 -\e_\pi^2} \left[
		\frac{2K_1^\pi}{x^{Ln}_{\pi}}
		-K_0^\pi
		-K_2^\pi
	\right]
	\right]
\nonumber\\&\qquad 
	+\frac{2}{3}\frac{\e_{\D_{ju}}^2\e_{\D_{rs}}^2 \e_\pi^2}{(\e_\pi^2-\e_X^2)(\e_\pi^2-\e_{ss}^2)}
	 \frac{K_1^\pi}{x^{Ln}_{\pi}}
	\bigg\}
\end{align}
where $K_i^\phi = K_i(m_\phi L |\mathbf{n}|)$, $x^{LN}_\phi = m_\phi L |\mathbf{n}|$ and $\e_Y^2 = m_Y^2 / (4\pi F)^2$.
The first 10 weights in the finite volume sum are listed in Table~\ref{tab:cn_weigths}.


%%%%%%%%%%%%%%%%%%%%%%%%%
% cN table
\begin{table}
\begin{ruledtabular}
\begin{tabular}{c|cccccccccc}
$|\mathbf{n}|$& 1 & $\sqrt{2}$& $\sqrt{3}$& $\sqrt{4}$& $\sqrt{5}$& $\sqrt{6}$& $\sqrt{7}$& $\sqrt{8}$& $\sqrt{9}$& $\sqrt{10}$\\
\hline
$c_n$& 6&12& 8& 6& 24& 24& 0& 12& 30& 24
\end{tabular}
\end{ruledtabular}
\caption{\label{tab:cn_weigths}
Finite volume weight factors for the first few finite volume modes.
}
\end{table}




\bibliography{c51_bib}



%%%%%%%%%%%%%%%%%%%%%%%%%%%%%%%%%
%%%%%%%%%%%%%%%%%%%%%%%%%%%%%%%%%
%%%%%%%%%%%%%%%%%%%%%%%%%%%%%%%%%

\end{document}

