\documentclass[12pt,a4paper]{article}

\usepackage[utf8]{inputenc}
\usepackage{hyperref}
\usepackage{makeidx}
\usepackage{a4wide}
\usepackage{titlesec}
\usepackage{amssymb}
\usepackage{amsmath}

\makeindex

%%%%%%%%%%%%%%%%%%%%%%%%%%%%%%%%%%%%%%%%%%%%%%%%%%%%%%%%%%%%%%%%%%%%%%%%%%%%%%
\newcommand{\mytt}[1]{\texttt{#1}}
\newcommand{\newfunction}[1]{\mytt{#1}\index{\mytt{#1}}}
\newcommand{\cernlib}{\textsc{CERNLIB}\cite{cernlib}}
\newcommand{\cpp}{\textsc{C++}}
\newcommand{\fortran}{\textsc{FORTRAN}}

% commands (with the package titlesec) that put paragraph as extra numbered
% level and into the table of contents
\setcounter{secnumdepth}{4}
\setcounter{tocdepth}{4}
\titleformat{\paragraph}
{\normalfont\normalsize\bfseries}{\theparagraph}{1em}{}
\titlespacing*{\paragraph}
{0pt}{3.25ex plus 1ex minus .2ex}{1.5ex plus .2ex}


\setlength{\parskip}{0cm}
\setlength{\parindent}{0cm}
\allowdisplaybreaks
%%%%%%%%%%%%%%%%%%%%%%%%%%%%%%%%%%%%%%%%%%%%%%%%%%%%%%%%%%%%%%%%%%%%%%%%%%%%%%
% change these when updating
% remember to run makeindex manual and run latex a few times afterwards
\newcommand{\chironversion}{v0.54}
\newcommand{\versiondate}{September 2015}

\begin{document}
\begin{titlepage}
\setcounter{page}{-1} % this avoids the pdflatex warning about same destination
\phantom{p}
\vfill
\begin{center}
{\large \bf CHIRON \chironversion\ Manual and User Guide}
\\[1cm]
{\bf Johan Bijnens}\\[0.5cm]
{Department of Astronomy and Theoretical Physics, Lund University\\
Sölvegatan 14A, SE 22362 Lund, Sweden}
\vfill
Manual version of \versiondate
\vfill
\end{center}
\begin{abstract}
This manual and user guide describes the classes and functions contained
in the ChPT program collection \textsc{CHIRON}\chironversion\ which includes
the numerical library \mytt{jbnumlib} and the ChPT
routine library \mytt{chiron}.
\end{abstract}
\vfill
\noindent This text is licensed under the creative commons license\\CC-BY 4.0
(\url{http://creativecommons.org/licenses/by/4.0/}), except some parts which are
under their own license.
\end{titlepage}

\phantomsection % to make the link go correctly
\addcontentsline{toc}{section}{Contents}
\tableofcontents

\section{Introduction}
\label{introduction}
\index{Introduction}

This is the manual and user guide for the Chiral Perturbation Theory
package \textsc{CHIRON} \chironversion. It also defines the functions
included in a more extended fashion as compared to the
published short description \cite{Bijnens:2014gsa}. There is obviously
a large overlap with that publication.

The numerical routines are described in Sect.~\ref{sec:jbnumlib}.
The remaining sections are devoted to the \mytt{chiron} library.

This manual is released under the creative commons license CC-BY 4.0
\cite{CClicense}
as reproduced in App.~\ref{appCCBY4} except for the parts
in App.~\ref{appGPL} and App.~\ref{appCCBY4} which have their own licenses.
The software itself is released under the GNU General Public License (GPL)
version 2 or later, which is reproduced in App.~\ref{appGPL}.

Kheiron, $X\varepsilon\iota\rho\omega\nu$, or Chiron, was the wisest
and eldest of the Centaurs, half-horse men of Greek mythology.
His name comes from the Greek word for hand (Kheir)
which is also the origin of the word chiral which is why his name was deemed
appropriate for this package \cite{chiron}.

\section{Guidelines}
\label{guidelines}
\index{Guidelines}

\subsection{Main comments}

Most of these routines were produced during and after scientific research.
They are licensed under the GPL v2 or later, see App.~\ref{appGPL}, \cite{GPLv2}
or the file \mytt{COPYING} in the main directory, so you have very strong rights
in using and modifying them. However, please respect the guidelines as described
in the file \mytt{GUIDELINES} in the main directory. A summary of these
is
\begin{itemize}
\item Citations are important in the academic world so when using
these please both cite the relevant \textsc{CHIRON} publication
\cite{Bijnens:2014gsa} and the papers where the work itself was done
as quoted in the different chapters.
\item Suspected bugs, proposed fixes and suggestions should preferably be
communicated to the author(s) so they can be added in future releases.
\item If you distribute modified versions, please indicate clearly the
modifications in the source and at the point of distributions. However,
the preferred way to introduce changes is via future releases.
\item To make published results reproducible, the exact versions of the code
that were used should be kept. This includes the values of all parameters
used including the precisions.
\end{itemize}

\subsection{Some caution for use}

These routines have been used and tested in a ChPT environment using
units in powers of GeV. Typical accuracies are set by default to relevant
and obtainable values for that case. In addition, there are often special cases
where the routines might not work, often due to $0/0$ or large cancelations.

Similar comments apply to the special functions 
included. They are sufficiently accurate for the purposes they were used for
originally and usually return values with a precision close to
double precision but this is not guaranteed.

In some cases, the large formulas have inherently large cancelations.
This might lead to degrading of precision in unexpected places. Use common
(scientific) sense to judge the quality of the results.

Finally, there are a number of internal functions and extensions
already present in the source code but not yet documented in this manual.
These might change and have not been tested as well as the documented ones.
In particular interfaces etc. might change.

\section{Files, installation and testroutines}

The package can be downloaded from \cite{chironsite}.
There are ready to install libaries there for some cases, but in general
it is better to compile it for your own system.
\cpp\ can have a large overhead in calling classes and functions
compared to \fortran. Therefore always compile the library with
optimization. The interfaces are as much as possible defined with the
keyword \mytt{const} to allow the compiler to optimize more efficiently.

\subsection{Files}
\label{files}
\index{Files}

The gzipped tarred file (\mytt{chiron.vvvv.tar.gz}) will produce
a directory \mytt{chiron.vvvv} with a number of subdirectories.
\mytt{vvvv} is version information. The created directory is called the
main directory in the remainder.

The main directory contains the files \mytt{COPYING}, \mytt{INSTRUCTIONS},
\mytt{GUIDELINES} and a \mytt{Makefile}.

The subdirectory \mytt{doc}
contains the documentation. The latest published article
about \textsc{CHIRON}, this manual (\mytt{manual.tex}), a list of files
(\mytt{filelist.txt}) and a summary of things added since earlier versions
(\mytt{Changelog.vvv.to.www.txt}).

The subdirectory \mytt{lib} will after compiling contain the compiled
libraries \mytt{libjbnumlib.a} and \mytt{libchiron.a}.

The subdirectory \mytt{include} contains all the needed header files.
The subdirectory \mytt{src} contains the source files. \mytt{test} contains
the testing and example programs. \mytt{testoutputs} contains the output
the testprograms should produce.

Typically for each subject \mytt{xxx} there are files \mytt{xxx.h},
\mytt{xxx.cc}, \mytt{testxxx.cc} and \mytt{testxxx.dat} in the
respective directories.

There are a few extra files around as well. These typically contain
inputs needed or large sets of constants.

\subsection{Installation}
\label{installation}
\index{Installation}

The main steps are to run make in the main directory.
This should produce the files \mytt{libjbnumlib.a} and \mytt{libchiron.a}
and also copy them to the \mytt{lib} subdirectory.
You might have to change the variables \mytt{CC}, \mytt{CFLAGS} and
\mytt{CFLAGTESTS}. \mytt{CC} should specify the \cpp\
compiler and the options to be used for everything.
\mytt{CFLAGS} can be used to specify additional options in compiling the
libraries and \mytt{CFLAGTESTS} to specify additional options
for the testing programs.

``make clean'' can be used to remove many of the files created during compiling.

The actual installation is by putting the contents of the \mytt{include}
directory somewhere in the include path of your compiler and the two
files \mytt{libjbnumlib.a} and \mytt{libchiron.a} somewhere in the library
path. For many \cpp\ compilers the paths are given in the environment
variables \mytt{CPLUS\_INCLUDE\_PATH} and \mytt{LIBRARY\_PATH} respectively.

\subsection{testroutines}
\label{testroutines}
\index{Testroutines}

For every file \mytt{xxx.h} and \mytt{xxx.cc} included for \mytt{chiron}
there is a testing/example code \mytt{testxxx.cc} in the subdirectory
\mytt{test}. These can be compiled using ``make testxxx'' in the main directory.
Executing the resulting file \mytt{a.out} should then produce output
identical (up to the precision specified and possible randomly generated
cases) to the file \mytt{testxxx.dat} in the subdirectory \mytt{testoutputs}.

\section{\newfunction{jbnumlib}}
\label{sec:jbnumlib}

\subsection{Complex numbers}

Complex numbers are defined via the standard \cpp\ library and an abbreviation
provided as\\
\mytt{typedef std::complex<double> dcomplex;}

All variables declared complex will be of the this type and referred
to as \newfunction{dcomplex} in the remainder.

\subsection{Special functions}

\subsubsection{\newfunction{jbdli2}}

\mytt{dcomplex jbdli2(const dcomplex x)}

Returns the complex dilogarithm or Spence function defined by
\begin{equation}
\mathrm{Li_2}(x) = -\int_0^1 dt \frac{\log(1-xt)}{t}\,,
\end{equation}
where it converges and analytic continuation. Cut defined on the
positive real axis from 1 to $\infty$.
Uses the properties of the dilogarithm to transform the argument
and then the Bernouilly series as described in \cite{'tHooft:1978xw}.

Defined in \mytt{jbnumlib.h} and implemented in \mytt{jbdli2.cc}.

\subsubsection{Bessel functions}

\paragraph{\newfunction{jbdbesi0}}

\mytt{double jbdbesi0(const double x)}

Returns the modified Bessel function $I_0$ for real values of the argument.
A simple port to \cpp\ of \cernlib\ routine DBESI0. 

Defined in \mytt{jbnumlib.h}, implemented in \mytt{jbdbesik.cc}.

\paragraph{\newfunction{jbdbesi1}}

\mytt{double jbdbesi1(const double x)}

Returns the modified Bessel function $I_1$ for real values of the argument.
A simple port to \cpp\ of \cernlib\ routine DBESI1. 

Defined in \mytt{jbnumlib.h}, implemented in \mytt{jbdbesik.cc}.

\paragraph{\newfunction{jbdbesk0}}

\mytt{double jbdbesk0(const double x)}

Returns the modified Bessel function $K_0$ for real values of the argument.
A simple port to \cpp\ of \cernlib\ routine DBESK0. 

Defined in \mytt{jbnumlib.h}, implemented in \mytt{jbdbesik.cc}.

\paragraph{\newfunction{jbdbesk1}}

\mytt{double jbdbesk1(const double x)}

Returns the modified Bessel function $K_1$ for real values of the argument.
A simple port to \cpp\ of \cernlib\ routine DBESK1. 

Defined in \mytt{jbnumlib.h}, implemented in \mytt{jbdbesik.cc}.

\paragraph{\newfunction{jbdbesk2}}

\mytt{double jbdbesk2(const double x)}

Returns the modified Bessel function $K_2$ for real values of the argument.
Uses the recursion relations for Bessel functions
and \mytt{jbdbesk0} and \mytt{jbdbesk1}.

Defined in \mytt{jbnumlib.h}, implemented in \mytt{jbdbesik.cc}.

\paragraph{\newfunction{jbdbesk3}}

\mytt{double jbdbesk3(const double x)}

Returns the modified Bessel function $K_3$ for real values of the argument.
Uses the recursion relations for Bessel functions
and \mytt{jbdbesk0} and \mytt{jbdbesk1}.

Defined in \mytt{jbnumlib.h}, implemented in \mytt{jbdbesik.cc}.

\paragraph{\newfunction{jbdbesk4}}

\mytt{double jbdbesk4(const double x)}

Returns the modified Bessel function $K_4$ for real values of the argument.
Uses the recursion relations for Bessel functions
and \mytt{jbdbesk0} and \mytt{jbdbesk1}.

Defined in \mytt{jbnumlib.h}, implemented in \mytt{jbdbesik.cc}.


\subsubsection{Theta and related functions}

\paragraph{\newfunction{jbdtheta30}}

\mytt{double jbdtheta30(const double q)}

Returns the value of the function
\begin{equation}
\label{deftheta30}
\theta_{30}(q) = 1+2\sum_{n=1,\infty} q^{(n^2)}
= \sum_{n=-\infty,\infty} q^{(n^2)}\,.
\end{equation}
This function is related to the third Jacobi theta function.
For small $q$ the summation in (\ref{deftheta30}) is used directly.
For larger $q$ the identity
\begin{equation}
\label{theta30relation}
\theta_{30}(q) = \sqrt{\frac{\lambda}{\pi}}\theta_{30}\left(e^{-\lambda}\right)
\end{equation}
with $\lambda = \pi^2/|\log(q)|$ is used instead.
This is related to the modular invariance for the higher dimensional case.
Precision can be judged by comparing the two series to each other.
Same idea as used in the \cernlib\ routine \mytt{DTHETA}.

Defined in \mytt{jbnumlib.h}, implemented in \mytt{jbdtheta30.cc}.

\paragraph{\newfunction{jbdtheta30m1}}

\mytt{double jbdtheta30m1(const double q)}

Returns the value of the function
\begin{equation}
\label{deftheta30m1}
\theta_{30}(q)-1 = 2\sum_{n=1,\infty} q^{(n^2)}
=\sum_{n\in\mathbb{Z},n\ne0} q^{(n^2)}\,.
\end{equation}
Implementation as for \mytt{jbdtheta30} but without the 1.
Especially for small $q$ often needed to keep accuracy in the finite volume
applications in ChPT.

Defined in \mytt{jbnumlib.h}, implemented in \mytt{jbdtheta30m1.cc}.

\paragraph{\newfunction{jbdtheta32}}

\mytt{double jbdtheta32(const double q)}

Returns the value of the function
\begin{equation}
\label{deftheta32}
\theta_{32}(q) = 2\sum_{n=1,\infty} q^{(n^2)}
= \sum_{n=-\infty,\infty}n^2 q^{(n^2)}= q\frac{d}{dq}\theta_{30}(q)\,.
\end{equation}
For small $q$ the summation in (\ref{deftheta32}) is used directly.
For larger $q$ the derivative of the right-hand-side of the
identity (\ref{theta30relation}) is used.

Defined in \mytt{jbnumlib.h}, implemented in \mytt{jbdtheta32.cc}.

\paragraph{\newfunction{jbdtheta34}}

\mytt{double jbdtheta34(const double q)}

Returns the value of the function
\begin{equation}
\label{deftheta34}
\theta_{34}(q) = \sum_{n=1,\infty}n^4 q^{(n^2)}
= \sum_{n=-\infty,\infty} n^4 q^{(n^2)}= \left(q\frac{d}{dq}\right)^2\theta_{30}(q)\,.
\end{equation}
For small $q$ the summation in (\ref{deftheta34}) is used directly.
For larger $q$ the appropriate derivative of the right-hand-side of the
identity (\ref{theta30relation}) is used.

Defined in \mytt{jbnumlib.h}, implemented in \mytt{jbdtheta34.cc}.

\paragraph{\newfunction{jbdtheta3}}

\mytt{double jbdtheta3(const double u, const double q)}

Returns the value of the function
\begin{equation}
\label{deftheta3}
\theta_{3}(u,q) = 1+2\sum_{n=1,\infty} q^{(n^2)} \cos(2\pi n u)
= \sum_{n=-\infty,\infty}  q^{(n^2)} e^{i2\pi n u}\,.
\end{equation}
For small $q$ the summation in (\ref{deftheta3}) is used directly.
For larger $q$ the expansion after using the relation
\begin{equation}
\label{theta3relation}
  \theta_3(u,q) = \sqrt{\pi/|\log(q)|}\exp(-\pi^2 u^2/|\log(q)|)
	\theta_3(-iu \pi/|\log(q)|,\exp(-\pi^2/|\log(q)|))
\end{equation}
is used.

Defined in \mytt{jbnumlib.h}, implemented in \mytt{jbdtheta3.cc}.

\paragraph{\newfunction{jbderivutheta3}}

\mytt{double jbderivutheta3(const double u, const double q)}

Returns the value of the function
\begin{equation}
\label{deftheta3u}
\frac{\partial}{\partial u}\theta_{3}(u,q) = -4\pi\sum_{n=1,\infty} q^{(n^2)} \sin(2\pi n u)
= i2\pi\sum_{n=-\infty,\infty}  q^{(n^2)} e^{i2\pi n u}\,.
\end{equation}
For small $q$ the summation in (\ref{deftheta3u}) is used directly.
For larger $q$ the appropriate derivative of the relation
(\ref{theta3relation}) is used.

Defined in \mytt{jbnumlib.h}, implemented in \mytt{jbderivutheta3.cc}.

\paragraph{\newfunction{jbderiv2utheta3}}

\mytt{double jbderiv2utheta3(const double u, const double q)}

Returns the value of the function
\begin{equation}
\label{deftheta3u2}
\frac{\partial^2}{\partial u^2}\theta_{3}(u,q) = -8\pi^2\sum_{n=1,\infty} q^{(n^2)} \cos(2\pi n u)
= -4\pi^2\sum_{n=-\infty,\infty}  q^{(n^2)} e^{i2\pi n u}\,.
\end{equation}
For small $q$ the summation in (\ref{deftheta3u2}) is used directly.
For larger $q$ the appropriate derivative of the relation
(\ref{theta3relation}) is used.

Defined in \mytt{jbnumlib.h}, implemented in \mytt{jbderiv2utheta3.cc}.


\paragraph{\newfunction{jbdtheta2d0}}

\mytt{double jbdtheta2d0(const double a, const double b, const double c)}

Returns the value of the function
\begin{equation}
\label{deftheta2d0}
\theta^{(2)}_0(a,b,c) = \sum_{n_1,n_2=-\infty,\infty}
e^{-an_1^2-bn_2^2-c(n_1-n_2)^2}\,.
\end{equation}
There are many higher-dimensional generalizations of the Jacobi theta functions.
The modular invariance properties of these are discussed in App.~B
of \cite{Bijnens:2013doa} and are used in the evaluation to speed
up the calculation. It should be noted that $\theta^{(2)}_0(a,b,c)$ is
fully symmetric in $a,b,c$.

Defined in \mytt{jbnumlib.h} and implemented in \mytt{jbdtheta2d0.cc}. 

\paragraph{\newfunction{jbdtheta2d0m1}}

\mytt{double jbdtheta2d0m1(const double a, const double b, const double c)}

Returns the value of the function
\begin{equation}
\label{deftheta2d0m1}
\theta^{(2)}_0(a,b,c) -1 
= \sum_{n_1,n_2=-\infty,\infty}
e^{-an_1^2-bn_2^2-c(n_1-n_2)^2}-1
= \sum_{\substack{n_1,n_2\in\mathbb{Z}\\(n_1,n_2)\ne(0,0)}}
e^{-an_1^2-bn_2^2-c(n_1-n_2)^2}-1\,.
\end{equation}
Method as in \mytt{jbdtheta2d0} but the 1 removed, more accurate for small
$a,b,c$ as often needed in finite volume ChPT.

Defined in \mytt{jbnumlib.h} and implemented in \mytt{jbdtheta2d0m1.cc}. 

\paragraph{\newfunction{jbdtheta2d02}}

\mytt{double jbdtheta2d02(const double a, const double b, const double c)}

Returns the value of the function
\begin{equation}
\label{deftheta2d02}
\theta^{(2)}_{02}(a,b,c) = \sum_{n_1,n_2=-\infty,\infty} n_1^2
e^{-an_1^2-bn_2^2-c(n_1-n_2)^2}
=-\frac{\partial}{\partial a}\theta^{(2)}_0(a,b,c)\,.
\end{equation}
It should be noted that $\theta^{(2)}_{02}(a,b,c)$ is
symmetric in $b,c$. Method similar to \mytt{jbdtheta2d0}.

Defined in \mytt{jbnumlib.h} and implemented in \mytt{jbdtheta2d02.cc}. 

\subsection{Integration routines}

\subsubsection{One dimension, real}

The interface of these routines is identical so they can be simply
interchanged.
For most problems the speed decrases as \mytt{jbdquad15}, \mytt{jbdquad21},
\mytt{jbdgauss2}, \mytt{jbdgauss} but this is somewhat dependent on the function
integrated and the precision requested.

The routines do not use the endpoints so an integrable singularity at the
endpoint can be done but an integrand transformation that removes the
singularity will lead to a much better performance.

An example program that shows the relative speeds is in
\newfunction{testintegralsreal.cc}.

\paragraph{\newfunction{jbdgauss}}

\mytt{double jbdgauss(f,a,b,eps)}

\mytt{f}: \mytt{double (*f)(const double x)} The double precision
function to be integrated over.

\mytt{a,b,eps}: \mytt{const double}

\mytt{a}: Lower limit of integration.

\mytt{b}: Upper limit of integration.

\mytt{eps}: Precision attempted to be reached: relative precision if absolute
value of the integral is above 1, otherwise absolute precision.

Subroutine translated from the \cernlib\ routine \mytt{DGAUSS}. Uses
8 and 16 point Gaussian rules with the 16 point for the estimate
and the difference for the error estimate. Adaptive with a subdivision strategy.

Defined in \mytt{jbnumlib.h}, implemented in \mytt{jbdgauss.cc}.

\paragraph{\newfunction{jbdgauss2}}

\mytt{double jbdgauss2(f,a,b,eps)}

\mytt{f}: \mytt{double (*f)(const double x)} The double precision
function to be integrated over.

\mytt{a,b,eps}: \mytt{const double}

\mytt{a}: Lower limit of integration.

\mytt{b}: Upper limit of integration.

\mytt{eps}: Precision attempted to be reached: relative precision if absolute
value of the integral is above 1, otherwise absolute precision.

Uses 8 and 16 point Gaussian rules with the 16 point for the estimate
and the difference for the error estimate. Adaptive with a subdivision strategy.
Very similar to \mytt{jbdgauss} but the subdivision strategy is more
appropriate for high precision.

Defined in \mytt{jbnumlib.h}, implemented in \mytt{jbdgauss2.cc}.

\paragraph{\newfunction{jbdquad15}}

\mytt{double jbdquad15(f,a,b,eps)}

\mytt{f}: \mytt{double (*f)(const double x)} The double precision
function to be integrated over.

\mytt{a,b,eps}: \mytt{const double}

\mytt{a}: Lower limit of integration.

\mytt{b}: Upper limit of integration.

\mytt{eps}: Precision attempted to be reached: relative precision if absolute
value of the integral is above 1, otherwise absolute precision.

Uses 15 point Gauss-Kronrod rule for the estimate
and the difference with the embedded 7 point Gauss rule for
the error estimate. Adaptive with a subdivision strategy
appropriate for high precision.

Defined in \mytt{jbnumlib.h}, implemented in \mytt{jbdquad15.cc}.

\paragraph{\newfunction{jbdquad21}}

\mytt{double jbdquad21(f,a,b,eps)}

\mytt{f}: \mytt{double (*f)(const double x)} The double precision
function to be integrated over.

\mytt{a,b,eps}: \mytt{const double}

\mytt{a}: Lower limit of integration.

\mytt{b}: Upper limit of integration.

\mytt{eps}: Precision attempted to be reached: relative precision if absolute
value of the integral is above 1, otherwise absolute precision.

Uses 21 point Gauss-Kronrod rule for the estimate
and the difference with the embedded 10 point Gauss rule for
the error estimate. Adaptive with a subdivision strategy
appropriate for high precision.

Defined in \mytt{jbnumlib.h}, implemented in \mytt{jbdquad21.cc}.


\subsubsection{One dimension, real with singularity}

The interface of these routines is identical so they can be simply
interchanged.
For most problems the speed decrases as \mytt{jbdquad15} or \mytt{jbdquad21},
\mytt{jbdgauss2}, \mytt{jbdgauss} but this is somewhat dependent on the function
integrated and the precision requested.

An example program that shows the relative speeds is in
\newfunction{testintegralsrealsingular.cc}.

\paragraph{\newfunction{jbdcauch}}

\mytt{double jbdcauch(f,a,b,s,eps)}

\mytt{f}: \mytt{double (*f)(const double x)} The double precision
function to be integrated over.

\mytt{a,b,s,eps}: \mytt{const double}

\mytt{a}: Lower limit of integration.

\mytt{b}: Upper limit of integration.

\mytt{s}: Place of the singularity.

\mytt{eps}: Precision attempted to be reached: relative precision if absolute
value of the integral is above 1, otherwise absolute precision.

Subroutine translated from the \cernlib\ routine \mytt{DCAUCH}.Uses
8 and 16 point Gaussian rules with the 16 point for the estimate
and the difference for the error estimate. Adaptive with a subdivision strategy.
Integrates symmetrically around
the singularity so it returns the integral in the sense of the
principal value prescription. Uses \mytt{jbdgauss}.

Defined in \mytt{jbnumlib.h}, implemented in \mytt{jbdcauch.cc}.

\paragraph{\newfunction{jbdcauch2}}

\mytt{double jbdcauch2(f,a,b,s,eps)}

\mytt{f}: \mytt{double (*f)(const double x)} The double precision
function to be integrated over.

\mytt{a,b,s,eps}: \mytt{const double}

\mytt{a}: Lower limit of integration.

\mytt{b}: Upper limit of integration.

\mytt{s}: Place of the singularity.

\mytt{eps}: Precision attempted to be reached: relative precision if absolute
value of the integral is above 1, otherwise absolute precision.

Subroutine translated from the \cernlib\ routine \mytt{DCAUCH}.Uses
8 and 16 point Gaussian rules with the 16 point for the estimate
and the difference for the error estimate. Adaptive with a subdivision strategy
more suitable for high precision.
Integrates symmetrically around
the singularity so it returns the integral in the sense of the
principal value prescription. Uses \mytt{jbdgauss2}.

Defined in \mytt{jbnumlib.h}, implemented in \mytt{jbdcauch2.cc}.

\paragraph{\newfunction{jbdsing15}}

\mytt{double jbdsing15(f,a,b,s,eps)}

\mytt{f}: \mytt{double (*f)(const double x)} The double precision
function to be integrated over.

\mytt{a,b,s,eps}: \mytt{const double}

\mytt{a}: Lower limit of integration.

\mytt{b}: Upper limit of integration.

\mytt{s}: Place of the singularity.

\mytt{eps}: Precision attempted to be reached: relative precision if absolute
value of the integral is above 1, otherwise absolute precision.

Subroutine similar to \mytt{jbdcauch2}
but uses a Gauss-Kronrod 15 point rule for the estimate
and the difference withe embedded 7 point Gauss rule for the error estimate.
Adaptive with a subdivision strategy
more suitable for high precision.
Integrates symmetrically around
the singularity so it returns the integral in the sense of the
principal value prescription. Uses \mytt{jbdquad15}.

Defined in \mytt{jbnumlib.h}, implemented in \mytt{jbdsing15.cc}.

\paragraph{\newfunction{jbdsing21}}

\mytt{double jbdsing21(f,a,b,s,eps)}

\mytt{f}: \mytt{double (*f)(const double x)} The double precision
function to be integrated over.

\mytt{a,b,s,eps}: \mytt{const double}

\mytt{a}: Lower limit of integration.

\mytt{b}: Upper limit of integration.

\mytt{s}: Place of the singularity.

\mytt{eps}: Precision attempted to be reached: relative precision if absolute
value of the integral is above 1, otherwise absolute precision.

Subroutine similar to \mytt{jbdcauch2}
but uses a Gauss-Kronrod 21 point rule for the estimate
and the difference withe embedded 10 point Gauss rule for the error estimate.
Adaptive with a subdivision strategy
more suitable for high precision.
Integrates symmetrically around
the singularity so it returns the integral in the sense of the
principal value prescription. Uses \mytt{jbdquad21}.

Defined in \mytt{jbnumlib.h}, implemented in \mytt{jbdsing21.cc}.

\subsubsection{One dimension, complex}

The interface of these routines is identical so they can be simply
interchanged.
For most problems the speed decrases as \mytt{jbwquad15} or \mytt{jbwquad21}
or \mytt{jbwgauss2}, \mytt{jbwgauss} but this is somewhat dependent on
the function
integrated and the precision requested.

An example program that shows the relative speeds is in
\newfunction{testintegralscomplex.cc}.

\paragraph{\newfunction{jbwgauss}}

\mytt{dcomplex jbwgauss(f,a,b,eps)}

\mytt{f}: \mytt{dcomplex (*f)(const dcomplex x)} The complex double precision
function to be integrated over.

\mytt{a,b}: \mytt{const dcomplex}

\mytt{a}: Lower endpoint of integration.

\mytt{b}: Upper endpoint of integration.

\mytt{eps}: \mytt{const double} Precision attempted to be reached:
relative precision if absolute
value of the integral is above 1, otherwise absolute precision.

Subroutine translated from the \cernlib\ routine \mytt{WGAUSS}. Uses
8 and 16 point Gaussian rules with the 16 point for the estimate
and the difference for the error estimate. Adaptive with a subdivision strategy.
The integration is the
lineintegral over the straight line between \mytt{a} and \mytt{b}.

Defined in \mytt{jbnumlib.h}, implemented in \mytt{jbwgauss.cc}.

\paragraph{\newfunction{jbwgauss2}}

\mytt{dcomplex jbwgauss2(f,a,b,eps)}

\mytt{f}: \mytt{dcomplex (*f)(const dcomplex x)} The complex double precision
function to be integrated over.

\mytt{a,b}: \mytt{const dcomplex}

\mytt{a}: Lower endpoint of integration.

\mytt{b}: Upper endpoint of integration.

\mytt{eps}: \mytt{const double} Precision attempted to be reached:
relative precision if absolute
value of the integral is above 1, otherwise absolute precision.

Subroutine translated from the \cernlib\ routine \mytt{WGAUSS}. Uses
8 and 16 point Gaussian rules with the 16 point for the estimate
and the difference for the error estimate. Adaptive with a subdivision strategy
better suited for high precision.
The integration is the
lineintegral over the straight line between \mytt{a} and \mytt{b}.

Defined in \mytt{jbnumlib.h}, implemented in \mytt{jbwgauss2.cc}.

\paragraph{\newfunction{jbwquad15}}

\mytt{dcomplex jbwquad15(f,a,b,eps)}

\mytt{f}: \mytt{dcomplex (*f)(const dcomplex x)} The complex double precision
function to be integrated over.

\mytt{a,b}: \mytt{const dcomplex}

\mytt{a}: Lower endpoint of integration.

\mytt{b}: Upper endpoint of integration.

\mytt{eps}: \mytt{const double} Precision attempted to be reached:
relative precision if absolute
value of the integral is above 1, otherwise absolute precision.

Subroutinre similar to \mytt{jbwgauss2} but uses a 15 point Gauss-Kronrod rule
for the estimate
and the difference with the embedded 7 point Gauss rule for the error estimate.
Adaptive with a subdivision strategy
better suited for high precision.
The integration is the
lineintegral over the straight line between \mytt{a} and \mytt{b}.

Defined in \mytt{jbnumlib.h}, implemented in \mytt{jbwquad15.cc}.

\paragraph{\newfunction{jbwquad21}}

\mytt{dcomplex jbwquad21(f,a,b,eps)}

\mytt{f}: \mytt{dcomplex (*f)(const dcomplex x)} The complex double precision
function to be integrated over.

\mytt{a,b}: \mytt{const dcomplex}

\mytt{a}: Lower endpoint of integration.

\mytt{b}: Upper endpoint of integration.

\mytt{eps}: \mytt{const double} Precision attempted to be reached:
relative precision if absolute
value of the integral is above 1, otherwise absolute precision.

Subroutinre similar to \mytt{jbwgauss2} but uses a 21 point Gauss-Kronrod rule
for the estimate
and the difference with the embedded 10 point Gauss rule for the error estimate.
Adaptive with a subdivision strategy
better suited for high precision.
The integration is the
lineintegral over the straight line between \mytt{a} and \mytt{b}.

Defined in \mytt{jbnumlib.h}, implemented in \mytt{jbwquad21.cc}.

\subsubsection{Two dimensions, real}

\paragraph{\newfunction{jbdad2}}

\mytt{double jbdad2(f,a,b,releps, relerr, ifail)}

\mytt{f}: \mytt{double (*f)(double x[])} The double precision
function to be integrated over, \mytt{x[0]} and \mytt{x[1]} contain
the values of the two variables to be integrated over.


\mytt{a}: \mytt{double a[]}  \mytt{a[0]} and \mytt{a[1]} are the lower limits of integration.

\mytt{b}: \mytt{double b[]}  \mytt{b[0]} and \mytt{b[1]} are the upper limits of integration.

\mytt{releps}: \mytt{const double} requested relative precision of the integral.

\mytt{relerr}: \mytt{double \&} returns the obtained relative precision
via a reference.

\mytt{ifail}: \mytt{int \&} returns an integer. Zero indicates success,
if not zero the routine did not obtain the requested precision..

The function does a two dimensional integration over a hypercube.
The underlying routine is \mytt{jbdadmul} which is
a simple port to \cpp\ of the \cernlib\ routine
 \mytt{DADMUL}. This in turn was based on \cite{radmulpaper}.

Defined in \mytt{jbnumlib.h}, implemented in \mytt{jbdadmul.cc}.

\subsubsection{Three dimensions, real}
\paragraph{\newfunction{jbdad3}}

\mytt{double jbdad3(f,a,b,releps, relerr, ifail)}

\mytt{f}: \mytt{double (*f)(double x[])} The double precision
function to be integrated over, \mytt{x[0]}, \mytt{x[1]} and  \mytt{x[2]}
contain the values of the three variables to be integrated over.

\mytt{a}: \mytt{double a[]}  \mytt{a[0]}, \mytt{a[1]} and \mytt{a[2]} are the lower limits of integration.

\mytt{b}: \mytt{double b[]}  \mytt{b[0]}, \mytt{b[1]} and \mytt{b[2]} are the upper limits of integration.

\mytt{releps}: \mytt{const double} requested relative precision of the integral.

\mytt{relerr}: \mytt{double \&} returns the obtained relative precision
via a reference.

\mytt{ifail}: \mytt{int \&} returns an integer. Zero indicates success,
if not zero the routine did not obtain the requested precision..

The function does a three dimensional integration over a hypercube.
The underlying routine is \mytt{jbdadmul} which is
a simple port to \cpp\ of the \cernlib\ routine
 \mytt{DADMUL}. This in turn was based on \cite{radmulpaper}.

Defined in \mytt{jbnumlib.h}, implemented in \mytt{jbdadmul.cc}.

\section{Chiral Perturbation Theory}

The classic papers introducing ChPT are
\cite{Weinberg:1978kz,Gasser:1983yg,Gasser:1984gg}.
References to lectures and introductions can be found in \cite{webpage}.
Areview at two-loop order is \cite{Bijnens:2006zp}.
The notation used here correspond to the notation introduced by Gasser
and Leutwyler, $B,F,l_i^r,B_0$ \cite{Gasser:1983yg} and
$F_0L_i^r$ \cite{Gasser:1984gg} for the two and three flavour case.
In general the decay constants are defined with a normalization of
$F_\pi\approx 92~$MeV. The coupling constants in the higher order Lagrangians
are usually referred to as low-energy constants (LECs). Power counting
is the usual dimensional counting with orders referred to as $p^n$
with alternatively $p^2$ or lowest-order (LO), $p^4$ or next-to-leading-order
(NLO) and $p^6$ or next-to-next-to-leading order (NNLO).

\section{Data structures}

This section describes a number of classes to deal with input parameters
and LECs. The default value mechanism of \cpp\ is used to give them
initial values if not specified. These are visible below as ``\mytt=value''
in the definitions.

\subsection{Three flavour ChPT}

\subsubsection{Class: \newfunction{physmass}}

\mytt{physmass(mpiin=0.135,mkin=0.495,metain=0.548,fpiin=0.0922,muin=0.77)}

\mytt{mpiin,mkin,metain,fpiin,muin: const double}

Private data: \mytt{double mpi,mk,meta,fpi,mu}

Physical quantities: pion, kaon and eta mass, pion-decay constant and
subtraction scale $\mu$.

Relevant physical case: three flavour ChPT, isospin limit.\\

Input member functions:\\
\mytt{void \newfunction{setmpi}(const double mpiin=0.135)}\\
\mytt{void \newfunction{setmk}(const double mkin=0.495)}\\
\mytt{void \newfunction{setmeta}(const double metain=0.548)}\\
\mytt{void \newfunction{setfpi}(const double fpiin=0.0922)}\\
\mytt{void \newfunction{setmu}(const double muiin=0.77)}\\

Output member functions exist in two varieties. Those that return
all or a subset of values using references or those that
return one value as the function value.\\
\mytt{void \newfunction{out}(double \&mpiout, double \&mkout, double \&metaout, double \&fpiout,\\\hspace*{1cm} double \&muout)}\\
\mytt{double \newfunction{getmpi}(void)}\\
\mytt{double \newfunction{getmk}(void)}\\
\mytt{double \newfunction{getmeta}(void)}\\
\mytt{double \newfunction{getfpi}(void)}\\
\mytt{double \newfunction{getmu}(void)}\\


Operators defined: \newfunction{<<}, \newfunction{>>} and \newfunction{==}.

\mytt{<<} and \mytt{>>} are  defined such that output and input
streams work as expected. The input stream should be exactly in the format
provided by the output stream.

\mytt{==} checks for equality within relative precision of $10^{-7}$.
An error will occur if any of the data members is zero.\\

Defined in \mytt{inputs.h}, implemented in \mytt{inputs.cc},
examples of use in \mytt{testinputs.cc}.

\subsubsection{Class: \newfunction{lomass}}

\mytt{lomass(mp0in=0.135, mk0in=0.495, f0in=0.090, muin=0.77)}

\mytt{mp0in,mk0in,f0in,muin: const double}

\mytt{lomass(const quarkmass mass)}



Private data: \mytt{double mp0,mk0,f0,mu}

Physical quantities: lowest order pion mass, lowest order kaon mass, lowest
order pion-decay constant and subtraction scale $\mu$.

Relevant physical case: three flavour ChPT, isospin limit.\\

The constructor from a \mytt{quarkmass} is provided such that conversions can be used.\\
Input member functions:\\
\mytt{void \newfunction{setmp0}(const double mp0in=0.135)}
\mytt{void \newfunction{setmk0}(const double mk0in=0.495)}
\mytt{void \newfunction{setf0}(const double f0in=0.09)}
\mytt{void \newfunction{setmu}(const double muin=0.77)}\\

Output member functions exist in two varieties. Those that return
all or a subset of values using references or those that
return one value as the function value.\\
\mytt{void \newfunction{out}(double \&mp0out, double \&mk0out, double \&f0out, double \&muout)}\\
\mytt{double \newfunction{getmp0}(void)}\\
\mytt{double \newfunction{getmk0}(void)}\\
\mytt{double \newfunction{getf0}(void)}\\
\mytt{double \newfunction{getmu}(void)}\\

Operators defined: \newfunction{<<}, \newfunction{>>} and \newfunction{==}.

\mytt{<<} and \mytt{>>} are  defined such that output and input
streams work as expected. The input stream should be exactly in the format
provided by the output stream.

\mytt{==} checks for equality within relative precision of $10^{-7}$.
An error will occur if any of the data members is zero.\\

Defined in \mytt{inputs.h}, implemented in \mytt{inputs.cc},
examples of use in \mytt{testinputs.cc}.


\subsubsection{Class: \newfunction{quarkmass}}

\mytt{quarkmass(B0mhatin=0.01, B0msin=0.25, f0in=0.090, muin=0.77)}

\mytt{B0mhatin,B0msin,f0in,muin: const double}

\mytt{quarkmass(const lomass mass)}

Private data: \mytt{double B0mhat,B0ms,f0,mu}

Physical quantities: $B_0\hat m, B_0 m_s$, lowest
order pion-decay constant and subtraction scale $\mu$.\\
The quantities $B_0 m\hat m$ and $B_0 m_s$ are the LEC $B_0$ \cite{Gasser:1984gg} multiplied by the up-down quark mass and strange quark mass respectively.
These are independent of the QCD scale. The lowest order pion and kaon masses
are given by $m_{\pi\,\text{LO}}=\sqrt{2B_0 \hat m}$ and
$m_{K\,\text{LO}}=\sqrt{B_0 (\hat m+m_s)}$

Relevant physical case: three flavour ChPT, isospin limit.\\

The constructor from a \mytt{lomass} is provided such that conversions can be used.\\
Input member functions:\\
\mytt{void \newfunction{setB0mhat}(const double B0mhatin=0.01)}\\
\mytt{void \newfunction{setB0ms}(const double B0msin=0.25)}\\
\mytt{void \newfunction{setf0}(const double f0in=0.09)}\\
\mytt{void \newfunction{setmu}(const double muin=0.77)}\\


Output member functions exist in two varieties. Those that return
all or a subset of values using references or those that
return one value as the function value.\\
\mytt{void \newfunction{out}(double \&B0mhatout, double \&B0msout, double \&f0out, double \&muout)}\\
\mytt{double \newfunction{getB0mhat}(void)}\\
\mytt{double \newfunction{getB0ms}(void)}\\
\mytt{double \newfunction{getf0}(void)}\\
\mytt{double \newfunction{getmu}(void)}\\

Operators defined: \newfunction{<<}, \newfunction{>>} and \newfunction{==}.

\mytt{<<} and \mytt{>>} are  defined such that output and input
streams work as expected. The input stream should be exactly in the format
provided by the output stream.

\mytt{==} checks for equality within relative precision of $10^{-7}$.
An error will occur if any of the data members is zero.\\

Defined in \mytt{inputs.h}, implemented in \mytt{inputs.cc},
examples of use in \mytt{testinputs.cc}.

\subsubsection{NLO LECs: Class \newfunction{Li}}

\mytt{Li(l1r=0.,l2r=0.,l3r=0.,l4r=0.,l5r=0.,l6r=0.,l7r=0.,l8r=0.,l9r=0.,l10r=0.,\\\hspace*{0.7cm}h1r=0.,h2r=0.,mu=0.77,Name="nameless Li")}

\mytt{const double: l1r,\ldots,l10r,h1r,h2r,mu}

\mytt{const string: Name}


Private data: \mytt{double L1r,L2r,L3r,L4r,L5r,L6r,L7r,L8r,L9r,L10r,H1r,H2r,mu}
and \mytt{string name}

Physical quantities the 12 LECs, $L_i^r,H_i^r$
(of which two are so-called contact terms)
of three-flavour ChPT as introduced in \cite{Gasser:1984gg} and the subtraction
scale $\mu$.

Relevant physical case: three flavour ChPT\\

Input member functions:\\
\mytt{void \newfunction{setli}(const int n, const double lin)}\\
\mytt{void setli(const double lin, const int n)}\\
Set the value of the LECs with index $n$. $n=11,12$ correspond to
$H_1^r,H_2r$.\\
\mytt{\newfunction{setmu}(const double muin)}\\
Sets the scale $\mu$ to the value \mytt{muin}. This does \emph{not}
change the LECs, for that use \mytt{changescale}.\\
\mytt{\newfunction{setname}(const string namein)}
Sets the name of the set of LECs.\\

Output member functions:\\
\mytt{double \newfunction{out}(const int n)} returns the value of the n'th LEC.\\
\mytt{void out} exists in many varieties, 13 double references and a string
returning all private data, 13 double references returning all LECS and the subtraction scale, 12 double references returning all LECs,
11 double references returning $L_1^r,\ldots,L_{10}^r$ and the subtraction scale
and 10 double references returning $L_1^r,\ldots,L_{10}^r$.\\

\mytt{void \newfunction{changescale}(const double newmu)}\\
This changes the subtraction scale to the new value given by \mytt{muin} and
changes the LECs according to the running derived
in \cite{Gasser:1984gg}.\\

Operators defined: \newfunction{<<}, \newfunction{>>}, \newfunction{$+$}, \newfunction{$-$} and \newfunction{*}.

\mytt{<<} and \mytt{>>} are  defined such that output and input
streams work as expected. The input stream should be exactly in the format
provided by the output stream.\\
\mytt{*} allows to multiply an \mytt{Li} by a \mytt{double} in either order.
The resulting value has all LECs multiplied by the value of the \mytt{double}.\\
\mytt{$+$} and \mytt{$-$} allow to add or subtract set of LECs.
The resulting value of all LECs is the sum respectively the difference.
A warning is printed of the scales are different.

Extra functions:\\
\mytt{Li \newfunction{Lirandom}(void)}\\
\mytt{Li \newfunction{LirandomlargeNc}(void)}\\
\mytt{Li \newfunction{LirandomlargeNc2}(void)}\\
These return a set of random NLO LECs.
The values are uniformly distributed between $\pm1/(16\pi^2)$
for \mytt{Lirandom}. \mytt{LirandomlargeNc}
does the same except that it leaves $L^r_4, L^r_6$ and $L^r_7$ zero.
\mytt{LirandomlargeNc2} does the same but
$L^r_4, L^r_6$ and $L^r_7$
get a random value between $\pm(1/3)/(16\pi^2)$.
The random numbers are generated using the
system generator \mytt{rand()} so initializing using something like
\mytt{srand(time(0))}.
These latter functions were used in the random walks in the $L^r_i$ in \cite{Bijnens:2011tb}.\\
Defined in \mytt{Li.h}, implemented in \mytt{Li.cc},
examples of use in \mytt{testLi.cc}.\\
In the subdirectory \mytt{test} there is a file \newfunction{LiCiBE14.dat}
that contains the last fit of the LECs \cite{Bijnens:2014lea}.

\subsubsection{NNLO LECs: Class \newfunction{Ci}}

\mytt{Ci(Cr, mu=0.77,Name="nameless Ci")}\\
\mytt{Ci(mu=0.77,Name="nameless Ci")}

\mytt{const double: mu}

\mytt{const string: Name}

Private data: \mytt{double Cr[95], mu}
and \mytt{string name}

Physical quantities the 94 LECs, $C_i^r$
(of which four are so-called contact terms)
of three-flavour ChPT as introduced in \cite{Bijnens:1999sh,Bijnens:1999hw}
and the subtraction
scale $\mu$. The $C_i^r$ are the dimensionless version.
Scale to the dimensionfull version with appropriate powers of $F_0$
but in practice normally with $F_\pi$.

Relevant physical case: three flavour ChPT\\

Input member functions:\\
\mytt{void \newfunction{setci}(const int n, const double lin)}\\
\mytt{void setci(const double lin, const int n)}\\
Set the value of the LECs with index $n$.\\
\mytt{\newfunction{setmu}(const double muin)}\\
Sets the scale $\mu$ to the value \mytt{muin}. This does \emph{not}
change the LECs, for that use \mytt{changescale}.\\
\mytt{\newfunction{setname}(const string namein)}
Sets the name of the set of LECs.\\

Output member functions:\\
\mytt{double \newfunction{out}(const int n)} returns the value of the n'th LEC.\\
\mytt{void out} exists in many varieties, with a \mytt{double Cit[95]},
a double reference and a string
returning all private data,  a \mytt{double Cit[95]},
a double reference returning all LECS and the subtraction scale,
and  a \mytt{double Cit[95]} returning the LECs only.\\

\mytt{void \newfunction{changescale}(const double newmu, Li \& Liin)}\\
\mytt{void changescale(Li \& Liin, const double newmu)}\\
This changes the subtraction scale to the new value given by \mytt{muin} and
changes the LECs according to the running derived
in \cite{Bijnens:1999hw}. Note that it changes the scale of the values of the
NLO LECs $L_i^r$ in \mytt{Liin} as well.\\

Operators defined: \newfunction{<<}, \newfunction{>>}, \newfunction{$+$}, \newfunction{$-$} and \newfunction{*}.

\mytt{<<} and \mytt{>>} are  defined such that output and input
streams work as expected. The input stream should be exactly in the format
provided by the output stream.\\
\mytt{*} allows to multiply a \mytt{Ci} by a \mytt{double} in either order.
The resulting value has all LECs multiplied by the value of the \mytt{double}.\\
\mytt{$+$} and \mytt{$-$} allow to add or subtract set of LECs.
The resulting value of all LECs is the sum respectively the difference.
A warning is printed of the scales are different.

Extra functions:\\
\mytt{Ci \newfunction{Cirandom}(void)}\\
\mytt{Ci \newfunction{CirandomlargeNc}(void)}\\
\mytt{Ci \newfunction{CirandomlargeNc2}(void)}\\
These return a set of random NNLO LECs.
The values are uniformly distributed between $\pm1/(16\pi^2)^2$
for \mytt{Cirandom}. \mytt{CirandomlargeNc}
does the same except that it leaves all LECs that are not single
trace terms zero.
\mytt{CirandomlargeNc2} does the same but
the non-single-trace terms get a LEC with
a random value between $\pm(1/3)/(16\pi^2)^2$.
The random numbers are generated using the
system generator \mytt{rand()} so initializing using something like
\mytt{srand(time(0))}.
These latter functions were used in the random walks in the $C^r_i$ in \cite{Bijnens:2011tb}.\\
Defined in \mytt{Ci.h}, implemented in \mytt{Ci.cc},
examples of use in \mytt{testCi.cc}.

\subsection{$n_F$ flavour ChPT}

\subsubsection{Class: \newfunction{quarkmassnf}}

\mytt{quarkmassnf(f0in=0.090, muin=0.77, nqin=3)}\\
\mytt{quarkmassnf(const vector<double> B0mqin,f0in=0.090, muin=0.77)}\\
\mytt{quarkmassnf(const lomassnf mass)}


\mytt{const double: f0in,muin}\\
\mytt{const int: nfin}\\
\mytt{const vector<double> B0mqin}\\
\mytt{const lomassnf mass}\\
Private data: \mytt{vector<double> B0mq},\mytt{double f0,mu},
\mytt{int nq}.

Physical quantities: $B_0 m_i$ quark masses multiplied by $B_0$,
lowest order (pion-)decay constant and subtraction scale $\mu$.
The quantities $B_0 m_i$ are the $n_q=\mytt{nq}$
quark masses multiplied by the LEC $B_0$ \cite{Gasser:1984gg},
$B_0$ for the relevant number of quarks $n_F$.
These are independent of the QCD scale. The lowest order charged kaon mass
is given by $m_{K\,\text{LO}}=\sqrt{B_0 (m_u+m_s)}$

Relevant physical case: $n_F$ flavour ChPT, possibly partially quenched
where we need $n_q$ different quark masses. The masses are referred to
as 1,\ldots,nq (i.e. the counting does not start with 0).

\mytt{void \newfunction{setB0mq}(const double B0miin,const int i)}\\
\mytt{void \newfunction{setB0mq}(const int i, const double B0miin=0.)}\\
\mytt{void \newfunction{setB0mq}(const vector<double> B0mqin)}\\
\mytt{void \newfunction{setf0}(const double f0in=0.09)}\\
\mytt{void \newfunction{setmu}(const double muin=0.77)}\\


Output member functions exist in two varieties. Those that return
all or a subset of values using references or those that
return one value as the function value.\\
\mytt{void \newfunction{out}(vector<double> \&B0mq) const;}\\
\mytt{void \newfunction{out}(vector<double> \&B0mq, double \&f0out, double \&muout) const;}\\
\mytt{void \newfunction{out}(vector<double> \&B0mq, double \&f0out, double \&muout, int \&nq) const;}\\
\mytt{int \newfunction{getnq}(void) const;}\\
\mytt{vector<double> \newfunction{getB0mq}(void) const;}\\
\mytt{double \newfunction{getB0mq}(const int i) const;}\\
\mytt{double \newfunction{getf0}(void) const;}\\
\mytt{double \newfunction{getmu}(void) const;}\\

Operators defined: \newfunction{<<}, \newfunction{>>} and \newfunction{==}.

\mytt{<<} and \mytt{>>} are  defined such that output and input
streams work as expected. The input stream should be exactly in the format
provided by the output stream.

\mytt{==} checks for equality within relative precision of $10^{-7}$.
An error will occur if any of the data members is zero.\\

Defined in \mytt{inputsnf.h}, implemented in \mytt{inputsnf.cc},
examples of use in \mytt{testinputsnf.cc}.

\subsubsection{Class: \newfunction{lomassnf}}

\mytt{lomassnf(f0in=0.090, muin=0.77, nmassin=3)}\\
\mytt{lomassnf(const vector<double> massin,f0in=0.090, muin=0.77)}\\
\mytt{lomassnf(const quarkmassnf mass)}

\mytt{const double: f0in,muin}\\
\mytt{const int: nfin}\\
\mytt{const vector<double> massin}\\
Private data: \mytt{vector<double> mass},\mytt{double f0,mu},
\mytt{int nmass}.

Physical quantities: $m_{ii}$ the lowest order meson
masses,
lowest order (pion-)decay constant and subtraction scale $\mu$.
The quantities $m_{ii}$ are the $\mytt{nmass}$
lowest order meson masses. They correspond to $m_{ii}=sqrt{B_0 m_i}$ with
the $\mytt{nmass}$
quark masses multiplied by the LEC $B_0$ \cite{Gasser:1984gg},
$B_0$ for the relevant number of quarks $n_F$.

Relevant physical case: $n_F$ flavour ChPT, possibly partially quenched
where we need $n_q$ different quark masses. The masses are referred to
as 1,\ldots,nq (i.e. the counting does not start with 0).

\mytt{void \newfunction{setmass}(const double massin,const int i)}\\
\mytt{void \newfunction{setmass}(const int i, const double massin=0.)}\\
\mytt{void \newfunction{setmass}(const vector<double> B0mqin)}\\
\mytt{void \newfunction{setf0}(const double f0in=0.09)}\\
\mytt{void \newfunction{setmu}(const double muin=0.77)}\\


Output member functions exist in two varieties. Those that return
all or a subset of values using references or those that
return one value as the function value.\\
\mytt{void \newfunction{out}(vector<double> \&massout) const;}\\
\mytt{void \newfunction{out}(vector<double> \&massout, double \&f0out, double \&muout) const;}\\
\mytt{void \newfunction{out}(vector<double> \&massout, double \&f0out, double \&muout, int \&nmass) const;}\\
\mytt{int \newfunction{getnmass}(void) const;}\\
\mytt{vector<double> \newfunction{getmass}(void) const;}\\
\mytt{double \newfunction{getmass}(const int i) const;}\\
\mytt{double \newfunction{getf0}(void) const;}\\
\mytt{double \newfunction{getmu}(void) const;}\\

Operators defined: \newfunction{<<}, \newfunction{>>} and \newfunction{==}.

\mytt{<<} and \mytt{>>} are  defined such that output and input
streams work as expected. The input stream should be exactly in the format
provided by the output stream.

\mytt{==} checks for equality within relative precision of $10^{-7}$.
An error will occur if any of the data members is zero.\\

Defined in \mytt{inputsnf.h}, implemented in \mytt{inputsnf.cc},
examples of use in \mytt{testinputsnf.cc}.


\subsubsection{NLO LECs: Class \newfunction{Linf}}

\mytt{Linf(l0r=0.,l1r=0.,l2r=0.,l3r=0.,l4r=0.,l5r=0.,l6r=0.,l7r=0.,l8r=0.,l9r=0.,\\\hspace*{0.7cm}l10r=0.,l11r=0.,h1r=0.,h2r=0.,mu=0.77,Name="nameless Linf",const int nfin=3)}

\mytt{const double: l0r,\ldots,l11r,h1r,h2r,mu}

\mytt{const string: Name}

\mytt{const int: nfin}


Private data: \mytt{double L0r,L1r,L2r,L3r,L4r,L5r,L6r,L7r,L8r,L9r,L10r,L11r,H1r,H2r,mu}, \mytt{int nf}
and \mytt{string name}

Physical quantities the 13 LECs, $L_{i=0,10}^r,H_i^r$
(of which two are so-called contact terms) and the extra equation
of motion term LEC $L_{11}^r$
of $n_F$-flavour ChPT as introduced in \cite{Bijnens:1999hw}
and the subtraction
scale $\mu$. The extra constant $L_{11}^r$ is added to be able to deal
with two-flavour partially quenched ChPT.

Relevant physical case: $n_F$ flavour ChPT and partially quenched ChPT\\

Input member functions:\\
\mytt{void \newfunction{setnf}(const int nfin)}\\
\mytt{void \newfunction{setli}(const int n, const double lin)}\\
\mytt{void setli(const double lin, const int n)}\\
Sets the value of the LECs with index $n$. $n=12,13$ correspond to
$H_1^r,H_2r$.\\
\mytt{\newfunction{setmu}(const double muin)}\\
Sets the scale $\mu$ to the value \mytt{muin}. This does \emph{not}
change the LECs, for that use \mytt{changescale}.\\
\mytt{\newfunction{setname}(const string namein)}
Sets the name of the set of LECs.\\

Output member functions:\\
\mytt{double \newfunction{out}(const int n)} returns the value of the n'th LEC.\\
\mytt{void out} exists in many varieties, 15 double references, a string
and an integer
returning all private data, 15 double references and an integer returning all LECS and the subtraction scale and the number of flavours, 13 double references
and an integer returning all LECs and $n_F$,
12 double and an integer references returning $L_1^r,\ldots,L_{11}^r$ and the subtraction scale and the number of flavours,
11 double and an integer references returning $L_1^r,\ldots,L_{11}^r$  and the number of flavours,
and 11 double references returning $L_1^r,\ldots,L_{11}^r$.\\
\mytt{int \newfunction{getnf}(void)} returns \mytt{nf}\\

\mytt{void \newfunction{changescale}(const double newmu)}\\
This changes the subtraction scale to the new value given by \mytt{muin} and
changes the LECs according to the running derived
in \cite{Gasser:1984gg}.\\

Operators defined: \newfunction{<<}, \newfunction{>>}, \newfunction{$+$}, \newfunction{$-$} and \newfunction{*}.

\mytt{<<} and \mytt{>>} are  defined such that output and input
streams work as expected. The input stream should be exactly in the format
provided by the output stream.\\
\mytt{*} allows to multiply an \mytt{Li} by a \mytt{double} in either order.
The resulting value has all LECs multiplied by the value of the \mytt{double}.\\
\mytt{$+$} and \mytt{$-$} allow to add or subtract set of LECs.
The resulting value of all LECs is the sum respectively the difference.
A warning is printed of the scales are different.

Extra functions:\\
\mytt{Linf \newfunction{Linfrandom}(void)}\\
This returns a set of random NLO LECs.
The values are uniformly distributed between $\pm1/(16\pi^2)$.
The random numbers are generated using the
system generator \mytt{rand()} so initializing using something like
\mytt{srand(time(0))}.\\

Defined in \mytt{Linf.h}, implemented in \mytt{Linf.cc},
examples of use in \mytt{testLinf.cc}.

\subsubsection{NNLO LECs: Class \newfunction{Ki}}

\mytt{Ki(Kr, mu=0.77,Name="nameless Ci",nfin=3)}\\
\mytt{Ci(mu=0.77,Name="nameless Ci")}

\mytt{const double: mu}

\mytt{const string: Name}

\mytt{const int: nfin}

Private data: \mytt{double Kr[116], mu}, \mytt{int nf}
and \mytt{string name}

Physical quantities the 115 LECs, $K_i^r$
(of which three are so-called contact terms)
of $n_F$-flavour ChPT as introduced in \cite{Bijnens:1999sh,Bijnens:1999hw}
and the subtraction
scale $\mu$. The $K_i^r$ are the dimensionless version.
Scale to the dimensionfull version with appropriate powers of $F_0$
but in practice normally with $F_\pi$.

Relevant physical case: $n_F$ flavour ChPT\\

Input member functions:\\
\mytt{void \newfunction{setki}(const int n, const double kin)}\\
\mytt{void setki(const double kin, const int n)}\\
Set the value of the LECs with index $n$.\\
\mytt{\newfunction{setmu}(const double muin)}\\
Sets the scale $\mu$ to the value \mytt{muin}. This does \emph{not}
change the LECs, for that use \mytt{changescale}.\\
\mytt{\newfunction{setname}(const string namein)}
Sets the name of the set of LECs.\\

Output member functions:\\
\mytt{double \newfunction{out}(const int n)} returns the value of the n'th LEC.\\
\mytt{void out} exists in many varieties, with a \mytt{double Kit[116]},
a double reference and a string
returning all private data,  a \mytt{double Kit[116]},
a double reference returning all LECS and the subtraction scale,
and  a \mytt{double Kit[116]} returning the LECs only.\\
\mytt{int \newfunction{getnf}(void)} returns \mytt{nf}\\

\mytt{void \newfunction{changescale}(const double newmu, Linf \& Liin)}\\
\mytt{void changescale(Linf \& Liin, const double newmu)}\\
This changes the subtraction scale to the new value given by \mytt{muin} and
changes the LECs according to the running derived
in \cite{Bijnens:1999hw}. Note that it changes the scale of the values of the
NLO LECs $L_i^r$ in \mytt{Liin} as well.\\

Operators defined: \newfunction{<<}, \newfunction{>>}, \newfunction{$+$}, \newfunction{$-$} and \newfunction{*}.

\mytt{<<} and \mytt{>>} are  defined such that output and input
streams work as expected. The input stream should be exactly in the format
provided by the output stream.\\
\mytt{*} allows to multiply a \mytt{Ci} by a \mytt{double} in either order.
The resulting value has all LECs multiplied by the value of the \mytt{double}.\\
\mytt{$+$} and \mytt{$-$} allow to add or subtract set of LECs.
The resulting value of all LECs is the sum respectively the difference.
A warning is printed of the scales are different.

Extra functions:\\
\mytt{Ki \newfunction{Kirandom}(void)}\\
This returns a set of random NNLO LECs.
The values are uniformly distributed between $\pm1/(16\pi^2)^2$.
The random numbers are generated using the
system generator \mytt{rand()} so initializing using something like
\mytt{srand(time(0))}.\\

Defined in \mytt{Ki.h}, implemented in \mytt{Ki.cc},
examples of use in \mytt{testKi.cc}.


\section{Loop integrals}
\index{Loop integrals}

Loop integrals are done with dimensional regularization and we use
the standard ChPT variant of $\overline{MS}$.
At one-loop order it was defined in \cite{Gasser:1983yg,Gasser:1984gg}.
The definition at two-loop order can be found in \cite{Bijnens:1999hw}.

We define for subtraction purposes:
\begin{align}
\label{defCC}
d = 4-2\epsilon,\quad C=\ln(4\pi)+1-\gamma,\quad
\lambda_0 = \frac{1}{\epsilon}+C,\quad\lambda_1=\lambda_0+C,\quad
\lambda_2=\lambda_0^2+C^2\,.
\end{align}
The $d$-dimensional Feynman integrals do not depend directly on the subtraction
scale. However, renormalization will always introduce the correct dependence.
We define the one-loop integrals multiplied by an extra factor
of $\mu^{2\epsilon}$ and the two-loop integrals with an extra factor
of $\mu^{4\epsilon}$. This introduces the $\mu$ dependence in the expressions
given below.

References are to places where the integrals are defined and/or the method used
elaborated.

\subsection{Tadpole or one-propagator integrals}
\label{tadpoles}
\index{Tadpoles}

These are defined by
\begin{align}
A(n,m^2) =\,& \frac{\mu^{4-d}}{i}\int \frac{d^d q}{\left(2\pi\right)^d}
\frac{1}{\left(q^2-m^2\right)^n}\,,
\nonumber\\
A(m^2,\mu^2) =\,& A(1,m^2)\,,\quad B(m^2,\mu^2) = A(2,m^2)\,,\quad C(m^2,\mu^2) = A(3,m^2)\,.
\end{align}
The expansions in $\epsilon$ are given by, see e.g. \cite{Amoros:1999dp},
\begin{align}
A(m^2,\mu^2) = \,&\frac{\lambda_0 m^2}{16\pi^2}+\overline A(m^2,\mu^2)
+\epsilon A^\epsilon(m^2,\mu^2) +\mathcal{O}(\epsilon^2)\,,
\nonumber\\
B(m^2,\mu^2) = \,&\frac{\lambda_0 }{16\pi^2}+\overline B(m^2,\mu^2)
+\epsilon B^\epsilon(m^2,\mu^2) +\mathcal{O}(\epsilon^2)\,,
\nonumber\\
C(m^2,\mu^2) = \,&\overline C(m^2,\mu^2)
+\epsilon C^\epsilon(m^2,\mu^2) +\mathcal{O}(\epsilon^2)\,.
\end{align}
The $\mathcal{O}(\epsilon)$ terms are further expanded as
\begin{align}
A^\epsilon(m^2,\mu^2) =\, & \frac{m^2}{16\pi^2}
\left(\frac{1}{2}C^2-C\log\frac{m^2}{\mu^2}\right)+\overline A^\epsilon(m^2,\mu^2)\,,
\nonumber\\
B^\epsilon(m^2,\mu^2) =\, & \frac{1}{16\pi^2}
\left(\frac{1}{2}C^2-C\log\frac{m^2}{\mu^2}-C\right)+\overline B^\epsilon(m^2,\mu^2)\,,
\nonumber\\
C^\epsilon(m^2,\mu^2) =\, & \frac{1}{16\pi^2}
\left(-\frac{C}{2m^2}\right)+\overline C^\epsilon(m^2,\mu^2)\,.
\end{align}
The analytical expressions are
\begin{align}
\overline A(m^2,\mu^2) &= \frac{-m^2}{16\pi^2}\log\frac{m^2}{\mu^2}\,
&  \overline A^\epsilon(m^2,\mu^2) &= \frac{m^2}{16\pi^2}
\left(\frac{1}{2}+\frac{\pi^2}{12}+\frac{1}{2}\log^2\frac{m^2}{\mu^2}\right)\,,
\nonumber\\
\overline B(m^2,\mu^2) &= \frac{1}{16\pi^2}\left(-1-\log\frac{m^2}{\mu^2}\right)\,
&  \overline B^\epsilon(m^2,\mu^2) &= \frac{1}{16\pi^2}
\left(\frac{1}{2}+\frac{\pi^2}{12}+\frac{1}{2}\log^2\frac{m^2}{\mu^2}+\log\frac{m^2}{\mu^2}\right)\,,
\nonumber\\
\overline C(m^2,\mu^2) &= \frac{1}{16\pi^2}\frac{-1}{2m^2}\,
&  \overline C^\epsilon(m^2,\mu^2) &= \frac{1}{16\pi^2}
\left(\frac{1}{2m^2}+\frac{1}{2m^2}\log\frac{m^2}{\mu^2}\right)\,,
\end{align}

\mytt{double \newfunction{Ab}(const double msq, const double mu2)}:
returns $\overline A(m^2,\mu^2)$.\\
\mytt{double \newfunction{Bb}(const double msq, const double mu2)}:
returns $\overline B(m^2,\mu^2)$.\\
\mytt{double \newfunction{Cb}(const double msq, const double mu2)}:
returns $\overline C(m^2,\mu^2)$.\\
\mytt{double \newfunction{Abeps}(const double msq, const double mu2)}:
returns $\overline A^\epsilon(m^2,\mu^2)$.\\
\mytt{double \newfunction{Bbeps}(const double msq, const double mu2)}:
returns $\overline B^\epsilon(m^2,\mu^2)$.\\
\mytt{double \newfunction{Cbeps}(const double msq, const double mu2)}:
returns $\overline C^\epsilon(m^2,\mu^2)$.\\
\mytt{double \newfunction{Ab}(const int n,const double msq, const double mu2)}:
returns $\overline A(m^2,\mu^2)$, $\overline B(m^2,\mu^2)$, $\overline C(m^2,\mu^2)$
for $n=1,2,3$.\\

Defined in \newfunction{oneloopintegrals.h}, implemented in \newfunction{oneloopintegrals.cc},
examples of use in \newfunction{testoneloopintegrals.cc}.

\subsection{Bubbles or two-propagator integrals}
\label{bubbles}
\index{Bubbles}

\subsubsection{Definitions}

We first define the abbreviation
\begin{equation}
\label{defB}
\langle X \rangle = \frac{\mu^{4-d}}{i}\int \frac{d^d q}{(2\pi)^d}
 \frac{X}{\left(q^2-m_1^2\right) \left((q-p)^2-m_2^2\right) }\,.
\end{equation}
The bubble integrals themselves are defined by, see e.g. \cite{Bijnens:2002hp},
\begin{align}
B(m_1^2,m_2^2,p^2,\mu^2) =\, &
\langle 1 \rangle\,, &&
\nonumber\\
B_\mu(m_1^2,m_2^2,p,\mu^2) =\, &
\langle q_\mu \rangle
& = &\,p_\mu B_1(m_1^2,m_2^2,p^2,\mu^2)\,,
\nonumber\\
B_{\mu\nu}(m_1^2,m_2^2,p,\mu^2) =\, &
\langle q_\mu q_\nu\rangle
&=&\, p_\mu p_\nu B_{21}(m_1^2,m_2^2,p^2,\mu^2)
+g_{\mu\nu} B_{22}(m_1^2,m_2^2,p^2,\mu^2)\,,
\nonumber\\
B_{\mu\nu\rho}(m_1^2,m_2^2,p,\mu^2) =\, &
\langle q_\mu q_\nu q_\rho\rangle
&=&\, p_\mu p_\nu p_\rho B_{31}(m_1^2,m_2^2,p^2,\mu^2)
\nonumber\\
& & &+\left(g_{\mu\nu}p_\rho+g_{\mu\rho}p_\nu+g_{\rho\nu}p_\mu\right) B_{32}(m_1^2,m_2^2,p^2,\mu^2)\,.
\end{align}
The methods of \cite{Passarino:1978jh}
can be used to deduce the relations
\begin{align}
B_1(m_1^2,m_2^2,p^2,\mu^2) =\, &-\frac{1}{2p^2}\left(
  A(m_1^2,\mu^2)-A(m_2^2,\mu^2)+(m_2^2-m_1^2-p^2) B(m_1^2,m_2^2,p^2,\mu^2)
\right)\,,
\nonumber\\
B_{22}(m_1^2,m_2^2,p^2,\mu^2) =\, & \frac{1}{2(d-1)}\Big(
 A(m_2^2,\mu^2)+2 m_1^2 B(m_1^2,m_2^2,p^2,\mu^2)
\nonumber\\&
+(m_2^2-m_1^2-p^2) B_1(m_1^2,m_2^2,p^2,\mu^2)
\Big)\,,
\nonumber\\
B_{21}(m_1^2,m_2^2,p^2,\mu^2) =\, &\frac{1}{p^2}\left(
  A(m_2^2,\mu^2)+m_1^2 B(m_1^2,m_2^2,p^2,\mu^2)-d B_{22}(m_1^2,m_2^2,p^2,\mu^2)
\right)\,.
\end{align}
This allows to rewrite all in terms of $B(m_1^2,m_2^2,p^2,\mu^2)$.
These relations are used for the analytical evaluations given below.

The final evalaution is done by using a Feynman parameter $x$ to
combine the propagators and use the results for the tadpoles.
The $x$ integral needed can be done analytically or numerically.

The functions are then all expanded in terms of $\epsilon$. The arguments of
the various Bubble integrals are not written out.
\begin{align}
 B = \,&\frac{\lambda_0}{16\pi^2}+\overline B +\epsilon B^\epsilon
 +\mathcal{O}(\epsilon^2)\,,
\nonumber\\
 B_1 = \,&\frac{\lambda_0}{16\pi^2}\frac{1}{2}+\overline B_1 
+\epsilon B_1^\epsilon +\mathcal{O}(\epsilon^2)\,,
\nonumber\\
 B_{21} = \,&\frac{\lambda_0}{16\pi^2}\frac{1}{3}+\overline B_{21} 
+\epsilon B_{21}^\epsilon +\mathcal{O}(\epsilon^2)\,,
\nonumber\\
 B_{22} = \,&\frac{\lambda_0}{16\pi^2}
\left(\frac{m_1^2}{4}+\frac{m_2^2}{4}-\frac{p^2}{12}\right)+\overline B_{22} 
+\epsilon B_{22}^\epsilon +\mathcal{O}(\epsilon^2)\,,
\nonumber\\
 B_{31} = \,&\frac{\lambda_0}{16\pi^2}\frac{1}{4}+\overline B_{31} 
+\epsilon B_{31}^\epsilon +\mathcal{O}(\epsilon^2)\,,
\nonumber\\
 B_{32} = \,&\frac{\lambda_0}{16\pi^2}
\left(\frac{m_1^2}{12}+\frac{m_2^2}{6}-\frac{p^2}{24}\right)+\overline B_{32} 
+\epsilon B_{32}^\epsilon +\mathcal{O}(\epsilon^2)\,,
\end{align}

\subsubsection{Analytical implementation}

The functions in this section are all implemented fully analytically.\\

\mytt{const double: msq, m1sq, m2sq, psq, mu2}
these are $m^2,m_1^2,m_2^2,p^2,\mu^2$.

\mytt{dcomplex \newfunction{Bb}(m1sq, m2sq, psq, mu2)}:
returns $\overline B(m_1^2,m_2^2,p^2,\mu^2)$\\
\mytt{dcomplex \newfunction{Bb}(msq, psq, mu2)}:
returns $\overline B(m^2,m^2,p^2,\mu^2)$ using the simpler equal mass formula.\\
\mytt{dcomplex \newfunction{B1b}(m1sq, m2sq, psq, mu2)}:
returns $\overline B_1(m_1^2,m_2^2,p^2,\mu^2)$\\
\mytt{dcomplex \newfunction{B21b}(m1sq, m2sq, psq, mu2)}:
returns $\overline B_{21}(m_1^2,m_2^2,p^2,\mu^2)$\\
\mytt{dcomplex \newfunction{B22b}(m1sq, m2sq, psq, mu2)}:
returns $\overline B_{22}(m_1^2,m_2^2,p^2,\mu^2)$\\
\mytt{dcomplex \newfunction{B22b}(msq, psq, mu2)}:
returns $\overline B_{22}(m^2,m^2,p^2,\mu^2)$ using the simpler equal mass formula.\\

Defined in \newfunction{oneloopintegrals.h}, implemented in \newfunction{oneloopintegrals.cc},
examples of use in \newfunction{testoneloopintegrals.cc}.

\subsubsection{Numerical implementation}

The functions in this section are all implemented using a numerical
complex integration over $x$. The integration routine used
can be specified using the macro \newfunction{WINTEGRAL} which defaults
to \mytt{jbwgauss}. Any of the complex integration routines
of \mytt{jbnumlib} can be used instead.


\mytt{const double: m1sq, m2sq, psq, mu2}
these are $m_1^2,m_2^2,p^2,\mu^2$.

\mytt{dcomplex \newfunction{Bbnum}(m1sq, m2sq, psq, mu2)}:
returns $\overline B(m_1^2,m_2^2,p^2,\mu^2)$\\
\mytt{dcomplex \newfunction{B1bnum}(m1sq, m2sq, psq, mu2)}:
returns $\overline B(m_1^2,m_2^2,p^2,\mu^2)$\\
\mytt{dcomplex \newfunction{B21bnum}(m1sq, m2sq, psq, mu2)}:
returns $\overline B_{21}(m_1^2,m_2^2,p^2,\mu^2)$\\
\mytt{dcomplex \newfunction{B22bnum}(m1sq, m2sq, psq, mu2)}:
returns $\overline B_{22}(m_1^2,m_2^2,p^2,\mu^2)$\\
\mytt{dcomplex \newfunction{B31bnum}(m1sq, m2sq, psq, mu2)}:
returns $\overline B_{31}(m_1^2,m_2^2,p^2,\mu^2)$\\
\mytt{dcomplex \newfunction{B32bnum}(m1sq, m2sq, psq, mu2)}:
returns $\overline B_{32}(m_1^2,m_2^2,p^2,\mu^2)$\\

The precision of the numerical integration can be set and obtained:\\


\mytt{void \newfunction{setprecisiononeloopintegrals}(const double eps)}
sets the precison to \mytt{eps}.\\
\mytt{double \newfunction{getprecisiononeloopintegrals}(void)} returns the present
precision. The default is \mytt{1e-10}.\\

Defined in \newfunction{oneloopintegrals.h}, implemented in \newfunction{oneloopintegrals.cc},
examples of use in \newfunction{testoneloopintegrals.cc}:

\subsection{Sunset integrals}
\label{sunsets}
\index{Sunsets}

\subsubsection{Definition}

We first define the abbreviation
\begin{equation}
\label{defS}
\langle\langle X \rangle\rangle = 
\left(\frac{\mu^{4-d}}{i}\right)\int \frac{d^d r}{(2\pi)^d} \frac{d^d s}{(2\pi)^d}
 \frac{X}{\left(r^2-m_1^2\right)\left(s^2-m_2^2\right) \left((r+s-p)^2-m_3^2\right) }\,.
\end{equation}
The sunset integrals themselves are defined by
\begin{align}
H(m_1^2,m_2^2,m_3^2,p^2,\mu^2) =\, &
\langle\langle 1 \rangle\rangle\,, &&
\nonumber\\
H_\mu(m_1^2,m_2^2,m_3^2,p,\mu^2) =\, &
\langle\langle r_\mu \rangle\rangle
& = &\,p_\mu H_1(m_1^2,m_2^2,m_3^2,p^2,\mu^2)\,,
\nonumber\\
H_{\mu\nu}(m_1^2,m_2^2,m_3^2,p,\mu^2) =\, &
\langle\langle r_\mu r_\nu\rangle\rangle
&=&\, p_\mu p_\nu H_{21}(m_1^2,m_2^2,m_3^2,p^2,\mu^2)
\nonumber\\ & & &
+g_{\mu\nu} H_{22}(m_1^2,m_2^2,m_3^2,p^2,\mu^2)\,,
\nonumber\\
H_{\mu\nu\rho}(m_1^2,m_2^2,m_3^2,p,\mu^2) =\, &
\langle\langle r_\mu r_\nu r_\rho\rangle\rangle
&=&\, p_\mu p_\nu p_\rho H_{31}(m_1^2,m_2^2,m_3^2,p^2,\mu^2)
\nonumber\\ & & &
+\left(g_{\mu\nu}p_\rho+g_{\mu\rho}p_\nu+g_{\rho\nu}p_\mu\right)
 H_{32}(m_1^2,m_2^2,m_3^2,p^2,\mu^2)\,.
\end{align}
The needed integrals with $s_\mu$ replacing some of the $r_\mu$ in the
definitions can be related to those without $s_\mu$ as descibed in
\cite{Amoros:1999dp}.
The evaluation of these sunsetintegrals has been done in
\cite{Amoros:1999dp}. Further references can be found there.

We extract the parts the divergent parts and the parts containing
$C$ via
\begin{align}
\label{defHF}
H(m_1^2,m_2^2,m_3^2,p^2,\mu^2) =\,&
\frac{1}{\left(16\pi^2\right)^2}
\Big[(\lambda_2/2)\left(m_1^2+m_2^2+m_3^2\right)
+(\lambda_1/2)\big(m_1^2(1-\log(m_1^2/\mu^2))
\nonumber\\&\quad
+m_2^2(1-\log(m_2^2/\mu^2))+m_3^2(1-\log(m_3^2/\mu^2))-(p^2/2)\big)\Big]
\nonumber\\&\quad
+H^F(m_1^2,m_2^2,m_3^2,p^2,\mu^2)+\mathcal{O}(\epsilon)\,,
\\
H_1(m_1^2,m_2^2,m_3^2,p^2,\mu^2) =\,&
\frac{1}{\left(16\pi^2\right)^2}
\Big[(\lambda_2/4)\left(m_2^2+m_3^2\right)
+(\lambda_1/8)\big(2m_1^2
\nonumber\\&\quad
+m_2^2(1-4\log(m_2^2/\mu^2))+m_3^2(1-4\log(m_3^2/\mu^2))-(2p^2/3)\big)\Big]
\nonumber\\&\quad
+H_1^F(m_1^2,m_2^2,m_3^2,p^2,\mu^2)+\mathcal{O}(\epsilon)\,,
\\
H_{21}(m_1^2,m_2^2,m_3^2,p^2,\mu^2) =\,&
\frac{1}{\left(16\pi^2\right)^2}
\Big[(\lambda_2/6)\left(m_2^2+m_3^2\right)
+(\lambda_1/36)\big(3m_1^2
\nonumber\\&\quad
+m_2^2(2-12\log(m_2^2/\mu^2))+m_3^2(2-12\log(m_3^2/\mu^2))
\nonumber\\&\quad
-(3p^2/2)\big)\Big]
+H_{21}^F(m_1^2,m_2^2,m_3^2,p^2,\mu^2)+\mathcal{O}(\epsilon)\,,
\\
H_{31}(m_1^2,m_2^2,m_3^2,p^2,\mu^2) =\,&
\frac{1}{\left(16\pi^2\right)^2}
\Big[(\lambda_2/8)\left(m_2^2+m_3^2\right)
+(\lambda_1/96)\big(4m_1^2
\nonumber\\&\quad
+m_2^2(3-24\log(m_2^2/\mu^2))
+m_3^2(3-24\log(m_3^2/\mu^2))
\nonumber\\&\quad
-(12p^2/5)\big)\Big]
+H_{31}^F(m_1^2,m_2^2,m_3^2,p^2,\mu^2)+\mathcal{O}(\epsilon)\,,
\end{align}

The routines for the sunset integrals calculate the value at $p^2=0$
and the derivative there analytically. The remainder is then calculated
with a rather smoot integral valid below threshold for the
\mytt{double hh} functions and with a dispersive method for the
\mytt{dcomplex zhh} functions. The latter is valid above and below
threshold. The functions returning the derivative w.r.t. $p^2$
calculate the value at $p^2=0$ analytically and the remainder via
a numerical integration as above.


\subsubsection{Functions}

The integration routines needed can be set using the macro \newfunction{DINTEGRAL}
for the real integration, default is \mytt{jbdgauss}, and
\newfunction{SINTEGRAL} for the real integration with a singularity, default
is \mytt{jbdcauch}. Any of the similar routines in \mytt{jbnumlib} can
be used instead.\\
 
\mytt{const double: m1sq,m2sq,m3sq,psq,mu2}: these are
$m_1^2,m_2^2,m_3^2,p^2,\mu^2$.\\

Valid below threshold:\\
\mytt{double \newfunction{hh}(m1sq, m2sq, m3sq, psq, mu2)} returns
 $H^F(m_1^2,m_2^2,m_3^2,p^2,\mu^2)$\\
\mytt{double \newfunction{hh1}(m1sq, m2sq, m3sq, psq, mu2)} returns
 $H^F_1(m_1^2,m_2^2,m_3^2,p^2,\mu^2)$\\
\mytt{double \newfunction{hh21}(m1sq, m2sq, m3sq, psq, mu2)} returns
 $H^F_{21}(m_1^2,m_2^2,m_3^2,p^2,\mu^2)$\\
\mytt{double \newfunction{hh31}(m1sq, m2sq, m3sq, psq, mu2)} returns
 $H^F_{31}(m_1^2,m_2^2,m_3^2,p^2,\mu^2)$\\
\mytt{double \newfunction{hhd}(m1sq, m2sq, m3sq, psq, mu2)} returns
 $(\partial/\partial p^2)H^F(m_1^2,m_2^2,m_3^2,p^2,\mu^2)$\\
\mytt{double \newfunction{hh1d}(m1sq, m2sq, m3sq, psq, mu2)} returns
 $(\partial/\partial p^2)H^F_1(m_1^2,m_2^2,m_3^2,p^2,\mu^2)$\\
\mytt{double \newfunction{hh21d}(m1sq, m2sq, m3sq, psq, mu2)} returns
 $(\partial/\partial p^2)H^F_{21}(m_1^2,m_2^2,m_3^2,p^2,\mu^2)$\\

Valid above and below threshold:\\
\mytt{dcomplex \newfunction{zhh}(m1sq, m2sq, m3sq, psq, mu2)} returns
 $H^F(m_1^2,m_2^2,m_3^2,p^2,\mu^2)$\\
\mytt{dcomplex \newfunction{zhh1}(m1sq, m2sq, m3sq, psq, mu2)} returns
 $H^F_1(m_1^2,m_2^2,m_3^2,p^2,\mu^2)$\\
\mytt{dcomplex \newfunction{zhh21}(m1sq, m2sq, m3sq, psq, mu2)} returns
 $H^F_{21}(m_1^2,m_2^2,m_3^2,p^2,\mu^2)$\\
\mytt{dcomplex \newfunction{zhh31}(m1sq, m2sq, m3sq, psq, mu2)} returns
 $H^F_{31}(m_1^2,m_2^2,m_3^2,p^2,\mu^2)$\\
\mytt{dcomplex \newfunction{zhhd}(m1sq, m2sq, m3sq, psq, mu2)} returns
 $(\partial/\partial p^2)H^F(m_1^2,m_2^2,m_3^2,p^2,\mu^2)$\\
\mytt{dcomplex \newfunction{zhh1d}(m1sq, m2sq, m3sq, psq, mu2)} returns
 $(\partial/\partial p^2)H^F_1(m_1^2,m_2^2,m_3^2,p^2,\mu^2)$\\
\mytt{dcomplex \newfunction{zhh21d}(m1sq, m2sq, m3sq, psq, mu2)} returns
 $(\partial/\partial p^2)H^F_{21}(m_1^2,m_2^2,m_3^2,p^2,\mu^2)$\\

\mytt{void \newfunction{setprecisionsunsetintegrals}(const double eps)}
sets the precison to \mytt{eps}.\\
\mytt{double \newfunction{getprecisionsunsetintegrals}(void)} returns the present
precision. The default is \mytt{1e-10}.\\

Defined in \newfunction{sunsetintegrals.h}, implemented in \newfunction{sunsetintegrals.cc},
examples of use in \newfunction{testsunsetintegrals.cc}:

\subsection{Sunsetintegrals with different powers of propagators}
\index{Quenched sunsets}
\subsubsection{Definition}

We first define the abbreviation
\begin{equation}
\label{defQS}
\langle\langle X \rangle\rangle_n = 
\left(\frac{\mu^{4-d}}{i}\right)\int \frac{d^d r}{(2\pi)^d} \frac{d^d s}{(2\pi)^d}
 \frac{X}{\left(r^2-m_1^2\right)^i\left(s^2-m_2^2\right)^j \left((r+s-p)^2-m_3^2\right)^k }\,.
\end{equation}
The translation of $n$ to values for $i,j,k$ is given in Tab.~\ref{tabn}.
\begin{table}
\begin{center}
\begin{tabular}{c|cccccccc}
n & 1 & 2 & 3 & 4 & 5 & 6 & 7 & 8\\
\hline
i & 1 & 2 & 1 & 1 & 2 & 2 & 1 & 2\\
j & 1 & 1 & 2 & 1 & 2 & 1 & 2 & 2\\
k & 1 & 1 & 1 & 2 & 1 & 2 & 2 & 2
\end{tabular}  
\end{center}
\caption{\label{tabn}
The relation between the value of $n$ and the powers $i,j,k$
of the three propagators.}
\end{table}
The sunset integrals themselves are defined by
\begin{align}
H(n,m_1^2,m_2^2,m_3^2,p^2,\mu^2) =\, &
\langle\langle 1 \rangle\rangle_n\,, &&
\nonumber\\
H_\mu(n,m_1^2,m_2^2,m_3^2,p,\mu^2) =\, &
\langle\langle r_\mu \rangle\rangle_n
& = &\,p_\mu H_1(n,m_1^2,m_2^2,m_3^2,p^2,\mu^2)\,,
\nonumber\\
H_{\mu\nu}(n,m_1^2,m_2^2,m_3^2,p,\mu^2) =\, &
\langle\langle r_\mu r_\nu\rangle\rangle_n
&=&\, p_\mu p_\nu H_{21}(n,m_1^2,m_2^2,m_3^2,p^2,\mu^2)
\nonumber\\ & & &
+g_{\mu\nu} H_{22}(n,m_1^2,m_2^2,m_3^2,p^2,\mu^2)\,.
\end{align}
The needed integrals with $s_\mu$ replacing some of the $r_\mu$ in the
definitions can be related to those without $s_\mu$ as descibed in
\cite{Bijnens:2006jv,Bijnens:2005pa,Bijnens:2005ae,Bijnens:2004hk}
The evaluation of these sunsetintegrals is by the generalziation of
the methods of
\cite{Amoros:1999dp}. Further references can be found there.

The divergent parts and the parts containing
$C$ via taking derivatives w.r.t. masses of (\ref{defHF}).
We thus define the functions $H^F_i(n,m_1^2,m_2^2,m_3^2,p^2,\mu^2)$
for all cases above, $i=0(\text{blank}),1,21$.

The routines for the sunset integrals calculate the value at $p^2=0$
and the derivative there analytically. The remainder is then calculated
with a rather smoot integral valid below threshold for the
\mytt{double hh} functions.
% and with a dispersive method for the
%\mytt{dcomplex zhh} functions. The latter is valid above and below
%threshold. 
The functions returning the derivative w.r.t. $p^2$
calculate the value at $p^2=0$ analytically and the remainder via
a numerical integration as above.

An added addition here is that case where the K\"ahl\'en function
$$\lambda(m_1^2,m_2^2,m_3^2)=\sqrt{(m_1^2-m_2^2-m_3^2)^2-4m_2^2m_3^2}$$
vanishes, is treated correctly.

\subsubsection{Functions}

The integration routines needed can be set using the macro \newfunction{DINTEGRAL}
for the real integration, default is \mytt{jbdgauss}.
% and
%\newfunction{SINTEGRAL} for the real integration with a singularity, default
%is \mytt{jbdcauch}. 
Any of the similar routines in \mytt{jbnumlib} can
be used instead.\\
 
\mytt{const int n}: the integer $n$ labelling the powers of the propagators as
defined in Tab.~\ref{tabn}.\\
\mytt{const double: m1sq,m2sq,m3sq,psq,mu2}: these are
$m_1^2,m_2^2,m_3^2,p^2,\mu^2$.\\

Valid below threshold:\\
\mytt{double \newfunction{hh}(n,m1sq, m2sq, m3sq, psq, mu2)} returns
 $H^F(n,m_1^2,m_2^2,m_3^2,p^2,\mu^2)$\\
\mytt{double \newfunction{hh1}(n,m1sq, m2sq, m3sq, psq, mu2)} returns
 $H^F_1(n,m_1^2,m_2^2,m_3^2,p^2,\mu^2)$\\
\mytt{double \newfunction{hh21}(n,m1sq, m2sq, m3sq, psq, mu2)} returns
 $H^F_{21}(n,m_1^2,m_2^2,m_3^2,p^2,\mu^2)$\\
\mytt{double \newfunction{hhd}(n,m1sq, m2sq, m3sq, psq, mu2)} returns
 $(\partial/\partial p^2)H^F(n,m_1^2,m_2^2,m_3^2,p^2,\mu^2)$\\
\mytt{double \newfunction{hh1d}(n,m1sq, m2sq, m3sq, psq, mu2)} returns
 $(\partial/\partial p^2)H^F_1(n,m_1^2,m_2^2,m_3^2,p^2,\mu^2)$\\
\mytt{double \newfunction{hh21d}(n,m1sq, m2sq, m3sq, psq, mu2)} returns
 $(\partial/\partial p^2)H^F_{21}(n,m_1^2,m_2^2,m_3^2,p^2,\mu^2)$\\

\mytt{void \newfunction{setprecisionquenchedsunsetintegrals}(const double eps)}
sets the precison to \mytt{eps}.\\
\mytt{double \newfunction{getprecisionquenchedsunsetintegrals}(void)} returns the present
precision. The default is \mytt{1e-10}.\\

Defined in \newfunction{quenchedsunsetintegrals.h}, implemented in \newfunction{quenchedsunsetintegrals.cc},
examples of use in \newfunction{testquenchedsunsetintegrals.cc}:

\subsection{Finite volume tadpole integrals}

\subsubsection{Definitions}

The methods used for these are derived in detail in \cite{Bijnens:2013doa},
references to earlier literature can be found there.
The integrals used here are given in the Minkowski conventions as
defined in \cite{Bijnens:2014dea}.
All of the integrals are available with two different methods, one using
a summation over Bessel function and the other an integral over a
Jacobi theta function. The versions included at present are using
periodic boundary conditions, all three spatial sizes of the same length $L$
and the time direction of infinite extent.

The tadpole integrals $A$ and $A_{\mu\nu}$ are defined as
\begin{equation}
\left\{\tilde A^V(m^2,L,\mu^2),\tilde A^V_{\mu\nu}(m^2,L,\mu^2)\right\}
 = \frac{\mu^{4-d}}{i}\int_V\frac{d^d r}{(2\pi)^d}
\frac{\left\{1,r_\mu r_\nu\right\}}{(r^2-m^2)}\,.
\end{equation}
The $B$ tadpole integrals are the same but with a doubled propagator,
$C$ tadpoles are with a tripled propagator and $D$ tadpoles with a quadrupled
propagator.
The subscript $V$ on the integral indicates that the integral is a discrete
sum over the three spatial components and an integral over the remainder. 
The size of the spatial directions is $L$.

At finite volume, there are more Lorentz-structures possible. 
The tensor $t_{\mu\nu}$, the spatial part of the Minkowski metric
$g_{\mu\nu}$, is needed for these.
The functions for $\tilde A^V_{\mu\nu}$ are
\begin{equation}
\tilde A^V_{\mu\nu}(m^2,L,\mu^2) = g_{\mu\nu}\tilde A^V_{22}(m^2,L,\mu^2)+t_{\mu\nu}\tilde A^V_{23}(m^2,L,\mu^2)\,.
\end{equation}
Similar definitions are relevant for the $B,C,D$ tadpoles.
In infinite volume $A_{22}$ is related to $A$ and $A_{23}$ vanishes.
The relations in finite volume is given by
\begin{equation}
d\tilde A^V_{22}(m^2,L,\mu^2)+3\tilde A^V_{23}(m^2,L,\mu^2) =
 m^2\tilde A^V(m^2,L,\mu^2)\,.
\end{equation}
The relations for the other cases are:
\begin{align}
d\tilde B^V_{22}(m^2,L,\mu^2)+3\tilde B^V_{23}(m^2,L,\mu^2)& =
 m^2\tilde B^V(m^2,L,\mu^2)+\tilde A^V(m^2,L,\mu^2)\,,\nonumber\\
d\tilde C^V_{22}(m^2,L,\mu^2)+3\tilde C^V_{23}(m^2,L,\mu^2)& =
 m^2\tilde C^V(m^2,L,\mu^2)+\tilde B^V(m^2,L,\mu^2)\,,\nonumber\\
d\tilde D^V_{22}(m^2,L,\mu^2)+3\tilde D^V_{23}(m^2,L,\mu^2)& =
 m^2\tilde D^V(m^2,L,\mu^2)+\tilde C^V(m^2,L,\mu^2)\,.
\end{align}

The full integrals are now split in the infinite volume part
which was defined earlier in Sect.~\ref{tadpoles} and the finite volume
remainder as
\begin{align}
\tilde A^V(m^2,L,\mu^2)\!
=&\frac{\lambda_0 m^2}{16\pi^2}
  +\overline A(m^2,\mu^2)+\overline A^V(m^2,L)
  +\epsilon\left(A^\epsilon(m^2,\mu^2)+A^{V\epsilon}(m^2,L,\mu^2)
     \right)
  +\mathcal{O}(\epsilon^2)\,,
\nonumber\\
\tilde B^V(m^2,L,\mu^2)\!
=&\frac{\lambda_0}{16\pi^2}
  +\overline B(m^2,\mu^2)+\overline B^V(m^2,L)
  +\epsilon\left(B^\epsilon(m^2,\mu^2)+B^{V\epsilon}(m^2,L,\mu^2)
     \right)
  +\mathcal{O}(\epsilon^2)\,,
\nonumber\\
\tilde C^V(m^2,L,\mu^2)\!
=&\overline C(m^2,\mu^2)+\overline C^V(m^2,L)
  +\epsilon\left(C^\epsilon(m^2,\mu^2)+C^{V\epsilon}(m^2,L,\mu^2)
     \right)
  +\mathcal{O}(\epsilon^2)\,,
\nonumber\\
\tilde D^V(m^2,L,\mu^2)\!
=&\overline D(m^2,\mu^2)+\overline D^V(m^2,L)
  +\epsilon\left(D^\epsilon(m^2,\mu^2)+D^{V\epsilon}(m^2,L,\mu^2)
     \right)
  +\mathcal{O}(\epsilon^2)\,,
\nonumber\\
\tilde A^V_{22}(m^2,L,\mu^2)\!
=&\frac{\lambda_0 m^4}{4(16\pi^2)}
  +\overline A_{22}(m^2,\mu^2)+\overline A^V_{22}(m^2,L)
  +\epsilon\left(A^\epsilon_{22}(m^2,\mu^2)+A^{V\epsilon}_{22}(m^2,L,\mu^2)
     \right)
\nonumber\\
\tilde B^V_{22}(m^2,L,\mu^2)\!
=&\frac{\lambda_0 m^2}{2(16\pi^2)}
  +\overline B_{22}(m^2,\mu^2)+\overline B^V_{22}(m^2,L)
  +\epsilon\left(B^\epsilon_{22}(m^2,\mu^2)+B^{V\epsilon}_{22}(m^2,L,\mu^2)
     \right)
\nonumber\\
\tilde C^V_{22}(m^2,L,\mu^2)\!
=&\frac{\lambda_0}{4(16\pi^2)}
  +\overline C_{22}(m^2,\mu^2)+\overline C^V_{22}(m^2,L)
  +\epsilon\left(C^\epsilon_{22}(m^2,\mu^2)+C^{V\epsilon}_{22}(m^2,L,\mu^2)
     \right)
\nonumber\\
\tilde D^V_{22}(m^2,L,\mu^2)\!
=&\overline D_{22}(m^2,\mu^2)+\overline D^V_{22}(m^2,L)
  +\epsilon\left(D^\epsilon_{22}(m^2,\mu^2)+D^{V\epsilon}_{22}(m^2,L,\mu^2)
     \right)
  +\mathcal{O}(\epsilon^2)\,,
\nonumber\\
\tilde A^V_{23}(m^2,L,\mu^2)
=&\overline A^V_{23}(m^2,\mu^2)
  +\epsilon A^{V\epsilon}_{22}(m^2,L)
  +\mathcal{O}(\epsilon^2)\,.
\nonumber\\
\tilde B^V_{23}(m^2,L,\mu^2)
=&\overline B^V_{23}(m^2,\mu^2)
  +\epsilon B^{V\epsilon}_{22}(m^2,L)
  +\mathcal{O}(\epsilon^2)\,.
\nonumber\\
\tilde C^V_{23}(m^2,L,\mu^2)
=&\overline C^V_{23}(m^2,\mu^2)
  +\epsilon C^{V\epsilon}_{22}(m^2,L)
  +\mathcal{O}(\epsilon^2)\,.
\nonumber\\
\tilde D^V_{23}(m^2,L,\mu^2)
=&\overline D^V_{23}(m^2,\mu^2)
  +\epsilon D^{V\epsilon}_{22}(m^2,L)
  +\mathcal{O}(\epsilon^2)\,.
\end{align}

\subsubsection{Functions}


The integration routines needed can be set using the macro \newfunction{DINTEGRAL}
for the real integration, default is \mytt{jbdgauss}.
% and
%\newfunction{SINTEGRAL} for the real integration with a singularity, default
%is \mytt{jbdcauch}. 
Any of the similar routines in \mytt{jbnumlib} can
be used instead.\\


\mytt{const double: msq,L} : \mytt{msq} is $m^2$ and \mytt{L} is
the size $L$ of the spatial dimension.\\

Evaluated with theta functions:\\
\mytt{double \newfunction{AbVt}(msq,L)}: returns $\overline A^V(m^2,L)$.\\
\mytt{double \newfunction{A22bVt}(msq,L)}:
 returns $\overline A^V_{22}(m^2,L)$.\\
\mytt{double \newfunction{A23bVt}(msq,L)}:
 returns $\overline A^V_{23}(m^2,L)$.\\
\mytt{double \newfunction{BbVt}(msq,L)}: returns $\overline B^V(m^2,L)$.\\
\mytt{double \newfunction{B22bVt}(msq,L)}:
 returns $\overline B^V_{22}(m^2,L)$.\\
\mytt{double \newfunction{B23bVt}(msq,L)}:
 returns $\overline B^V_{23}(m^2,L)$.\\
\mytt{double \newfunction{CbVt}(msq,L)}: returns $\overline C^V(m^2,L)$.\\
\mytt{double \newfunction{C22bVt}(msq,L)}:
 returns $\overline C^V_{22}(m^2,L)$.\\
\mytt{double \newfunction{C23bVt}(msq,L)}:
 returns $\overline C^V_{23}(m^2,L)$.\\
\mytt{double \newfunction{DbVt}(msq,L)}: returns $\overline D^V(m^2,L)$.\\
\mytt{double \newfunction{D22bVt}(msq,L)}:
 returns $\overline D^V_{22}(m^2,L)$.\\
\mytt{double \newfunction{D23bVt}(msq,L)}:
 returns $\overline D^V_{23}(m^2,L)$.\\

Evaluated with Bessel functions:\\
\mytt{double \newfunction{AbVb}(msq,L)}: returns $\overline A^V(m^2,L)$.\\
\mytt{double \newfunction{A22bVb}(msq,L)}:
 returns $\overline A^V_{22}(m^2,L)$.\\
\mytt{double \newfunction{A23bVb}(msq,L)}:
 returns $\overline A^V_{23}(m^2,L)$.\\
\mytt{double \newfunction{BbVb}(msq,L)}: returns $\overline B^V(m^2,L)$.\\
\mytt{double \newfunction{B22bVb}(msq,L)}:
 returns $\overline B^V_{22}(m^2,L)$.\\
\mytt{double \newfunction{B23bVb}(msq,L)}:
 returns $\overline B^V_{23}(m^2,L)$.\\
\mytt{double \newfunction{CbVb}(msq,L)}: returns $\overline C^V(m^2,L)$.\\
\mytt{double \newfunction{C22bVb}(msq,L)}:
 returns $\overline C^V_{22}(m^2,L)$.\\
\mytt{double \newfunction{C23bVb}(msq,L)}:
 returns $\overline C^V_{23}(m^2,L)$.\\
\mytt{double \newfunction{DbVb}(msq,L)}: returns $\overline D^V(m^2,L)$.\\
\mytt{double \newfunction{D22bVb}(msq,L)}:
 returns $\overline D^V_{22}(m^2,L)$.\\
\mytt{double \newfunction{D23bVb}(msq,L)}:
 returns $\overline D^V_{23}(m^2,L)$.\\


The last letter
indicates whether they are computed with the theta function or Bessel
function method.
The results were checked by comparing against each other and
by comparing when possible with the independent Bessel function
implementation done in \cite{Bijnens:2006ve}.\\

\mytt{void \newfunction{setprecisionfinitevolumeoneloopt}(const double Abacc=1e-10,}
\\\hspace*{5mm}\mytt{const double Bbacc=1e-9,const bool printout=true)}
sets the precision for the finite volume integrals evaluated with
theta function to \mytt{Abacc} for the tadpole integrals, \mytt{Bbacc}
for the bubble integrals. The last variable printout is a logical variable
which can be se tto true or false, default is false.
Default values are those indicated.\\

\mytt{void \newfunction{setprecisionfinitevolumeoneloopb}(const int maxsum=100,}
\\\hspace*{5mm}\mytt{const double Bbacc=1e-5,const bool printout=true)}
sets the precision for the finite volume integrals evaluated with Bessel
functions. The first argument
indicates how far the sum over Bessel functions is taken.
Maximum at present is 1000.
The second argument gives the
precision of the numerical integration for the
bubble integrals.

Defined in \newfunction{finitevolumeoneloopintegrals.h},
implemented in \newfunction{finitevolumeoneloop\-integrals.cc},
examples of use in \newfunction{testfinitevolumeoneloopintegrals.cc}.

\subsection{Finite volume bubble integrals}

\subsubsection{Definitions}
\label{defFVbubble}

The methods used for these are derived in detail in \cite{Bijnens:2013doa},
references to earlier literature can be found there.
The integrals used here are given in the Minkowski conventions as
defined in \cite{Bijnens:2014dea}.
All of the integrals are available with two different methods, one using
a summation over Bessel function and the other an integral over a
Jacobi theta function. The versions included at present are using
periodic boundary conditions, all three spatial sizes of the same length $L$
and the time direction of infinite extent.

The bubble integrals $B$, $B_\mu$ and $B_{\mu\nu}$ are defined as
\begin{equation}
\left\{\tilde B^V,\tilde B^V_\mu,\tilde B^V_{\mu\nu}\right\}(m_1^2,m_2^2,p,L,\mu^2)
 = \frac{\mu^{4-d}}{i}\int_V\frac{d^d r}{(2\pi)^d}
\frac{\left\{1,r_\mu,r_\mu r_\nu\right\}}{(r^2-m_1^2)^{n_1}((p-r)^2-m_1^2)^{n_2}}\,.
\end{equation}
The subscript $V$ on the integral indicates that the integral is a discrete
sum over the three spatial components and an integral over the remainder. 
The size of the spatial directions is $L$.

At finite volume, there are more Lorentz-structures possible. 
The tensor $t_{\mu\nu}$, the spatial part of the Minkowski metric
$g_{\mu\nu}$, is needed for these. In addition, the functions can in principle
depend on the components of the vector $p$ as well, not only via $p^2$.

The functions themselves are split in the infinite volume part discussed
in Sect.~\ref{bubbles}
and a finite volume part via
\begin{align}
\tilde B^V\!
=&\frac{\lambda_0}{16\pi^2}
  +\overline B(m^2,\mu^2)+\overline B^V(m^2,L)
  +\epsilon\left(B^\epsilon+B^{V\epsilon}     \right)
  +\mathcal{O}(\epsilon^2)\,,
\nonumber\\
\end{align}
where all functions have as argument $(m_1^2,m_2^2,p,L,\mu^2)$.

\subsubsection{Functions}
\label{defFVbubbles}
The integration routines needed can be set using the macro \newfunction{DINTEGRAL}
for the real integration, default is \mytt{jbdgauss}.\\

\mytt{void \newfunction{setprecisionfinitevolumeoneloopt}(const double Abacc=1e-10,}
\\\hspace*{5mm}\mytt{const double Bbacc=1e-9,const bool printout=true)}
sets the precision for the finite volume integrals evaluated with
theta function to \mytt{Abacc} for the tadpole integrals, \mytt{Bbacc}
for the bubble integrals. The last variable printout is a logical variable
which can be set to true or false, default is false.
Default values are those indicated.\\

\mytt{void \newfunction{setprecisionfinitevolumeoneloopb}(const int maxsum=100,}
\\\hspace*{5mm}\mytt{const double Bbacc=1e-5,const bool printout=true)}
sets the precision for the finite volume integrals evaluated with Bessel
functions. The first argument
indicates how far the sum over Bessel functions is taken.
Maximum at present is 1000.
The second argument gives the
precision of the numerical integration for the
bubble integrals.\\

Defined in \newfunction{finitevolumeoneloopintegrals.h},
implemented in \newfunction{finitevolumeoneloop\-integrals.cc},
examples of use in \newfunction{testfinitevolumeoneloopintegrals.cc}.

\paragraph{$p=0$ and periodic boundary conditions}

\mytt{const double: m1sq,m2sq,L,mu2.}
These correspond to $msq,m_1^2,m_2^2,L,\mu^2$. $L$ is the size of the finite
dimensions.

Evaluated with theta functions:\\
\mytt{double \newfunction{BbVt}(m1sq,m2sq,L):}
returns $\overline{B}^V(m_1^2,m_2^2,p=0,L,\mu^2)$\\
\mytt{double \newfunction{BbVt}(msq,L):}
returns $\overline{B}^V(m^2,m^2,p=0,L)$\\

Evaluated with Bessel functions:\\
\mytt{double \newfunction{BbVb}(m1sq,m2sq,L):}
return $\overline{B}^V(m_1^2,m_2^2,p=0,L,\mu^2)$\\
\mytt{double \newfunction{BbVb}(msq,L):}
returns $\overline{B}^V(m^2,m^2,p=0,L,\mu^2)$\\

\subsection{Finite volume sunsetintegrals}

\subsubsection{Definitions}
\label{defFVsunset}

The sunset integrals are defined with
\begin{equation}
\langle\langle X \rangle\rangle_V =
\frac{\mu^{8-2d}}{i^2}\int_V\frac{d^d r}{(2\pi)^d}\frac{d^d 1}{(2\pi)^d}
\frac{\left\{1,r_\mu,r_\mu r_\nu\right\}}
{\left(r^2-m_1^2\right)\left(s^2-m_2^2\right)\left((r+s-p)^2-m_3^2\right)}\,.
\end{equation}
The subscript $V$ indicates that the spatial dimensions are a discrete
sum rather than an integral.
The conventions correspond to those in infinite volume of
\cite{Amoros:1999dp} and of Sect.~\ref{sunsets}.
Integrals with the other momentum $s$ in the numerator are related
using the relations shown in \cite{Amoros:1999dp} which remain
valid at finite volume in the cms frame \cite{Bijnens:2013doa}.

In the cms frame we define the functions\footnote{In the cms frame,
after the reduction to four dimensions,
$t_{\mu\nu}= g_{\mu\nu}-p_\mu p_\nu/p^2$ but the separation
appears naturally in the calculation \cite{Bijnens:2013doa}. In addition, it
avoids singularities in the limit $p\to0$.}
\begin{align}
\label{defHi}
\tilde H^V_\mu =& \langle\langle X \rangle\rangle_V
\\\nonumber
\tilde H^V_\mu =& \langle\langle r_\mu \rangle\rangle_V = p_\mu \tilde H^V_1\,
\\\nonumber
\tilde H^V_{\mu\nu} =&  \langle\langle r_\mu r_\nu \rangle\rangle_V
=p_\mu p_\nu \tilde H^V_{21} + g_{\mu\nu} \tilde H^V_{22} + t_{\mu\nu} \tilde H^V_{27}\,.
\end{align}
The arguments of all functions in the cms frame are
$(m_1^2,m_2^2,m_3^2,p^2,L,\mu^2)$.
These functions satisfy in finite
volume \cite{Bijnens:2013doa},
\begin{align}
\label{Hrelations}
\tilde H^V_1+
\tilde H^V_1(m_2^2,m_3^2,m_1^2,p^2,L,\mu^2)+
\tilde H^V_1(m_3^2,m_1^2,m_2^2,p^2,L,\mu^2)
=& \tilde H^V\,,
\nonumber\\
p^2 \tilde H^V_{21}+d \tilde H^V_{22} + 3 \tilde H^V_{27}-m_1^2 H =&
\tilde A^V(m_2^2)\tilde A^V(m_3^2)\,.
\end{align}
The arguments of the sunset functions in the relations, if not mentioned
explicitly, are  $(m_1^2,m_2^2,m_3^2,p^2,L,\mu^2)$.

We split the functions in an infinite volume part, $H_i$, and a finite
volume correction, $H^V_i$, with
$\tilde H^V_i= H_i+ H^V_i$. The infinite volume part has been
discussed above.
For the finite volume parts we define
\begin{align}
\label{defHVF}
H^V =& \frac{\lambda_0}{16\pi^2}
 \left(\overline A^V(m_1^2)+\overline A^V(m_2^2)+\overline A^V(m_3^2)\right)
 +\frac{1}{16\pi^2}
    \left(A^{V\epsilon}(m_1^2)+A^{V\epsilon}(m_2^2)+A^{V\epsilon}(m_3^2)\right)
\nonumber\\&
 +H^{VF}+\mathcal{O}(\epsilon)\,,
\nonumber\\
H^V_1 =& \frac{\lambda_0}{16\pi^2}\frac{1}{2}
 \left(\overline A^V(m_2^2)+\overline A^V(m_3^2)\right)
 +\frac{1}{16\pi^2}\frac{1}{2}
    \left(A^{V\epsilon}(m_2^2)+A^{V\epsilon}(m_3^2)\right)
 +H^{VF}_1+\mathcal{=}(\epsilon)\,,
\nonumber\\
H^V_{21} =& \frac{\lambda_0}{16\pi^2}\frac{1}{3}
 \left(\overline A^V(m_2^2)+\overline A^V(m_3^2)\right)
 +\frac{1}{16\pi^2}\frac{1}{3}
    \left(A^{V\epsilon}(m_2^2)+A^{V\epsilon}(m_3^2)\right)
 +H^{VF}_{21}+\mathcal{O}(\epsilon)\,,
\nonumber\\
 H^V_{27} =& \frac{\lambda_0}{16\pi^2}
 \left(\overline A^V_{23}(m_1^2)+\frac{1}{3}\overline A_{23}(m_2^2)+
       \frac{1}{3}\overline A^V_{23}(m_3^2)\right)
\nonumber\\&
 +\frac{1}{16\pi^2}
    \left(A^{V\epsilon}_{23}(m_1^2)
+\frac{1}{3}A^{V\epsilon}_{23}(m_2^2)
+\frac{1}{3}A^{V\epsilon}_{23}(m_3^2)\right)
 +H^{VF}_{27}+\mathcal{O}(\epsilon)\,.
\end{align}
The finite parts are defined differently from the
infinite volume case in \cite{Amoros:1999dp}.
The parts with $A^{V\epsilon}$ are removed here as well.

The functions $H^{VF}_i$ can be computed with the methods
of \cite{Bijnens:2013doa}.
They are obtained by adding the parts labeled with $G$ and $H$ in Sect. 4.3 and
the part of Sect. 4.4 in \cite{Bijnens:2013doa}.
The derivatives w.r.t. $p^2$ can be treated using a simple adaptation of that method.

The method for evaluation works only below threshold. The numerical evaluation
is rather slow. Playing with the precision settings for the specific case
you need is very strongly recommended.

\subsubsection{Functions}

\mytt{const double: m1sq,m2sq,m3sq,psq,L,mu2}. These correspond
to $m_1^2,m_2^2,m_3^2,p^2,L,\mu^2$.\\

Evaluation using theta functions:\\
\mytt{double \newfunction{hhVt}(m1sq,m2sq,m3sq,psq,L,mu2)}:
returns $H^{VF}(m_1^2,m_2^2,m_3^2,p^2,L,\mu^2)$.\\
\mytt{double \newfunction{hh1Vt}(m1sq,m2sq,m3sq,psq,L,mu2)}:
returns $H^{VF}_1(m_1^2,m_2^2,m_3^2,p^2,L,\mu^2)$.\\
\mytt{double \newfunction{hh21Vt}(m1sq,m2sq,m3sq,psq,L,mu2)}:
returns $H^{VF}_{21}(m_1^2,m_2^2,m_3^2,p^2,L,\mu^2)$.\\
\mytt{double \newfunction{hh22Vt}(m1sq,m2sq,m3sq,psq,L,mu2)}:
returns $H^{VF}_{22}(m_1^2,m_2^2,m_3^2,p^2,L,\mu^2)$.\\
\mytt{double \newfunction{hh27Vt}(m1sq,m2sq,m3sq,psq,L,mu2)}:
returns $H^{VF}_{27}(m_1^2,m_2^2,m_3^2,p^2,L,\mu^2)$.\\
\mytt{double \newfunction{hhdVt}(m1sq,m2sq,m3sq,psq,L,mu2)}:
returns $(\partial/\partial p^2)H^{VF}(m_1^2,m_2^2,m_3^2,p^2,L,\mu^2)$.\\
\mytt{double \newfunction{hh1dVt}(m1sq,m2sq,m3sq,psq,L,mu2)}:
returns $(\partial/\partial p^2)H^{VF}_1(m_1^2,m_2^2,m_3^2,p^2,L,\mu^2)$.\\
\mytt{double \newfunction{hh21dVt}(m1sq,m2sq,m3sq,psq,L,mu2)}:
returns $(\partial/\partial p^2)H^{VF}_{21}(m_1^2,m_2^2,m_3^2,p^2,L,\mu^2)$.\\
\mytt{double \newfunction{hh22dVt}(m1sq,m2sq,m3sq,psq,L,mu2)}:
returns $(\partial/\partial p^2)H^{VF}_{22}(m_1^2,m_2^2,m_3^2,p^2,L,\mu^2)$.\\
\mytt{double \newfunction{hh27dVt}(m1sq,m2sq,m3sq,psq,L,mu2)}:
returns $(\partial/\partial p^2)H^{VF}_{27}(m_1^2,m_2^2,m_3^2,p^2,L,\mu^2)$.\\

Evaluation using Bessel functions:\\
\mytt{double \newfunction{hhVb}(m1sq,m2sq,m3sq,psq,L,mu2)}:
returns $H^{VF}(m_1^2,m_2^2,m_3^2,p^2,L,\mu^2)$.\\
\mytt{double \newfunction{hh1Vb}(m1sq,m2sq,m3sq,psq,L,mu2)}:
returns $H^{VF}_1(m_1^2,m_2^2,m_3^2,p^2,L,\mu^2)$.\\
\mytt{double \newfunction{hh21Vb}(m1sq,m2sq,m3sq,psq,L,mu2)}:
returns $H^{VF}_{21}(m_1^2,m_2^2,m_3^2,p^2,L,\mu^2)$.\\
\mytt{double \newfunction{hh22Vb}(m1sq,m2sq,m3sq,psq,L,mu2)}:
returns $H^{VF}_{22}(m_1^2,m_2^2,m_3^2,p^2,L,\mu^2)$.\\
\mytt{double \newfunction{hh27Vb}(m1sq,m2sq,m3sq,psq,L,mu2)}:
returns $H^{VF}_{27}(m_1^2,m_2^2,m_3^2,p^2,L,\mu^2)$.\\
\mytt{double \newfunction{hhdVb}(m1sq,m2sq,m3sq,psq,L,mu2)}:
returns $(\partial/\partial p^2)H^{VF}(m_1^2,m_2^2,m_3^2,p^2,L,\mu^2)$.\\
\mytt{double \newfunction{hh1dVb}(m1sq,m2sq,m3sq,psq,L,mu2)}:
returns $(\partial/\partial p^2)H^{VF}_1(m_1^2,m_2^2,m_3^2,p^2,L,\mu^2)$.\\
\mytt{double \newfunction{hh21dVb}(m1sq,m2sq,m3sq,psq,L,mu2)}:
returns $(\partial/\partial p^2)H^{VF}_{21}(m_1^2,m_2^2,m_3^2,p^2,L,\mu^2)$.\\
\mytt{double \newfunction{hh22dVb}(m1sq,m2sq,m3sq,psq,L,mu2)}:
returns $(\partial/\partial p^2)H^{VF}_{22}(m_1^2,m_2^2,m_3^2,p^2,L,\mu^2)$.\\
\mytt{double \newfunction{hh27dVb}(m1sq,m2sq,m3sq,psq,L,mu2)}:
returns $(\partial/\partial p^2)H^{VF}_{27}(m_1^2,m_2^2,m_3^2,p^2,L,\mu^2)$.\\

For all cases discussed both methods, via Bessel or
(generalized) Jacobi theta functions, give the same results.
The derivatives w.r.t. $p^2$ for all the integrals
were compared with taking a numerical derivative.\\

Note that the sunset functions at finite volume call the tadpole integrals
evaluated with the same method. Do not forget to set precision for those
as well.\\

\mytt{void \newfunction{setprecisionfinitevolumesunsett}(const double racc=1e-5,}\\\hspace*{5mm}\mytt{const double rsacc=1e-4,const bool printout=true)}\\
The double values \mytt{sunsetracc} and \mytt{sunsetrsacc}
set the accuracies
of the numerical integration needed when one or two loop-momenta ``feel''
the finite volume. Default values are \mytt{1e-5} and \mytt{1e-4}
respectively.
The bool variable \mytt{printout} defaults to \mytt{true}
and sets whether the setting is
printed. \\

\mytt{void \newfunction{setprecisionfinitevolumesunsetb}(const int maxsum1=100,}
\\\hspace*{5mm}\mytt{
const int maxsum2=40,racc=1e-5,rsacc=1e-4,printout=true)}\\
The integers \mytt{maxsum1} and \mytt{maxsum2} give how far the sum over
Bessel functions
is used for the case with one or two loop momenta
``feeling'' the finite volume. The first is maximum 1000,
the second maximum 40 in the present implementation.
In the latter case a triple sum is needed, hence the much lower upper bound.
The double values \mytt{sunsetracc} and \mytt{sunsetrsacc}
set the accuracies
of the numerical integration which is still needed after the sum for both
cases.\\

For most applications it makes sense to have a higher precision for
the case with one loop momentum quantized, i.e. \mytt{racc} smaller than
\mytt{rsacc}.\\

Defined in \newfunction{finitevolumesunsetintegrals.h},
implemented in \newfunction{finitevolumesunsetintegrals.cc}
and examples of use in \newfunction{testfinitevolumesunsetintegrals.cc}

\subsection{Finite volume sunsetintegrals with different powers of propagators}

\subsubsection{Definitions}

The sunset integrals with different powers of momenta
are defined with
\begin{equation}
\langle\langle X \rangle\rangle_{nV} =
\frac{\mu^{8-2d}}{i^2}\int_V\frac{d^d r}{(2\pi)^d}\frac{d^d 1}{(2\pi)^d}
\frac{\left\{1,r_\mu,r_\mu r_\nu\right\}}
{\left(r^2-m_1^2\right)^i\left(s^2-m_2^2\right)^j\left((r+s-p)^2-m_3^2\right)^k}\,.
\end{equation}
The subscript $V$ indicates that the spatial dimensions are a discrete
sum rather than an integral.
The translation of $n$ to the powers $i,j,k$ is given in Tab.~\ref{tabn}.

In the cms frame we define the functions\footnote{In the cms frame
$t_{\mu\nu}= g_{\mu\nu}-p_\mu p_\nu/p^2$ but the separation
appears naturally in the calculation \cite{Bijnens:2013doa}. In addition, it
avoids singularities in the limit $p\to0$.}
\begin{align}
\label{defHiV}
\tilde H^V_\mu =& \langle\langle X \rangle\rangle_{nV}
\\\nonumber
\tilde H^V_\mu =& \langle\langle r_\mu \rangle\rangle_{nV} = p_\mu \tilde H^V_1\,
\\\nonumber
\tilde H^V_{\mu\nu} =&  \langle\langle r_\mu r_\nu \rangle\rangle_{nV}
=p_\mu p_\nu \tilde H^V_{21} + g_{\mu\nu} \tilde H^V_{22} + t_{\mu\nu} \tilde H^V_{27}\,.
\end{align}
The arguments of all functions in the cms frame are
$(n,m_1^2,m_2^2,m_3^2,p^2,L,\mu^2)$.

We split the functions in an infinite volume part, $H_i$, and a finite
volume correction, $H^V_i$, with
$\tilde H^V_i= H_i+ H^V_i$. The infinite volume part has been
discussed above.
Note that the functions in the section are defined with the derivative
w.r.t. $m_l^2$ for the propagators with mass $m_l^2$.
For the finite volume parts we define the subtraction
as before in (\ref{defHVF}), i.e. we subtract the $A^{V\epsilon},B^{V\epsilon}$
parts.

The functions $H^{VF}_i$ can be computed with the methods
of \cite{Bijnens:2013doa}.
They are obtained by adding the parts labeled with $G$ and $H$ in Sect. 4.3 and
the part of Sect. 4.4 in \cite{Bijnens:2013doa}.
The derivatives w.r.t. $p^2$ can be treated using a simple adaptation of that method.

The method for evaluation works only below threshold. The numerical evaluation
is rather slow. Playing with the precision settings for the specific case
you need is very strongly recommended.

\subsubsection{Functions}

\mytt{const int n} The propagator cases as given in Tab.~\ref{tabn}.\\
\mytt{const double: m1sq,m2sq,m3sq,psq,L,mu2}. These correspond
to $m_1^2,m_2^2,m_3^2,p^2,L,\mu^2$.\\

Evaluation using theta functions:\\
\mytt{double \newfunction{hhVt}(n,m1sq,m2sq,m3sq,psq,L,mu2)}:
returns $H^{VF}(n,m_1^2,m_2^2,m_3^2,p^2,L,\mu^2)$.\\
\mytt{double \newfunction{hh1Vt}(n,m1sq,m2sq,m3sq,psq,L,mu2)}:
returns $H^{VF}_1(n,m_1^2,m_2^2,m_3^2,p^2,L,\mu^2)$.\\
\mytt{double \newfunction{hh21Vt}(n,m1sq,m2sq,m3sq,psq,L,mu2)}:
returns $H^{VF}_{21}(n,m_1^2,m_2^2,m_3^2,p^2,L,\mu^2)$.\\
\mytt{double \newfunction{hh22Vt}(n,m1sq,m2sq,m3sq,psq,L,mu2)}:
returns $H^{VF}_{22}(n,m_1^2,m_2^2,m_3^2,p^2,L,\mu^2)$.\\
\mytt{double \newfunction{hh27Vt}(n,m1sq,m2sq,m3sq,psq,L,mu2)}:
returns $H^{VF}_{27}(n,m_1^2,m_2^2,m_3^2,p^2,L,\mu^2)$.\\
\mytt{double \newfunction{hhdVt}(n,m1sq,m2sq,m3sq,psq,L,mu2)}:
returns $(\partial/\partial p^2)H^{VF}(n,m_1^2,m_2^2,m_3^2,p^2,L,\mu^2)$.\\
\mytt{double \newfunction{hh1dVt}(n,m1sq,m2sq,m3sq,psq,L,mu2)}:
returns $(\partial/\partial p^2)H^{VF}_1(n,m_1^2,m_2^2,m_3^2,p^2,L,\mu^2)$.\\
\mytt{double \newfunction{hh21dVt}(n,m1sq,m2sq,m3sq,psq,L,mu2)}:
returns $(\partial/\partial p^2)H^{VF}_{21}(n,m_1^2,m_2^2,m_3^2,p^2,L,\mu^2)$.\\
\mytt{double \newfunction{hh22dVt}(n,m1sq,m2sq,m3sq,psq,L,mu2)}:
returns $(\partial/\partial p^2)H^{VF}_{22}(n,m_1^2,m_2^2,m_3^2,p^2,L,\mu^2)$.\\
\mytt{double \newfunction{hh27dVt}(n,m1sq,m2sq,m3sq,psq,L,mu2)}:
returns $(\partial/\partial p^2)H^{VF}_{27}(n,m_1^2,m_2^2,m_3^2,p^2,L,\mu^2)$.\\

Evaluation using Bessel functions:\\
\mytt{double \newfunction{hhVb}(n,m1sq,m2sq,m3sq,psq,L,mu2)}:
returns $H^{VF}(n,m_1^2,m_2^2,m_3^2,p^2,L,\mu^2)$.\\
\mytt{double \newfunction{hh1Vb}(n,m1sq,m2sq,m3sq,psq,L,mu2)}:
returns $H^{VF}_1(n,m_1^2,m_2^2,m_3^2,p^2,L,\mu^2)$.\\
\mytt{double \newfunction{hh21Vb}(n,m1sq,m2sq,m3sq,psq,L,mu2)}:
returns $H^{VF}_{21}(n,m_1^2,m_2^2,m_3^2,p^2,L,\mu^2)$.\\
\mytt{double \newfunction{hh22Vb}(n,m1sq,m2sq,m3sq,psq,L,mu2)}:
returns $H^{VF}_{22}(n,m_1^2,m_2^2,m_3^2,p^2,L,\mu^2)$.\\
\mytt{double \newfunction{hh27Vb}(n,m1sq,m2sq,m3sq,psq,L,mu2)}:
returns $H^{VF}_{27}(n,m_1^2,m_2^2,m_3^2,p^2,L,\mu^2)$.\\
\mytt{double \newfunction{hhdVb}(n,m1sq,m2sq,m3sq,psq,L,mu2)}:
returns $(\partial/\partial p^2)H^{VF}(n,m_1^2,m_2^2,m_3^2,p^2,L,\mu^2)$.\\
\mytt{double \newfunction{hh1dVb}(n,m1sq,m2sq,m3sq,psq,L,mu2)}:
returns $(\partial/\partial p^2)H^{VF}_1(n,m_1^2,m_2^2,m_3^2,p^2,L,\mu^2)$.\\
\mytt{double \newfunction{hh21dVb}(n,m1sq,m2sq,m3sq,psq,L,mu2)}:
returns $(\partial/\partial p^2)H^{VF}_{21}(n,m_1^2,m_2^2,m_3^2,p^2,L,\mu^2)$.\\
\mytt{double \newfunction{hh22dVb}(n,m1sq,m2sq,m3sq,psq,L,mu2)}:
returns $(\partial/\partial p^2)H^{VF}_{22}(n,m_1^2,m_2^2,m_3^2,p^2,L,\mu^2)$.\\
\mytt{double \newfunction{hh27dVb}(n,m1sq,m2sq,m3sq,psq,L,mu2)}:
returns $(\partial/\partial p^2)H^{VF}_{27}(n,m_1^2,m_2^2,m_3^2,p^2,L,\mu^2)$.\\

For all cases discussed both methods, via Bessel or
(generalized) Jacobi theta functions, give the same results.
The derivatives w.r.t. $p^2$ for all the integrals
were compared with taking a numerical derivative.\\

Note that the sunset functions at finite volume call the tadpole integrals
evaluated with the same method. Do not forget to set precision for those
as well.\\

\mytt{void \newfunction{setprecisionfinitevolumesunsett}(const double racc=1e-5,}\\\hspace*{5mm}\mytt{const double rsacc=1e-4,const bool printout=true)}\\
The double values \mytt{sunsetracc} and \mytt{sunsetrsacc}
set the accuracies
of the numerical integration needed when one or two loop-momenta ``feel''
the finite volume. Default values are \mytt{1e-5} and \mytt{1e-4}
respectively.
The bool variable \mytt{printout} defaults to \mytt{true}
and sets whether the setting is
printed. \\

\mytt{void \newfunction{setprecisionfinitevolumesunsetb}(const int maxsum1=100,}
\\\hspace*{5mm}\mytt{
const int maxsum2=40,racc=1e-5,rsacc=1e-4,printout=true)}\\
The integers \mytt{maxsum1} and \mytt{maxsum2} give how far the sum over
Bessel functions
is used for the case with one or two loop momenta
``feeling'' the finite volume. The first is maximum 1000,
the second maximum 40 in the present implementation.
In the latter case a triple sum is needed, hence the much lower upper bound.
The double values \mytt{sunsetracc} and \mytt{sunsetrsacc}
set the accuracies
of the numerical integration which is still needed after the sum for both
cases.\\

For most applications it makes sense to have a higher precision for
the case with one loop momentum quantized, i.e. \mytt{racc} smaller than
\mytt{rsacc}.\\

Defined in \newfunction{finitevolumesunsetintegrals.h},
implemented in \newfunction{finitevolumesunsetintegrals.cc}
and examples of use in \newfunction{testfinitevolumesunsetintegrals.cc}


\section{Three flavour isospin conserving results}

\subsection{Masses, decay constants and vacuum-expectation-values: in physical}
\label{massdecayvev}

The exansions in this subsection are defined in terms of the physical masses,
$m_\pi, m_K, m_\eta$ and the physical pion decay constant $F_\pi$.

\subsubsection{Masses}
\label{masses}

The masses of the pion, kaon and eta at two-loops in three flavour ChPT
were evaluated in \cite{Amoros:1999dp}. The pion and eta mass were done
earlier with a different subtraction scheme and a different way to perform the
sunset integrals in \cite{Golowich:1997zs}.

The expressions for the physical masses for $a=\pi,K,\eta$ are given by
\begin{equation}
m_{a\,\mathrm{phys}}^2 = m_{a\,0}^2+m_a^{2(4)}+m_a^{2(6)}\,.
\end{equation}
The superscripts indicate the order of the diagrams in $p$ that each
contribution comes from.
 
The lowest order masses are
\begin{equation}
m_{\pi\,0}^2 = 2 B_0\hat m\,,\quad m_{K\,0}^2 = B_0\left(\hat m+m_s\right)\,,
\quad m_{\eta\,0}^2 = \frac{2}{3}\left(\hat m+2m_2\right)\,.
\end{equation}
The higher order contributions are split in the parts
depending on the NLO LECs $L_i^r$, on the NNLO LECs $C_i^r$ and the
remainder as
\begin{equation}
\label{defmasses}
m_a^{2(4)}=m_{a\,L}^{2(4)}+m_{a\,R}^{2(4)}\,,
\qquad
m_a^{2(6)}=m_{a\,L}^{2(6)}+m_{a\,C}^{2(6)}+m_{a\,R}^{2(6)}\,.
\end{equation}
The expressions for these can be found in \cite{Amoros:1999dp}
and on \cite{webpage}. Note that when combining these with results
from other sources one should be sure to use a compatible LO and NLO.\\

Pion mass:\\
\mytt{double \newfunction{mpi4}(physmass,Li)}
returns $m_\pi^{2(4)}$\\ 
\mytt{double \newfunction{mpi4L}(physmass,Li)}
returns $m_{\pi\,L}^{2(4)}$\\ 
\mytt{double \newfunction{mpi4R}(physmass,Li)}
returns $m_{\pi\,R}^{2(4)}$\\ 
\mytt{double \newfunction{mpi6}(physmass,Li,Ci)}
returns $m_\pi^{2(6)}$\\ 
\mytt{double \newfunction{mpi6L}(physmass,Li)}
returns $m_{\pi\,L}^{2(6)}$\\ 
\mytt{double \newfunction{mpi6C}(physmass,Ci)}
returns $m_{\pi\,C}^{2(6)}$\\ 
\mytt{double \newfunction{mpi6R}(physmass)}
returns $m_{\pi\,R}^{2(6)}$\\ 

Kaon mass:\\
\mytt{double \newfunction{mk4}(physmass,Li)}
returns $m_K^{2(4)}$\\ 
\mytt{double \newfunction{mk4L}(physmass,Li)}
returns $m_{K\,L}^{2(4)}$\\ 
\mytt{double \newfunction{mk4R}(physmass,Li)}
returns $m_{K\,R}^{2(4)}$\\ 
\mytt{double \newfunction{mk6}(physmass,Li,Ci)}
returns $m_K^{2(6)}$\\ 
\mytt{double \newfunction{mk6L}(physmass,Li)}
returns $m_{K\,L}^{2(6)}$\\ 
\mytt{double \newfunction{mk6C}(physmass,Ci)}
returns $m_{K\,C}^{2(6)}$\\ 
\mytt{double \newfunction{mk6R}(physmass)}
returns $m_{K\,R}^{2(6)}$\\ 

Eta mass:\\
\mytt{double \newfunction{meta4}(physmass,Li)}
returns $m_\eta^{2(4)}$\\ 
\mytt{double \newfunction{meta4L}(physmass,Li)}
returns $m_{\eta\,L}^{2(4)}$\\ 
\mytt{double \newfunction{meta4R}(physmass,Li)}
returns $m_{\eta\,R}^{2(4)}$\\ 
\mytt{double \newfunction{meta6}(physmass,Li,Ci)}
returns $m_\eta^{2(6)}$\\ 
\mytt{double \newfunction{meta6L}(physmass,Li)}
returns $m_{\eta\,L}^{2(6)}$\\ 
\mytt{double \newfunction{meta6C}(physmass,Ci)}
returns $m_{\eta\,C}^{2(6)}$\\ 
\mytt{double \newfunction{meta6R}(physmass)}
returns $m_{\eta\,R}^{2(6)}$\\ 

The functions are defined in \newfunction{massesdecayvev.h},
implemented in \newfunction{massesdecayvev.cc} and examples of use
are in \newfunction{testmassdecayvev.cc}.

\subsubsection{Decay constants}
\label{decay}

The decay constants of the pion, kaon and eta at two-loops in three flavour ChPT
were obtained in \cite{Amoros:1999dp}. The pion and eta decay constants
were done earlier
with a different subtraction scheme and a different way to perform the
sunset integrals in \cite{Golowich:1997zs}.

The expressions for the decay constants for $a=\pi,K,\eta$ are given by
\begin{equation}
\label{defdecayfirst}
F_{a\,\mathrm{phys}} = F_0\left(1+F_a^{(4)}+F_a^{(6)}\right)\,.
\end{equation}
The superscripts indicate the order of the diagrams in $p$ that each
contribution comes from. $F_0$ denotes the decay constant in the three-flavour
chiral limit.
The expressions were originally derived
in \cite{Amoros:1999dp},
but note the description in the erratum of \cite{Amoros:2000mc}.
The expressions corrected for the error can be found in the website
\cite{webpage}. The normalization is such that $F_\pi\approx92$~MeV.

The contributions themselves are divided into the parts
depending on the NLO LECs $L_i^r$, on the NNLO LECs $C_i^r$ and the
remainder as
\begin{equation}
\label{defdecay}
F_a^{(4)}=F_{a\,L}^{(4)}+F_{a\,R}^{(4)}\,,
\qquad
F_a^{(6)}=F_{a\,L}^{(6)}+F_{a\,C}^{(6)}+F_{a\,R}^{(6)}\,.
\end{equation}

For the $\eta$ the decay constant has been
defined with the octet axial-vector current.\\

Pion decay constant:\\
\mytt{double \newfunction{fpi4}(physmass,Li)}
returns $F_\pi^{(4)}$\\ 
\mytt{double \newfunction{fpi4L}(physmass,Li)}
returns $F_{\pi\,L}^{(4)}$\\ 
\mytt{double \newfunction{fpi4R}(physmass,Li)}
returns $F_{\pi\,R}^{(4)}$\\ 
\mytt{double \newfunction{fpi6}(physmass,Li,Ci)}
returns $F_\pi^{(6)}$\\ 
\mytt{double \newfunction{fpi6L}(physmass,Li)}
returns $F_{\pi\,L}^{(6)}$\\ 
\mytt{double \newfunction{fpi6C}(physmass,Ci)}
returns $F_{\pi\,C}^{(6)}$\\ 
\mytt{double \newfunction{fpi6R}(physmass)}
returns $F_{\pi\,R}^{(6)}$\\ 

Kaon decay constant:\\
\mytt{double \newfunction{fk4}(physmass,Li)}
returns $F_K^{(4)}$\\ 
\mytt{double \newfunction{fk4L}(physmass,Li)}
returns $F_{K\,L}^{(4)}$\\ 
\mytt{double \newfunction{fk4R}(physmass,Li)}
returns $F_{K\,R}^{(4)}$\\ 
\mytt{double \newfunction{fk6}(physmass,Li,Ci)}
returns $F_K^{(6)}$\\ 
\mytt{double \newfunction{fk6L}(physmass,Li)}
returns $F_{K\,L}^{(6)}$\\ 
\mytt{double \newfunction{fk6C}(physmass,Ci)}
returns $F_{K\,C}^{(6)}$\\ 
\mytt{double \newfunction{fk6R}(physmass)}
returns $F_{K\,R}^{(6)}$\\ 

Eta decay constant:\\
\mytt{double \newfunction{feta4}(physmass,Li)}
returns $F_\eta^{(4)}$\\ 
\mytt{double \newfunction{feta4L}(physmass,Li)}
returns $F_{\eta\,L}^{(4)}$\\ 
\mytt{double \newfunction{feta4R}(physmass,Li)}
returns $F_{\eta\,R}^{(4)}$\\ 
\mytt{double \newfunction{feta6}(physmass,Li,Ci)}
returns $F_\eta^{(6)}$\\ 
\mytt{double \newfunction{feta6L}(physmass,Li)}
returns $F_{\eta\,L}^{(6)}$\\ 
\mytt{double \newfunction{feta6C}(physmass,Ci)}
returns $F_{\eta\,C}^{(6)}$\\ 
\mytt{double \newfunction{feta6R}(physmass)}
returns $F_{\eta\,R}^{(6)}$\\ 

The functions are defined in \newfunction{massesdecayvev.h},
implemented in \newfunction{massesdecayvev.cc} and examples of use
are in \newfunction{testmassdecayvev.cc}.


\subsubsection{\newfunction{getfpimeta}}
\label{getfpimeta}

A problem that occurs in trying to compare to lattice QCD is that
many of the routines are written in terms of the physical
pion decay constant and physical masses.
In particular, the eta mass is treated as physical. 
One thus needs a consistent eta mass and pion decay constant when
varying the input pion and kaon mass. This assumes we have fitted the
LECs $L_i^r$ and $C_i^r$ with a known set of $m_\pi,m_K,m_\eta,F_\pi$.

With that input we can obtain an eta mass and pion decay constant
with as input values the original \mytt{Liin}, \mytt{Ciin}
and the \mytt{massin}.
The formulas used are (\ref{defmasses}) and (\ref{defdecay}) up to order
$p^6$ and $p^4$.
The solution is obtained by iteration and stops when six digits of precision
are reached. This method was used in \cite{Bijnens:2014dea} to obtain the
consistent set of masses and decay constants used there.\\

\mytt{physmass \newfunction{getfpimeta6}(const double mpiin, const double mkin,}
\\\hspace*{5mm}\mytt{
const physmass massin,const Li Liin, const Ci Ciin)}\\
returns a \mytt{physmass} containing \mytt{mpiin,mkin}
and the calculated compatibe \mytt{meta,fpi} with the formulas 
including order $p^6$, i.e. to NNLO.

\mytt{physmass \newfunction{getfpimeta4}(const double mpiin, const double mkin,}
\\\hspace*{5mm}\mytt{
const physmass massin,const Li Liin)}\\
returns a \mytt{physmass} containing \mytt{mpiin,mkin}
and the calculated compatibe \mytt{meta,fpi} with the formulas 
including order $p^4$, i.e. to NLO.


The functions are defined in \newfunction{getfpimeta.h},
implemented in \newfunction{getfpimeta.cc} and examples of use
are in \newfunction{testgetfpimeta.cc}.

\subsubsection{Vacuum-expectation-values}
\label{vevs}

The corrections to the vacuum expectation values (vevs)
$\langle0\vert \overline q q \vert 0\rangle$
for up, down and strange quarks in the isospin limit
were  calculated at two-loops in three flavour ChPT
in \cite{Amoros:2000mc}.
The expression for the up and down quark vev are identical since we
are in the isospin limit.

We write the expressions in a form analoguous to the decay constant
treatment:
\begin{equation}
\langle0\vert \overline q q \vert 0\rangle_{a\,\mathrm{phys}} =
- F_0^2 B_0\left(1+\langle0\vert \overline q q \vert 0\rangle_a^{(4)}+
\langle0\vert \overline q q \vert 0\rangle_a^{(6)}\right)\,.
\end{equation}
The superscripts indicate the order of the diagrams in $p$ that each
contribution comes from. The lowest order values are $-F_0^2 B_0$.

Note that the vevs are not directly measurable quantities. They depend
on exactly the way the scalar densities are defined in QCD. ChPT can be
used for them when a massindependent, chiral symmetry respecting subtraction
scheme is used. $\overline{MS}$ in QCD satisfies this, but there are other
possibilities. Even within a scheme, $B_0$ and the quark masses depend
on the QCD subtraction scale $\mu_\textrm{QCD}$ in such a way that
$B_0 m_q$ is independent of it. The higher order corrections in this
case also depend on the LECs for fully local counter-terms,
$H_1^r,H_2^r$ at order $p^4$ and $C_{91}^r,\ldots,C_{94}^r$ at $p^6$.
When the scalar density is fully defined, measuring these quantities in
e.g. lattice QCD and comparing with the ChPT expressions is a well defined
procedure.

The contributions at the different orders themselves are split in the parts
depending on the NLO LECs $L_i^r$, on the NNLO LECs $C_i^r$ and the
remainder as
\begin{eqnarray}
\label{defvev}
\langle0\vert \overline q q \vert 0\rangle_a^{(4)}&=&
\langle0\vert \overline q q \vert 0\rangle_{a\,L}^{(4)}
+\langle0\vert \overline q q \vert 0\rangle_{a\,R}^{(4)}\,,
\nonumber\\
\langle0\vert \overline q q \vert 0\rangle_a^{(6)}&=&
\langle0\vert \overline q q \vert 0\rangle_{a\,L}^{(6)}
+\langle0\vert \overline q q \vert 0\rangle_{a\,C}^{(6)}
+\langle0\vert \overline q q \vert 0\rangle_{a\,R}^{(6)}\,.
\end{eqnarray}
These are defined for $q=u,s$.\\

$\langle0\vert \overline q q \vert 0\rangle_{u\,\mathrm{phys}}$:\\
\mytt{double \newfunction{qqup4}(physmass,Li)} returns
  $\langle0\vert \overline q q \vert 0\rangle_{u}^{(4)}$\\
\mytt{double \newfunction{qqup4L}(physmass,Li)} returns
  $\langle0\vert \overline q q \vert 0\rangle_{u\,L}^{(4)}$\\
\mytt{double \newfunction{qqup4R}(physmass)} returns
  $\langle0\vert \overline q q \vert 0\rangle_{u\,R}^{(4)}$\\
\mytt{double \newfunction{qqup6}(physmass,Li,Ci)} returns
  $\langle0\vert \overline q q \vert 0\rangle_{u}^{(6)}$\\
\mytt{double \newfunction{qqup6L}(physmass,Li)} returns
  $\langle0\vert \overline q q \vert 0\rangle_{u\,L}^{(6)}$\\
\mytt{double \newfunction{qqup6C}(physmass,Li)} returns
  $\langle0\vert \overline q q \vert 0\rangle_{u\,C}^{(6)}$\\
\mytt{double \newfunction{qqup6R}(physmass)} returns
  $\langle0\vert \overline q q \vert 0\rangle_{u\,R}^{(6)}$\\

$\langle0\vert \overline q q \vert 0\rangle_{s\,\mathrm{phys}}$:\\
\mytt{double \newfunction{qqstrange4}(physmass,Li)} returns
  $\langle0\vert \overline q q \vert 0\rangle_{s}^{(4)}$\\
\mytt{double \newfunction{qqstrange4L}(physmass,Li)} returns
  $\langle0\vert \overline q q \vert 0\rangle_{s\,L}^{(4)}$\\
\mytt{double \newfunction{qqstrange4R}(physmass)} returns
  $\langle0\vert \overline q q \vert 0\rangle_{s\,R}^{(4)}$\\
\mytt{double \newfunction{qqstrange6}(physmass,Li,Ci)} returns
  $\langle0\vert \overline q q \vert 0\rangle_{s}^{(6)}$\\
\mytt{double \newfunction{qqstrange6L}(physmass,Li)} returns
  $\langle0\vert \overline q q \vert 0\rangle_{s\,L}^{(6)}$\\
\mytt{double \newfunction{qqstrange6C}(physmass,Li)} returns
  $\langle0\vert \overline q q \vert 0\rangle_{s\,C}^{(6)}$\\
\mytt{double \newfunction{qqstrange6R}(physmass)} returns
  $\langle0\vert \overline q q \vert 0\rangle_{s\,R}^{(6)}$\\


The functions are defined in \newfunction{massesdecayvev.h},
implemented in \newfunction{massesdecayvev.cc} and examples of use
are in \newfunction{testmassesdecayvev.cc}

\subsection{Masses, decay constants and vacuum-expectation-values: in lowest order}
\label{massdecayvevlo}

The exansions in this subsection are defined in terms of the
lowest order masses,
$m_{\pi\,0}, m_{K\,0}$, $m_\eta=\sqrt{(4m_{K\,0}^2-m_{\pi\,0}^2)/3}$ and the 
lowest order, or chiral limit, pion decay constant $F_0$.

\subsubsection{Masses: in lowest order}
\label{masseslo}

The masses of the pion, kaon and eta at two-loops in three flavour ChPT
were evaluated in \cite{Amoros:1999dp}. The pion and eta mass were done
earlier with a different subtraction scheme and a different way to perform the
sunset integrals in \cite{Golowich:1997zs}.

The expressions for the physical masses for $a=\pi,K,\eta$ are given by
\begin{equation}
\label{massesinlo}
m_{a\,\mathrm{phys}}^2 = m_{a\,0}^2+m_a^{2(4)0}+m_a^{2(6)0}\,.
\end{equation}
The superscripts indicate the order of the diagrams in $p$ that each
contribution comes from and the extra $0$ that it is defined in terms
of lowest-order quantities.
 
The lowest order masses are
\begin{equation}
m_{\pi\,0}^2 = 2 B_0\hat m\,,\quad m_{K\,0}^2 = B_0\left(\hat m+m_s\right)\,,
\quad m_{\eta\,0}^2 = \frac{2}{3}\left(\hat m+2m_2\right)\,.
\end{equation}
The higher order contributions are split in the parts
depending on the NLO LECs $L_i^r$, on the NNLO LECs $C_i^r$ and the
remainder as
\begin{equation}
\label{defmasseslo}
m_a^{2(4)0}=m_{a\,L}^{2(4)0}+m_{a\,R}^{2(4)0}\,,
\qquad
m_a^{2(6)0}=m_{a\,L}^{2(6)0}+m_{a\,C}^{2(6)0}+m_{a\,R}^{2(6)0}\,.
\end{equation}
The expressions for were derived during the work for \cite{Amoros:1999dp}
and on \cite{webpage}. Note that when combining these with results
from other sources one should be sure to use a compatible LO and NLO.\\

Pion mass:\\
\mytt{double \newfunction{mpi4lo}(lomass,Li)}
returns $m_\pi^{2(4)0}$\\ 
\mytt{double \newfunction{mpi4Llo}(lomass,Li)}
returns $m_{\pi\,L}^{2(4)0}$\\ 
\mytt{double \newfunction{mpi4Rlo}(lomass,Li)}
returns $m_{\pi\,R}^{2(4)0}$\\ 
\mytt{double \newfunction{mpi6lo}(lomass,Li,Ci)}
returns $m_\pi^{2(6)0}$\\ 
\mytt{double \newfunction{mpi6Llo}(lomass,Li)}
returns $m_{\pi\,L}^{2(6)0}$\\ 
\mytt{double \newfunction{mpi6Clo}(lomass,Ci)}
returns $m_{\pi\,C}^{2(6)0}$\\ 
\mytt{double \newfunction{mpi6Rlo}(lomass)}
returns $m_{\pi\,R}^{2(6)0}$\\ 

Kaon mass:\\
\mytt{double \newfunction{mk4lo}(lomass,Li)}
returns $m_K^{2(4)0}$\\ 
\mytt{double \newfunction{mk4Llo}(lomass,Li)}
returns $m_{K\,L}^{2(4)0}$\\ 
\mytt{double \newfunction{mk4Rlo}(lomass,Li)}
returns $m_{K\,R}^{2(4)0}$\\ 
\mytt{double \newfunction{mk6lo}(lomass,Li,Ci)}
returns $m_K^{2(6)0}$\\ 
\mytt{double \newfunction{mk6Llo}(lomass,Li)}
returns $m_{K\,L}^{2(6)0}$\\ 
\mytt{double \newfunction{mk6Clo}(lomass,Ci)}
returns $m_{K\,C}^{2(6)0}$\\ 
\mytt{double \newfunction{mk6Rlo}(lomass)}
returns $m_{K\,R}^{2(6)0}$\\ 

Eta mass:\\
\mytt{double \newfunction{meta4lo}(lomass,Li)}
returns $m_\eta^{2(4)0}$\\ 
\mytt{double \newfunction{meta4Llo}(lomass,Li)}
returns $m_{\eta\,L}^{2(4)0}$\\ 
\mytt{double \newfunction{meta4Rlo}(lomass,Li)}
returns $m_{\eta\,R}^{2(4)0}$\\ 
\mytt{double \newfunction{meta6lo}(lomass,Li,Ci)}
returns $m_\eta^{2(6)0}$\\ 
\mytt{double \newfunction{meta6Llo}(lomass,Li)}
returns $m_{\eta\,L}^{2(6)0}$\\ 
\mytt{double \newfunction{meta6Clo}(lomass,Ci)}
returns $m_{\eta\,C}^{2(6)0}$\\ 
\mytt{double \newfunction{meta6Rlo}(lomass)}
returns $m_{\eta\,R}^{2(6)0}$\\ 

The functions are defined in \newfunction{massesdecayvevlo.h},
implemented in \newfunction{massesdecayvevlo.cc} and examples of use
are in \newfunction{testmassdecayvevlo.cc}.

\subsubsection{Decay constants: in lowest order}
\label{decaylo}

The decay constants of the pion, kaon and eta at two-loops in three flavour ChPT
were obtained in \cite{Amoros:1999dp}. The pion and eta decay constants
were done earlier
with a different subtraction scheme and a different way to perform the
sunset integrals in \cite{Golowich:1997zs}.

The expressions for the decay constants for $a=\pi,K,\eta$ are given by
\begin{equation}
\label{defdecayfirstlo}
F_{a\,\mathrm{phys}} = F_0\left(1+F_a^{(4)0}+F_a^{(6)0}\right)\,.
\end{equation}
The superscripts indicate the order of the diagrams in $p$ that each
contribution comes from. The extra $0$ indicates that the expansion
is in terms of lowest-order quantities. 
$F_0$ denotes the decay constant in the three-flavour
chiral limit.
The expressions were originally derived during the work for
\cite{Amoros:1999dp} and can be found in the website
\cite{webpage}. The normalization is such that $F_\pi\approx92$~MeV.

The contributions themselves are divided into the parts
depending on the NLO LECs $L_i^r$, on the NNLO LECs $C_i^r$ and the
remainder as
\begin{equation}
\label{defdecaylo}
F_a^{(4)0}=F_{a\,L}^{(4)0}+F_{a\,R}^{(4)0}\,,
\qquad
F_a^{(6)0}=F_{a\,L}^{(6)0}+F_{a\,C}^{(6)0}+F_{a\,R}^{(6)0}\,.
\end{equation}

For the $\eta$ the decay constant has been
defined with the octet axial-vector current.\\

Pion decay constant:\\
\mytt{double \newfunction{fpi4lo}(lomass,Li)}
returns $F_\pi^{(4)0}$\\ 
\mytt{double \newfunction{fpi4Llo}(lomass,Li)}
returns $F_{\pi\,L}^{(4)0}$\\ 
\mytt{double \newfunction{fpi4Rlo}(lomass,Li)}
returns $F_{\pi\,R}^{(4)0}$\\ 
\mytt{double \newfunction{fpi6lo}(lomass,Li,Ci)}
returns $F_\pi^{(6)0}$\\ 
\mytt{double \newfunction{fpi6Llo}(lomass,Li)}
returns $F_{\pi\,L}^{(6)0}$\\ 
\mytt{double \newfunction{fpi6Clo}(lomass,Ci)}
returns $F_{\pi\,C}^{(6)0}$\\ 
\mytt{double \newfunction{fpi6Rlo}(lomass)}
returns $F_{\pi\,R}^{(6)0}$\\ 

Kaon decay constant:\\
\mytt{double \newfunction{fk4lo}(lomass,Li)}
returns $F_K^{(4)0}$\\ 
\mytt{double \newfunction{fk4Llo}(lomass,Li)}
returns $F_{K\,L}^{(4)0}$\\ 
\mytt{double \newfunction{fk4Rlo}(lomass,Li)}
returns $F_{K\,R}^{(4)0}$\\ 
\mytt{double \newfunction{fk6lo}(lomass,Li,Ci)}
returns $F_K^{(6)0}$\\ 
\mytt{double \newfunction{fk6Llo}(lomass,Li)}
returns $F_{K\,L}^{(6)0}$\\ 
\mytt{double \newfunction{fk6Clo}(lomass,Ci)}
returns $F_{K\,C}^{(6)0}$\\ 
\mytt{double \newfunction{fk6Rlo}(lomass)}
returns $F_{K\,R}^{(6)0}$\\ 

Eta decay constant:\\
\mytt{double \newfunction{feta4lo}(lomass,Li)}
returns $F_\eta^{(4)0}$\\ 
\mytt{double \newfunction{feta4Llo}(lomass,Li)}
returns $F_{\eta\,L}^{(4)0}$\\ 
\mytt{double \newfunction{feta4Rlo}(lomass,Li)}
returns $F_{\eta\,R}^{(4)0}$\\ 
\mytt{double \newfunction{feta6lo}(lomass,Li,Ci)}
returns $F_\eta^{(6)0}$\\ 
\mytt{double \newfunction{feta6Llo}(lomass,Li)}
returns $F_{\eta\,L}^{(6)0}$\\ 
\mytt{double \newfunction{feta6Clo}(lomass,Ci)}
returns $F_{\eta\,C}^{(6)0}$\\ 
\mytt{double \newfunction{feta6Rlo}(lomass)}
returns $F_{\eta\,R}^{(6)0}$\\ 

The functions are defined in \newfunction{massesdecayvevlo.h},
implemented in \newfunction{massesdecayvevlo.cc} and examples of use
are in \newfunction{testmassdecayvevlo.cc}.


\subsubsection{Vacuum-expectation-values: in lowest order}
\label{vevslo}

The corrections to the vacuum expectation values (vevs)
$\langle0\vert \overline q q \vert 0\rangle$
for up, down and strange quarks in the isospin limit
were  calculated at two-loops in three flavour ChPT
in \cite{Amoros:2000mc}.
The expression for the up and down quark vev are identical since we
are in the isospin limit.

We write the expressions in a form analoguous to the decay constant
treatment:
\begin{equation}
\langle0\vert \overline q q \vert 0\rangle_{a\,\mathrm{phys}} =
- F_0^2 B_0\left(1+\langle0\vert \overline q q \vert 0\rangle_a^{(4)0}+
\langle0\vert \overline q q \vert 0\rangle_a^{(6)0}\right)\,.
\end{equation}
The superscripts indicate the order of the diagrams in $p$ that each
contribution comes from. The extra $0$ indicates that the expansion
is defined in terms of lowest-order quantities.
The lowest order values are $-F_0^2 B_0$.

Note that the vevs are not directly measurable quantities. They depend
on exactly the way the scalar densities are defined in QCD. ChPT can be
used for them when a massindependent, chiral symmetry respecting subtraction
scheme is used. $\overline{MS}$ in QCD satisfies this, but there are other
possibilities. Even within a scheme, $B_0$ and the quark masses depend
on the QCD subtraction scale $\mu_\textrm{QCD}$ in such a way that
$B_0 m_q$ is independent of it. The higher order corrections in this
case also depend on the LECs for fully local counter-terms,
$H_1^r,H_2^r$ at order $p^4$ and $C_{91}^r,\ldots,C_{94}^r$ at $p^6$.
When the scalar density is fully defined, measuring these quantities in
e.g. lattice QCD and comparing with the ChPT expressions is a well defined
procedure.

The contributions at the different orders themselves are split in the parts
depending on the NLO LECs $L_i^r$, on the NNLO LECs $C_i^r$ and the
remainder as
\begin{eqnarray}
\label{defvevlo}
\langle0\vert \overline q q \vert 0\rangle_a^{(4)0}&=&
\langle0\vert \overline q q \vert 0\rangle_{a\,L}^{(4)0}
+\langle0\vert \overline q q \vert 0\rangle_{a\,R}^{(4)0}\,,
\nonumber\\
\langle0\vert \overline q q \vert 0\rangle_a^{(6)0}&=&
\langle0\vert \overline q q \vert 0\rangle_{a\,L}^{(6)0}
+\langle0\vert \overline q q \vert 0\rangle_{a\,C}^{(6)0}
+\langle0\vert \overline q q \vert 0\rangle_{a\,R}^{(6)0}\,.
\end{eqnarray}
These are defined for $q=u,s$.\\

$\langle0\vert \overline q q \vert 0\rangle_{u\,\mathrm{phys}}$:\\
\mytt{double \newfunction{qqup4lo}(lomass,Li)} returns
  $\langle0\vert \overline q q \vert 0\rangle_{u}^{(4)0}$\\
\mytt{double \newfunction{qqup4Llo}(lomass,Li)} returns
  $\langle0\vert \overline q q \vert 0\rangle_{u\,L}^{(4)0}$\\
\mytt{double \newfunction{qqup4Rlo}(lomass)} returns
  $\langle0\vert \overline q q \vert 0\rangle_{u\,R}^{(4)0}$\\
\mytt{double \newfunction{qqup6lo}(lomass,Li,Ci)} returns
  $\langle0\vert \overline q q \vert 0\rangle_{u}^{(6)0}$\\
\mytt{double \newfunction{qqup6Llo}(lomass,Li)} returns
  $\langle0\vert \overline q q \vert 0\rangle_{u\,L}^{(6)0}$\\
\mytt{double \newfunction{qqup6Clo}(lomass,Li)} returns
  $\langle0\vert \overline q q \vert 0\rangle_{u\,C}^{(6)0}$\\
\mytt{double \newfunction{qqup6Rlo}(lomass)} returns
  $\langle0\vert \overline q q \vert 0\rangle_{u\,R}^{(6)0}$\\

$\langle0\vert \overline q q \vert 0\rangle_{s\,\mathrm{phys}}$:\\
\mytt{double \newfunction{qqstrange4lo}(lomass,Li)} returns
  $\langle0\vert \overline q q \vert 0\rangle_{s}^{(4)0}$\\
\mytt{double \newfunction{qqstrange4Llo}(lomass,Li)} returns
  $\langle0\vert \overline q q \vert 0\rangle_{s\,L}^{(4)0}$\\
\mytt{double \newfunction{qqstrange4Rlo}(lomass)} returns
  $\langle0\vert \overline q q \vert 0\rangle_{s\,R}^{(4)0}$\\
\mytt{double \newfunction{qqstrange6lo}(lomass,Li,Ci)} returns
  $\langle0\vert \overline q q \vert 0\rangle_{s}^{(6)0}$\\
\mytt{double \newfunction{qqstrange6Llo}(lomass,Li)} returns
  $\langle0\vert \overline q q \vert 0\rangle_{s\,L}^{(6)0}$\\
\mytt{double \newfunction{qqstrange6Clo}(lomass,Li)} returns
  $\langle0\vert \overline q q \vert 0\rangle_{s\,C}^{(6)0}$\\
\mytt{double \newfunction{qqstrange6Rlo}(lomass)} returns
  $\langle0\vert \overline q q \vert 0\rangle_{s\,R}^{(6)0}$\\


The functions are defined in \newfunction{massesdecayvevlo.h},
implemented in \newfunction{massesdecayvevlo.cc} and examples of use
are in \newfunction{testmassesdecayvevlo.cc}


\subsection{Masses and decay constants at finite volume: in physical}
\label{massdecayvevV}

The expressions treated in this section have been
derived in \cite{Bijnens:2014dea}. A general remark is that care should be
taken to set the precision in the loop integrals sufficiently high.
For the one-loop integrals setting it very high is usually no problem.
For the sunset integrals the evaluation can become very slow. It is
strongly recommended to play around with the settings and compare the outputs
for the two ways to evaluate the integral. 
The theta and Bessel function evaluation approach the correct answer
differently.
For most cases
it is possible to have \mytt{rsacc} set smaller than \mytt{racc}.

For many applications it is useful to calculate the
very time consuming parts, those labeled \mytt{6RV}, once and store them.
They only depend nontrivially on the masses and size of the finite volume.
The decay constant dependence is very simple, an overall factor at each order,
and there is no dependence
on the NLO LECs $L_i^r$.

The results presented in this section are with periodic boundary conditions
and an infinite extension in the time direction. They are also restricted
to the case where the particle is at rest, i.e. $\vec p=0$.

\subsubsection{Masses at finite volume: in physical}

The finite volume corrections to the masses squared\footnote{Note that
in other papers the corrections to the mass itself are sometimes
quoted.} are defined as the difference of the mass squared in finite volume
and in infinite volume:
\begin{eqnarray}
\Delta^V m^2_a &=& m^{2V}_a-m^{2\,V=\infty}_a
= m_a^{2V(4)}+ m_a^{2V(6)}\,.
\nonumber\\
m_a^{2V(6)} &=& m_{a\,L}^{2V(6)}+m_{a\,R}^{2V(6)}\,.
\end{eqnarray}
These definitions are for $a=\pi,K,\eta$.\\

Pion mass (theta function method):\\
\mytt{double \newfunction{mpi4Vt}(const physmass massin,const double L)}
returns $m_\pi^{2V(4)}$.\\
\mytt{double \newfunction{mpi6Vt}(const physmass massin,const Li Liin,const double L)}
returns $m_\pi^{2V(6)}$.\\
\mytt{double \newfunction{mpi6VLt}(const physmass massin,const Li Liin,const double L)}
returns $m_{\pi\,L}^{2V(6)}$.\\
\mytt{double \newfunction{mpi6VRt}(const physmass massin,const double L)}
returns $m_{\pi\,R}^{2V(6)}$.\\

Pion mass (Bessel function method):\\
\mytt{double \newfunction{mpi4Vb}(const physmass massin,const double L)}
returns $m_\pi^{2V(4)}$.\\
\mytt{double \newfunction{mpi6Vb}(const physmass massin,const Li Liin,const double L)}
returns $m_\pi^{2V(6)}$.\\
\mytt{double \newfunction{mpi6VLb}(const physmass massin,const Li Liin,const double L)}
returns $m_{\pi\,L}^{2V(6)}$.\\
\mytt{double \newfunction{mpi6VRb}(const physmass massin,const double L)}
returns $m_{\pi\,R}^{2V(6)}$.\\

Kaon mass (theta function method):\\
\mytt{double \newfunction{mk4Vt}(const physmass massin,const double L)}
returns $m_K^{2V(4)}$.\\
\mytt{double \newfunction{mk6Vt}(const physmass massin,const Li Liin,const double L)}
returns $m_K^{2V(6)}$.\\
\mytt{double \newfunction{mk6VLt}(const physmass massin,const Li Liin,const double L)}
returns $m_{K\,L}^{2V(6)}$.\\
\mytt{double \newfunction{mk6VRt}(const physmass massin,const double L)}
returns $m_{K\,R}^{2V(6)}$.\\

Kaon mass (Bessel function method):\\
\mytt{double \newfunction{mk4Vb}(const physmass massin,const double L)}
returns $m_K^{2V(4)}$.\\
\mytt{double \newfunction{mk6Vb}(const physmass massin,const Li Liin,const double L)}
returns $m_K^{2V(6)}$.\\
\mytt{double \newfunction{mk6VLb}(const physmass massin,const Li Liin,const double L)}
returns $m_{K\,L}^{2V(6)}$.\\
\mytt{double \newfunction{mk6VRb}(const physmass massin,const double L)}
returns $m_{K\,R}^{2V(6)}$.\\

Eta mass (theta function method):\\
\mytt{double \newfunction{meta4Vt}(const physmass massin,const double L)}
returns $m_\eta^{2V(4)}$.\\
\mytt{double \newfunction{meta6Vt}(const physmass massin,const Li Liin,const double L)}
returns $m_\eta^{2V(6)}$.\\
\mytt{double \newfunction{meta6VLt}(const physmass massin,const Li Liin,const double L)}
returns $m_{\eta\,L}^{2V(6)}$.\\
\mytt{double \newfunction{meta6VRt}(const physmass massin,const double L)}
returns $m_{\eta\,R}^{2V(6)}$.\\

Eta mass (Bessel function method):\\
\mytt{double \newfunction{meta4Vb}(const physmass massin,const double L)}
returns $m_\eta^{2V(4)}$.\\
\mytt{double \newfunction{meta6Vb}(const physmass massin,const Li Liin,const double L)}
returns $m_\eta^{2V(6)}$.\\
\mytt{double \newfunction{meta6VLb}(const physmass massin,const Li Liin,const double L)}
returns $m_{\eta\,L}^{2V(6)}$.\\
\mytt{double \newfunction{meta6VRb}(const physmass massin,const double L)}
returns $m_{\eta\,R}^{2V(6)}$.\\

All these are defined in \newfunction{massdecayvevV.h}
and implemented in \newfunction{massdecayvevV.h}.
Examples of use are in \newfunction{testmassdecayvevV.cc}.

\subsubsection{Decay constants at finite volume: in physical}

The finite volume corrections to the decay constants
are defined as the difference of the decay constant in finite volume
and in infinite volume:
\begin{eqnarray}
\label{defdecayV}
\Delta^V F_a &=& F^{V}_a-F^{\,V=\infty}_a
= F_a^{V(4)}+ F_a^{V(6)}\,.
\nonumber\\
F_a^{V(6)} &=& F_{a\,L}^{V(6)}+F_{a\,R}^{V(6)}\,.
\end{eqnarray}
These definitions are for $a=\pi,K,\eta$.
Note that the correction is defined to the value of the
decay constant, not divided by the the lowest order decay constant as
in (\ref{defdecayfirst}).
The eta decay constant is defined with the octet axial current.\\

Pion decay constant (theta function method):\\
\mytt{double \newfunction{fpi4Vt}(const physmass massin,const double L)}
returns $F_\pi^{V(4)}$.\\
\mytt{double \newfunction{fpi6Vt}(const physmass massin,const Li Liin,const double L)}
returns $F_\pi^{V(6)}$.\\
\mytt{double \newfunction{fpi6VLt}(const physmass massin,const Li Liin,const double L)}
returns $F_{\pi\,L}^{V(6)}$.\\
\mytt{double \newfunction{fpi6VRt}(const physmass massin,const double L)}
returns $F_{\pi\,R}^{V(6)}$.\\

Pion decay constant (Bessel function method):\\
\mytt{double \newfunction{fpi4Vb}(const physmass massin,const double L)}
returns $F_\pi^{V(4)}$.\\
\mytt{double \newfunction{fpi6Vb}(const physmass massin,const Li Liin,const double L)}
returns $F_\pi^{V(6)}$.\\
\mytt{double \newfunction{fpi6VLb}(const physmass massin,const Li Liin,const double L)}
returns $F_{\pi\,L}^{V(6)}$.\\
\mytt{double \newfunction{fpi6VRb}(const physmass massin,const double L)}
returns $F_{\pi\,R}^{V(6)}$.\\

Kaon decay constant (theta function method):\\
\mytt{double \newfunction{fk4Vt}(const physmass massin,const double L)}
returns $F_K^{V(4)}$.\\
\mytt{double \newfunction{fk6Vt}(const physmass massin,const Li Liin,const double L)}
returns $F_K^{V(6)}$.\\
\mytt{double \newfunction{fk6VLt}(const physmass massin,const Li Liin,const double L)}
returns $F_{K\,L}^{V(6)}$.\\
\mytt{double \newfunction{fk6VRt}(const physmass massin,const double L)}
returns $F_{K\,R}^{V(6)}$.\\

Kaon decay constant (Bessel function method):\\
\mytt{double \newfunction{fk4Vb}(const physmass massin,const double L)}
returns $F_K^{V(4)}$.\\
\mytt{double \newfunction{fk6Vb}(const physmass massin,const Li Liin,const double L)}
returns $F_K^{V(6)}$.\\
\mytt{double \newfunction{fk6VLb}(const physmass massin,const Li Liin,const double L)}
returns $F_{K\,L}^{V(6)}$.\\
\mytt{double \newfunction{fk6VRb}(const physmass massin,const double L)}
returns $F_{K\,R}^{V(6)}$.\\

Eta decay constant (theta function method):\\
\mytt{double \newfunction{feta4Vt}(const physmass massin,const double L)}
returns $F_\eta^{V(4)}$.\\
\mytt{double \newfunction{feta6Vt}(const physmass massin,const Li Liin,const double L)}
returns $F_\eta^{V(6)}$.\\
\mytt{double \newfunction{feta6VLt}(const physmass massin,const Li Liin,const double L)}
returns $F_{\eta\,L}^{V(6)}$.\\
\mytt{double \newfunction{feta6VRt}(const physmass massin,const double L)}
returns $F_{\eta\,R}^{V(6)}$.\\

Eta decay constant (Bessel function method):\\
\mytt{double \newfunction{feta4Vb}(const physmass massin,const double L)}
returns $F_\eta^{V(4)}$.\\
\mytt{double \newfunction{feta6Vb}(const physmass massin,const Li Liin,const double L)}
returns $F_\eta^{V(6)}$.\\
\mytt{double \newfunction{feta6VLb}(const physmass massin,const Li Liin,const double L)}
returns $F_{\eta\,L}^{V(6)}$.\\
\mytt{double \newfunction{feta6VRb}(const physmass massin,const double L)}
returns $F_{\eta\,R}^{V(6)}$.\\

All these are defined in \newfunction{massdecayvevV.h}
and implemented in \newfunction{massdecayvevV.h}.
Examples of use are in \newfunction{testmassdecayvevV.cc}.

\subsection{Masses, decay constants and vacuum expectation
values at finite volume: in lowest order}
\label{massdecayvevloV}

The expressions treated in this section have been
derived in \cite{Bijnens:2014dea}. A general remark is that care should be
taken to set the precision in the loop integrals sufficiently high.
For the one-loop integrals setting it very high is usually no problem.
For the sunset integrals the evaluation can become very slow. It is
strongly recommended to play around with the settings and compare the outputs
for the two ways to evaluate the integral. 
The theta and Bessel function evaluation approach the correct answer
differently.
For most cases
it is possible to have \mytt{rsacc} set smaller than \mytt{racc}.

For many applications it is useful to calculate the
very time consuming parts, those labeled \mytt{6RV}, once and store them.
They only depend nontrivially on the masses and size of the finite volume.
The decay constant dependence is very simple, an overall factor at each order,
and there is no dependence
on the NLO LECs $L_i^r$.

The results presented in this section are with periodic boundary conditions
and an infinite extension in the time direction. They are also restricted
to the case where the particle is at rest, i.e. $\vec p=0$.

\subsubsection{Masses at finite volume: in lowest order}

The finite volume corrections to the masses squared\footnote{Note that
in other papers the corrections to the mass itself are sometimes
quoted.} are defined as the difference of the mass squared in finite volume
and in infinite volume:
\begin{eqnarray}
\label{defmassloV}
\Delta^V m^2_a &=& m^{2V}_a-m^{2\,V=\infty}_a
= m_a^{2V(4)0}+ m_a^{2V(6)0}\,.
\nonumber\\
m_a^{2V(6)0} &=& m_{a\,L}^{2V(6)0}+m_{a\,R}^{2V(6)0}\,.
\end{eqnarray}
These definitions are for $a=\pi,K,\eta$.\\

Pion mass (theta function method):\\
\mytt{double \newfunction{mpi4loVt}(const lomass massin,const double L)}
returns $m_\pi^{2V(4)0}$.\\
\mytt{double \newfunction{mpi6loVt}(const lomass massin,const Li Liin,const double L)}
returns $m_\pi^{2V(6)0}$.\\
\mytt{double \newfunction{mpi6LloVt}(const lomass massin,const Li Liin,const double L)}
returns $m_{\pi\,L}^{2V(6)0}$.\\
\mytt{double \newfunction{mpi6RloVt}(const lomass massin,const double L)}
returns $m_{\pi\,R}^{2V(6)0}$.\\

Pion mass (Bessel function method):\\
\mytt{double \newfunction{mpi4loVb}(const lomass massin,const double L)}
returns $m_\pi^{2V(4)0}$.\\
\mytt{double \newfunction{mpi6loVb}(const lomass massin,const Li Liin,const double L)}
returns $m_\pi^{2V(6)0}$.\\
\mytt{double \newfunction{mpi6LloVb}(const lomass massin,const Li Liin,const double L)}
returns $m_{\pi\,L}^{2V(6)0}$.\\
\mytt{double \newfunction{mpi6RloVb}(const lomass massin,const double L)}
returns $m_{\pi\,R}^{2V(6)0}$.\\

Kaon mass (theta function method):\\
\mytt{double \newfunction{mk4loVt}(const lomass massin,const double L)}
returns $m_K^{2V(4)0}$.\\
\mytt{double \newfunction{mk6loVt}(const lomass massin,const Li Liin,const double L)}
returns $m_K^{2V(6)0}$.\\
\mytt{double \newfunction{mk6LloVt}(const lomass massin,const Li Liin,const double L)}
returns $m_{K\,L}^{2V(6)0}$.\\
\mytt{double \newfunction{mk6RloVt}(const lomass massin,const double L)}
returns $m_{K\,R}^{2V(6)0}$.\\

Kaon mass (Bessel function method):\\
\mytt{double \newfunction{mk4loVb}(const lomass massin,const double L)}
returns $m_K^{2V(4)0}$.\\
\mytt{double \newfunction{mk6loVb}(const lomass massin,const Li Liin,const double L)}
returns $m_K^{2V(6)0}$.\\
\mytt{double \newfunction{mk6LloVb}(const lomass massin,const Li Liin,const double L)}
returns $m_{K\,L}^{2V(6)0}$.\\
\mytt{double \newfunction{mk6RloVb}(const lomass massin,const double L)}
returns $m_{K\,R}^{2V(6)}$.\\

Eta mass (theta function method):\\
\mytt{double \newfunction{meta4loVt}(const lomass massin,const double L)}
returns $m_\eta^{2V(4)0}$.\\
\mytt{double \newfunction{meta6loVt}(const lomass massin,const Li Liin,const double L)}
returns $m_\eta^{2V(6)0}$.\\
\mytt{double \newfunction{meta6LloVt}(const lomass massin,const Li Liin,const double L)}
returns $m_{\eta\,L}^{2V(6)0}$.\\
\mytt{double \newfunction{meta6RloVt}(const lomass massin,const double L)}
returns $m_{\eta\,R}^{2V(6)0}$.\\

Eta mass (Bessel function method):\\
\mytt{double \newfunction{meta4loVb}(const lomass massin,const double L)}
returns $m_\eta^{2V(4)0}$.\\
\mytt{double \newfunction{meta6loVb}(const lomass massin,const Li Liin,const double L)}
returns $m_\eta^{2V(6)0}$.\\
\mytt{double \newfunction{meta6LloVb}(const lomass massin,const Li Liin,const double L)}
returns $m_{\eta\,L}^{2V(6)0}$.\\
\mytt{double \newfunction{meta6RloVb}(const lomass massin,const double L)}
returns $m_{\eta\,R}^{2V(6)0}$.\\

All these are defined in \newfunction{massdecayvevloV.h}
and implemented in \newfunction{massdecayvevloV.h}.
Examples of use are in \newfunction{testmassdecayvevloV.cc}.

\subsubsection{Decay constants at finite volume: in lowest order}

The finite volume corrections to the decay constants
are defined as the difference of the decay constant in finite volume
and in infinite volume:
\begin{eqnarray}
\label{defdecayloV}
\Delta^V F_a &=& F^{V}_a-F^{\,V=\infty}_a
= F_0\left(F_a^{V(4)0}+ F_a^{V(6)0}\right)\,.
\nonumber\\
F_a^{V(6)0} &=& F_{a\,L}^{V(6)0}+F_{a\,R}^{V(6)0}\,.
\end{eqnarray}
Note that this is a different normalization compared to the expressions
in terms of physical masses and the physical $F_\pi$.
This was done to have the same normalization as the partially quenched
results.
These definitions are for $a=\pi,K,\eta$.
The correction is defined to the value of the
decay constant divided by the the lowest order decay constant as
in (\ref{defdecayfirst}).
The eta decay constant is defined with the octet axial current.\\

Pion decay constant (theta function method):\\
\mytt{double \newfunction{fpi4loVt}(const lomass massin,const double L)}
returns $F_\pi^{V(4)0}$.\\
\mytt{double \newfunction{fpi6loVt}(const lomass massin,const Li Liin,const double L)}
returns $F_\pi^{V(6)0}$.\\
\mytt{double \newfunction{fpi6LloVt}(const lomass massin,const Li Liin,const double L)}
returns $F_{\pi\,L}^{V(6)0}$.\\
\mytt{double \newfunction{fpi6RloVt}(const lomass massin,const double L)}
returns $F_{\pi\,R}^{V(6)0}$.\\

Pion decay constant (Bessel function method):\\
\mytt{double \newfunction{fpi4loVb}(const lomass massin,const double L)}
returns $F_\pi^{V(4)0}$.\\
\mytt{double \newfunction{fpi6loVb}(const lomass massin,const Li Liin,const double L)}
returns $F_\pi^{V(6)0}$.\\
\mytt{double \newfunction{fpi6LloVb}(const lomass massin,const Li Liin,const double L)}
returns $F_{\pi\,L}^{V(6)0}$.\\
\mytt{double \newfunction{fpi6RloVb}(const lomass massin,const double L)}
returns $F_{\pi\,R}^{V(6)0}$.\\

Kaon decay constant (theta function method):\\
\mytt{double \newfunction{fk4loVt}(const lomass massin,const double L)}
returns $F_K^{V(4)0}$.\\
\mytt{double \newfunction{fk6loVt}(const lomass massin,const Li Liin,const double L)}
returns $F_K^{V(6)0}$.\\
\mytt{double \newfunction{fk6LloVt}(const lomass massin,const Li Liin,const double L)}
returns $F_{K\,L}^{V(6)0}$.\\
\mytt{double \newfunction{fk6RloVt}(const lomass massin,const double L)}
returns $F_{K\,R}^{V(6)0}$.\\

Kaon decay constant (Bessel function method):\\
\mytt{double \newfunction{fk4loVb}(const lomass massin,const double L)}
returns $F_K^{V(4)0}$.\\
\mytt{double \newfunction{fk6loVb}(const lomass massin,const Li Liin,const double L)}
returns $F_K^{V(6)0}$.\\
\mytt{double \newfunction{fk6LloVb}(const lomass massin,const Li Liin,const double L)}
returns $F_{K\,L}^{V(6)0}$.\\
\mytt{double \newfunction{fk6RloVb}(const lomass massin,const double L)}
returns $F_{K\,R}^{V(6)0}$.\\

Eta decay constant (theta function method):\\
\mytt{double \newfunction{feta4loVt}(const lomass massin,const double L)}
returns $F_\eta^{V(4)0}$.\\
\mytt{double \newfunction{feta6loVt}(const lomass massin,const Li Liin,const double L)}
returns $F_\eta^{V(6)0}$.\\
\mytt{double \newfunction{feta6LloVt}(const lomass massin,const Li Liin,const double L)}
returns $F_{\eta\,L}^{V(6)0}$.\\
\mytt{double \newfunction{feta6RloVt}(const lomass massin,const double L)}
returns $F_{\eta\,R}^{V(6)0}$.\\

Eta decay constant (Bessel function method):\\
\mytt{double \newfunction{feta4loVb}(const lomass massin,const double L)}
returns $F_\eta^{V(4)0}$.\\
\mytt{double \newfunction{feta6loVb}(const lomass massin,const Li Liin,const double L)}
returns $F_\eta^{V(6)0}$.\\
\mytt{double \newfunction{feta6LloVb}(const lomass massin,const Li Liin,const double L)}
returns $F_{\eta\,L}^{V(6)0}$.\\
\mytt{double \newfunction{feta6RloVb}(const lomass massin,const double L)}
returns $F_{\eta\,R}^{V(6)0}$.\\

All these are defined in \newfunction{massdecayvevloV.h}
and implemented in \newfunction{massdecayvevloV.h}.
Examples of use are in \newfunction{testmassdecayvevloV.cc}.

\subsubsection{Vacuum-expectation-values at finite volume: in lowest order}
\label{vevsloV}

The finite volume corrections to the vacuum expectation values (vevs)
$\langle0\vert \overline q q \vert 0\rangle$
for up, down and strange quarks in the isospin limit
were  calculated at two-loops in three flavour ChPT
in \cite{Bijnens:2006ve}.
The expression for the up and down quark vev are identical since we
are in the isospin limit.
The finite volume correction is defined as the difference between
the infinite and finite volume value.

We write the expressions in a form analoguous to the decay constant
treatment:
\begin{equation}
\Delta^V\langle0\vert \overline q q \vert 0\rangle
\equiv
\langle0\vert \overline q q \vert 0\rangle_{a\,\mathrm{phys}}^V
-\langle0\vert \overline q q \vert 0\rangle_{a\,\mathrm{phys}}^{V=\infty}
 =
- F_0^2 B_0\left(\langle0\vert \overline q q \vert 0\rangle_a^{V(4)0}+
\langle0\vert \overline q q \vert 0\rangle_a^{V(6)0}\right)\,.
\end{equation}
The superscripts indicate the order of the diagrams in $p$ that each
contribution comes from. The extra $0$ indicates that the expansion
is defined in terms of lowest-order quantities.
The lowest order values are $-F_0^2 B_0$.

Note that the vevs are not directly measurable quantities. They depend
on exactly the way the scalar densities are defined in QCD. ChPT can be
used for them when a massindependent, chiral symmetry respecting subtraction
scheme is used. $\overline{MS}$ in QCD satisfies this, but there are other
possibilities. Even within a scheme, $B_0$ and the quark masses depend
on the QCD subtraction scale $\mu_\textrm{QCD}$ in such a way that
$B_0 m_q$ is independent of it. 
When the scalar density is fully defined, measuring these quantities in
e.g. lattice QCD and comparing with the ChPT expressions is a well defined
procedure.

The contributions at the different orders themselves are split in the parts
depending on the NLO LECs $L_i^r$ and the
remainder as
\begin{eqnarray}
\label{defvevloV}
\langle0\vert \overline q q \vert 0\rangle_a^{(6)0}&=&
\langle0\vert \overline q q \vert 0\rangle_{a\,L}^{(6)0}
+\langle0\vert \overline q q \vert 0\rangle_{a\,R}^{(6)0}\,.
\end{eqnarray}
These are defined for $q=u,s$.\\

The last letter \mytt{x} is \mytt{b} when the finite voilume integrals
are calculated using the Bessel finction method is \mytt{t} when the
theta function method is used. Do not forget to set the precision wanted for
the finite volume integrals.
\mytt{L} is the length of the three spatial directions.

$\Delta^V\langle0\vert \overline q q \vert 0\rangle_{u\,\mathrm{phys}}$:\\
\mytt{double \newfunction{qqup4loVx}(lomass,L)} returns
  $\langle0\vert \overline q q \vert 0\rangle_{u}^{V(4)0}$\\
\mytt{double \newfunction{qqup6loVx}(lomass,Li,L)} returns
  $\langle0\vert \overline q q \vert 0\rangle_{u}^{V(6)0}$\\
\mytt{double \newfunction{qqup6LloVx}(lomass,Li,L)} returns
  $\langle0\vert \overline q q \vert 0\rangle_{u\,L}^{V(6)0}$\\
\mytt{double \newfunction{qqup6RloVx}(lomass,L)} returns
  $\langle0\vert \overline q q \vert 0\rangle_{u\,R}^{V(6)0}$\\

$\Delta^V\langle0\vert \overline q q \vert 0\rangle_{s\,\mathrm{phys}}$:\\
\mytt{double \newfunction{qqstrange4loVx}(lomass,L)} returns
  $\langle0\vert \overline q q \vert 0\rangle_{s}^{V(4)0}$\\
\mytt{double \newfunction{qqstrange6loVx}(lomass,Li,L)} returns
  $\langle0\vert \overline q q \vert 0\rangle_{s}^{V(6)0}$\\
\mytt{double \newfunction{qqstrange6LloVx}(lomass,Li,L)} returns
  $\langle0\vert \overline q q \vert 0\rangle_{s\,L}^{V(6)0}$\\
\mytt{double \newfunction{qqstrange6RloVx}(lomass,L)} returns
  $\langle0\vert \overline q q \vert 0\rangle_{s\,R}^{V(6)0}$\\


The functions are defined in \newfunction{massesdecayvevloV.h},
implemented in \newfunction{massesdecayvevloV.cc} and examples of use
are in \newfunction{testmassesdecayvevloV.cc}

\section{Three flavour partially quenched results}

This section contains the routines used for the partially quenched
results with three sea quark flavours
of \cite{Bijnens:2006jv,Bijnens:2005ae,Bijnens:2004hk}.
The formulas used are analytically equivalent to those in the published
papers, but are longer and avoid some of the $0/0$ problems that can appear.
The finite volume expressions were derived in \cite{Bijnens:2015dra}.

Do not forget to set the precision for the needed sunset integrals
with\\ \mytt{setprecisionquenchedsunsetintegral} and the finite volume
equivalents.

The interface is always defined with the $n_F$ flavour NLO and NNLO LECs
\newfunction{Linf} and \newfunction{Ki} with $n_F=3$. The routines also
expect a \newfunction{quarkmassnf} with precisely the number of quark masses
needed for each case.

The reason why the quarkmasses are alternatively lowest-order meson masses
are used is that in these cases there are very many physical masses
compared to the number of quark masses. There would thus have been a very
large ambiguity in expressing the results in physical masses.

The inputs used are $Bm_1=m_{11}^2/2$, $Bm_1=m_{11}^2/2$
$Bm_3=m_{33}^2/2$, $Bm_4=m_{44}^2/2$, $Bm_5=m_{55}^2/2$,  $Bm_6=m_{66}^2/2$.

We give the cases for equal or different valence quark mass ,cases \mytt{v1}
or \mytt{v2}, and one, two or three different sea quark masses,
cases \mytt{s1}, \mytt{s2}, \mytt{s3} for always three sea flavours, case \mytt{nf3}. For the one sea mass case we have $Bm_4=Bm_5=Bm_6$ and for the two sea
mass case $Bm_4=Bm_5$.

The quark masses 
The masses are labelled starting with \mytt{m} and the decay constants starting with \mytt{f}. \mytt{f0} is $F_0$ the three flavour chiral limit decay constant
and \mytt{mu} is the subtraction scale $\mu$.

For the finite volume cases the names have an additional \mytt{V}
and a \mytt{b} or \mytt{t} dependending on whether the Bessel
function or the theta function method is used for the finite volume
integrals. \mytt{L} is the spatial extent of the finite directions.

\subsection{Masses}

The expansion are defined similar to (\ref{massesinlo}) via
\begin{equation}
\label{massesPQnf3}
m_{a\,\mathrm{phys}}^2 = m_{a\,0}^2+m_a^{2(4)0}+m_a^{2(6)0}\,.
\end{equation}
The masses are for the off-diagonal or charged meson with two different
valence quarks but the masses can be
equal or different.

\begin{equation}
\label{defmassesPQlo}
m_a^{2(4)0}=m_{a\,L}^{2(4)0}+m_{a\,R}^{2(4)0}\,,
\qquad
m_a^{2(6)0}=m_{a\,L}^{2(6)0}+m_{a\,K}^{2(6)0}+m_{a\,R}^{2(6)0}\,.
\end{equation}
\\

One valence mass, one sea mass:\\
\mytt{mass=quarkmassnf(\{Bm1,Bm4\},f0,mu,2)}\\
\mytt{double \newfunction{mv1s1nf3p4}(quarkmassnf mass, Linf Liin)} returns $m_a^{2(4)0}$\\
\mytt{double \newfunction{mv1s1nf3p4L}(quarkmassnf mass, Linf Liin)} returns $m_{a\,L}^{2(4)0}$\\
\mytt{double \newfunction{mv1s1nf3p4R}(quarkmassnf mass)} returns $m_{a\,R}^{2(4)0}$\\
\mytt{double \newfunction{mv1s1nf3p6}(quarkmassnf mass, Linf Liin, Ki Kiin)} returns $m_{a}^{2(6)0}$\\
\mytt{double \newfunction{mv1s1nf3p6L}(quarkmassnf mass, Linf Liin)} returns $m_{a\,L}^{2(6)0}$\\
\mytt{double \newfunction{mv1s1nf3p6K}(quarkmassnf mass, Ki Kiin)} returns $m_{a\,K}^{2(6)0}$\\
\mytt{double \newfunction{mv1s1nf3p6R}(quarkmassnf mass)} returns $m_{a\,R}^{2(6)0}$\\


Two valence mass, one sea mass:\\
\mytt{mass=quarkmassnf(\{Bm1,Bm3,Bm4\},f0,mu,3)}\\
\mytt{double \newfunction{mv2s1nf3p4}(quarkmassnf mass, Linf Liin)} returns $m_a^{2(4)0}$\\
\mytt{double \newfunction{mv2s1nf3p4L}(quarkmassnf mass, Linf Liin)} returns $m_{a\,L}^{2(4)0}$\\
\mytt{double \newfunction{mv2s1nf3p4R}(quarkmassnf mass)} returns $m_{a\,R}^{2(4)0}$\\
\mytt{double \newfunction{mv2s1nf3p6}(quarkmassnf mass, Linf Liin, Ki Kiin)} returns $m_{a}^{2(6)0}$\\
\mytt{double \newfunction{mv2s1nf3p6L}(quarkmassnf mass, Linf Liin)} returns $m_{a\,L}^{2(6)0}$\\
\mytt{double \newfunction{mv2s1nf3p6K}(quarkmassnf mass, Ki Kiin)} returns $m_{a\,K}^{2(6)0}$\\
\mytt{double \newfunction{mv2s1nf3p6R}(quarkmassnf mass)} returns $m_{a\,R}^{2(6)0}$\\

One valence mass, two sea mass:\\
\mytt{mass=quarkmassnf(\{Bm1,Bm4,Bm6\},f0,mu,3)}\\
\mytt{double \newfunction{mv1s1nf3p4}(quarkmassnf mass, Linf Liin)} returns $m_a^{2(4)0}$\\
\mytt{double \newfunction{mv1s1nf3p4L}(quarkmassnf mass, Linf Liin)} returns $m_{a\,L}^{2(4)0}$\\
\mytt{double \newfunction{mv1s1nf3p4R}(quarkmassnf mass)} returns $m_{a\,R}^{2(4)0}$\\
\mytt{double \newfunction{mv1s1nf3p6}(quarkmassnf mass, Linf Liin, Ki Kiin)} returns $m_{a}^{2(6)0}$\\
\mytt{double \newfunction{mv1s1nf3p6L}(quarkmassnf mass, Linf Liin)} returns $m_{a\,L}^{2(6)0}$\\
\mytt{double \newfunction{mv1s1nf3p6K}(quarkmassnf mass, Ki Kiin)} returns $m_{a\,K}^{2(6)0}$\\
\mytt{double \newfunction{mv1s1nf3p6R}(quarkmassnf mass)} returns $m_{a\,R}^{2(6)0}$\\

Two valence mass, two sea mass:\\
\mytt{mass=quarkmassnf(\{Bm1,Bm3,Bm4,Bm6\},f0,mu,4)}\\
\mytt{double \newfunction{mv2s2nf3p4}(quarkmassnf mass, Linf Liin)} returns $m_a^{2(4)0}$\\
\mytt{double \newfunction{mv2s2nf3p4L}(quarkmassnf mass, Linf Liin)} returns $m_{a\,L}^{2(4)0}$\\
\mytt{double \newfunction{mv2s2nf3p4R}(quarkmassnf mass)} returns $m_{a\,R}^{2(4)0}$\\
\mytt{double \newfunction{mv2s2nf3p6}(quarkmassnf mass, Linf Liin, Ki Kiin)} returns $m_{a}^{2(6)0}$\\
\mytt{double \newfunction{mv2s2nf3p6L}(quarkmassnf mass, Linf Liin)} returns $m_{a\,L}^{2(6)0}$\\
\mytt{double \newfunction{mv2s2nf3p6K}(quarkmassnf mass, Ki Kiin)} returns $m_{a\,K}^{2(6)0}$\\
\mytt{double \newfunction{mv2s2nf3p6R}(quarkmassnf mass)} returns $m_{a\,R}^{2(6)0}$\\

One valence mass, three sea mass:\\
\mytt{mass=quarkmassnf(\{Bm1,Bm4,Bm5,Bm6\},f0,mu,4)}\\
\mytt{double \newfunction{mv1s3nf3p4}(quarkmassnf mass, Linf Liin)} returns $m_a^{2(4)0}$\\
\mytt{double \newfunction{mv1s3nf3p4L}(quarkmassnf mass, Linf Liin)} returns $m_{a\,L}^{2(4)0}$\\
\mytt{double \newfunction{mv1s3nf3p4R}(quarkmassnf mass)} returns $m_{a\,R}^{2(4)0}$\\
\mytt{double \newfunction{mv1s3nf3p6}(quarkmassnf mass, Linf Liin, Ki Kiin)} returns $m_{a}^{2(6)0}$\\
\mytt{double \newfunction{mv1s3nf3p6L}(quarkmassnf mass, Linf Liin)} returns $m_{a\,L}^{2(6)0}$\\
\mytt{double \newfunction{mv1s3nf3p6K}(quarkmassnf mass, Ki Kiin)} returns $m_{a\,K}^{2(6)0}$\\
\mytt{double \newfunction{mv1s3nf3p6R}(quarkmassnf mass)} returns $m_{a\,R}^{2(6)0}$\\

Two valence mass, three sea mass:\\
\mytt{mass=quarkmassnf(\{Bm1,Bm3,Bm4,Bm5,Bm6\},f0,mu,5)}\\
\mytt{double \newfunction{mv2s3nf3p4}(quarkmassnf mass, Linf Liin)} returns $m_a^{2(4)0}$\\
\mytt{double \newfunction{mv2s3nf3p4L}(quarkmassnf mass, Linf Liin)} returns $m_{a\,L}^{2(4)0}$\\
\mytt{double \newfunction{mv2s3nf3p4R}(quarkmassnf mass)} returns $m_{a\,R}^{2(4)0}$\\
\mytt{double \newfunction{mv2s3nf3p6}(quarkmassnf mass, Linf Liin, Ki Kiin)} returns $m_{a}^{2(6)0}$\\
\mytt{double \newfunction{mv2s3nf3p6L}(quarkmassnf mass, Linf Liin)} returns $m_{a\,L}^{2(6)0}$\\
\mytt{double \newfunction{mv2s3nf3p6K}(quarkmassnf mass, Ki Kiin)} returns $m_{a\,K}^{2(6)0}$\\
\mytt{double \newfunction{mv2s3nf3p6R}(quarkmassnf mass)} returns $m_{a\,R}^{2(6)0}$\\

Defined in \newfunction{massdecayvevPQ.h}, implemented in \newfunction{massdecayvevPQ.cc} and examples of use in \newfunction{testmassdecayvevPQ.cc}.

\subsection{Decay constants}

The expansion are defined similar to (\ref{defdecaylo}) via
\begin{equation}
\label{decayPQnf3}
F_{a\,\mathrm{phys}} = F_0\left(1+F_a^{(4)0}+F_a^{(6)0}\right)\,.
\end{equation}
The decay constants are for the off-diagonal or charged meson with
two different valence quarks but the masses can be
equal or different. The normalization corresponds to the
pion decay constant $F_\pi\approx92$~MeV.

\begin{equation}
\label{defdecayPQlo}
F_a^{(4)0}=F_{a\,L}^{(4)0}+F_{a\,R}^{(4)0}\,,
\qquad
F_a^{(6)0}=F_{a\,L}^{(6)0}+F_{a\,K}^{(6)0}+F_{a\,R}^{(6)0}\,.
\end{equation}


One valence mass, one sea mass:\\
\mytt{mass=quarkmassnf(\{Bm1,Bm4\},f0,mu,2)}\\
\mytt{double \newfunction{fv1s1nf3p4}(quarkmassnf mass, Linf Liin)} returns $F_a^{(4)0}$\\
\mytt{double \newfunction{fv1s1nf3p4L}(quarkmassnf mass, Linf Liin)} returns $F_{a\,L}^{(4)0}$\\
\mytt{double \newfunction{fv1s1nf3p4R}(quarkmassnf mass)} returns $F_{a\,R}^{(4)0}$\\
\mytt{double \newfunction{fv1s1nf3p6}(quarkmassnf mass, Linf Liin, Ki Kiin)} returns $F_{a}^{(6)0}$\\
\mytt{double \newfunction{fv1s1nf3p6L}(quarkmassnf mass, Linf Liin)} returns $F_{a\,L}^{(6)0}$\\
\mytt{double \newfunction{fv1s1nf3p6K}(quarkmassnf mass, Ki Kiin)} returns $F_{a\,K}^{(6)0}$\\
\mytt{double \newfunction{fv1s1nf3p6R}(quarkmassnf mass)} returns $F_{a\,R}^{(6)0}$\\


Two valence mass, one sea mass:\\
\mytt{mass=quarkmassnf(\{Bm1,Bm3,Bm4\},f0,mu,3)}\\
\mytt{double \newfunction{fv2s1nf3p4}(quarkmassnf mass, Linf Liin)} returns $F_a^{(4)0}$\\
\mytt{double \newfunction{fv2s1nf3p4L}(quarkmassnf mass, Linf Liin)} returns $F_{a\,L}^{(4)0}$\\
\mytt{double \newfunction{fv2s1nf3p4R}(quarkmassnf mass)} returns $F_{a\,R}^{(4)0}$\\
\mytt{double \newfunction{fv2s1nf3p6}(quarkmassnf mass, Linf Liin, Ki Kiin)} returns $F_{a}^{(6)0}$\\
\mytt{double \newfunction{fv2s1nf3p6L}(quarkmassnf mass, Linf Liin)} returns $F_{a\,L}^{(6)0}$\\
\mytt{double \newfunction{fv2s1nf3p6K}(quarkmassnf mass, Ki Kiin)} returns $F_{a\,K}^{(6)0}$\\
\mytt{double \newfunction{fv2s1nf3p6R}(quarkmassnf mass)} returns $F_{a\,R}^{(6)0}$\\

One valence mass, two sea mass:\\
\mytt{mass=quarkmassnf(\{Bm1,Bm4,Bm6\},f0,mu,3)}\\
\mytt{double \newfunction{fv1s1nf3p4}(quarkmassnf mass, Linf Liin)} returns $F_a^{(4)0}$\\
\mytt{double \newfunction{fv1s1nf3p4L}(quarkmassnf mass, Linf Liin)} returns $F_{a\,L}^{(4)0}$\\
\mytt{double \newfunction{fv1s1nf3p4R}(quarkmassnf mass)} returns $F_{a\,R}^{(4)0}$\\
\mytt{double \newfunction{fv1s1nf3p6}(quarkmassnf mass, Linf Liin, Ki Kiin)} returns $F_{a}^{(6)0}$\\
\mytt{double \newfunction{fv1s1nf3p6L}(quarkmassnf mass, Linf Liin)} returns $F_{a\,L}^{(6)0}$\\
\mytt{double \newfunction{fv1s1nf3p6K}(quarkmassnf mass, Ki Kiin)} returns $F_{a\,K}^{(6)0}$\\
\mytt{double \newfunction{fv1s1nf3p6R}(quarkmassnf mass)} returns $F_{a\,R}^{(6)0}$\\

Two valence mass, two sea mass:\\
\mytt{mass=quarkmassnf(\{Bm1,Bm3,Bm4,Bm6\},f0,mu,4)}\\
\mytt{double \newfunction{fv2s2nf3p4}(quarkmassnf mass, Linf Liin)} returns $F_a^{(4)0}$\\
\mytt{double \newfunction{fv2s2nf3p4L}(quarkmassnf mass, Linf Liin)} returns $F_{a\,L}^{(4)0}$\\
\mytt{double \newfunction{fv2s2nf3p4R}(quarkmassnf mass)} returns $F_{a\,R}^{(4)0}$\\
\mytt{double \newfunction{fv2s2nf3p6}(quarkmassnf mass, Linf Liin, Ki Kiin)} returns $F_{a}^{(6)0}$\\
\mytt{double \newfunction{fv2s2nf3p6L}(quarkmassnf mass, Linf Liin)} returns $F_{a\,L}^{(6)0}$\\
\mytt{double \newfunction{fv2s2nf3p6K}(quarkmassnf mass, Ki Kiin)} returns $F_{a\,K}^{(6)0}$\\
\mytt{double \newfunction{fv2s2nf3p6R}(quarkmassnf mass)} returns $F_{a\,R}^{(6)0}$\\

One valence mass, three sea mass:\\
\mytt{mass=quarkmassnf(\{Bm1,Bm4,Bm5,Bm6\},f0,mu,4)}\\
\mytt{double \newfunction{fv1s3nf3p4}(quarkmassnf mass, Linf Liin)} returns $F_a^{(4)0}$\\
\mytt{double \newfunction{fv1s3nf3p4L}(quarkmassnf mass, Linf Liin)} returns $F_{a\,L}^{(4)0}$\\
\mytt{double \newfunction{fv1s3nf3p4R}(quarkmassnf mass)} returns $F_{a\,R}^{(4)0}$\\
\mytt{double \newfunction{fv1s3nf3p6}(quarkmassnf mass, Linf Liin, Ki Kiin)} returns $F_{a}^{(6)0}$\\
\mytt{double \newfunction{fv1s3nf3p6L}(quarkmassnf mass, Linf Liin)} returns $F_{a\,L}^{(6)0}$\\
\mytt{double \newfunction{fv1s3nf3p6K}(quarkmassnf mass, Ki Kiin)} returns $F_{a\,K}^{(6)0}$\\
\mytt{double \newfunction{fv1s3nf3p6R}(quarkmassnf mass)} returns $F_{a\,R}^{(6)0}$\\

Two valence mass, three sea mass:\\
\mytt{mass=quarkmassnf(\{Bm1,Bm3,Bm4,Bm5,Bm6\},f0,mu,5)}\\
\mytt{double \newfunction{fv2s3nf3p4}(quarkmassnf mass, Linf Liin)} returns $F_a^{(4)0}$\\
\mytt{double \newfunction{fv2s3nf3p4L}(quarkmassnf mass, Linf Liin)} returns $F_{a\,L}^{(4)0}$\\
\mytt{double \newfunction{fv2s3nf3p4R}(quarkmassnf mass)} returns $F_{a\,R}^{(4)0}$\\
\mytt{double \newfunction{fv2s3nf3p6}(quarkmassnf mass, Linf Liin, Ki Kiin)} returns $F_{a}^{(6)0}$\\
\mytt{double \newfunction{fv2s3nf3p6L}(quarkmassnf mass, Linf Liin)} returns $F_{a\,L}^{(6)0}$\\
\mytt{double \newfunction{fv2s3nf3p6K}(quarkmassnf mass, Ki Kiin)} returns $F_{a\,K}^{(6)0}$\\
\mytt{double \newfunction{fv2s3nf3p6R}(quarkmassnf mass)} returns $F_{a\,R}^{(6)0}$\\

Defined in \newfunction{massdecayvevPQ.h}, implemented in \newfunction{massdecayvevPQ.cc} and examples of use in \newfunction{testmassdecayvevPQ.cc}.


\subsection{Masses at finite volume}

The expressions treated in this section have been
derived in \cite{Bijnens:2015dra}. It contains the routines for the finite volume
corrections for the masses of the off-diagonal or charged mesons
in partially quenched ChPT with three sea quark flavours.

A general remark is that care should be
taken to set the precision in the loop integrals sufficiently high.
For the one-loop integrals setting it very high is usually no problem.
For the sunset integrals the evaluation can become very slow. It is
strongly recommended to play around with the settings and compare the outputs
for the two ways to evaluate the integral. 
The theta and Bessel function evaluation approach the correct answer
differently.
For most cases
it is possible to have \mytt{rsacc} set smaller than \mytt{racc}.

For many applications it is useful to calculate the
very time consuming parts, those labeled \mytt{6RV}, once and store them.
They only depend nontrivially on the masses and size of the finite volume.
The decay constant dependence is very simple, an overall factor at each order,
and there is no dependence
on the NLO LECs $L_i^r$.

The results presented in this section are with periodic boundary conditions
and an infinite extension in the time direction. They are also restricted
to the case where the particle is at rest, i.e. $\vec p=0$.

The expansion are defined similar to (\ref{defmassloV}) via
\begin{eqnarray}
\label{massesPQVnf3}
\Delta^V m^2_a &=& m^{2V}_a-m^{2\,V=\infty}_a
= m_a^{2V(4)0}+ m_a^{2V(6)0}\,.
\nonumber\\
m_a^{2V(6)0} &=& m_{a\,L}^{2V(6)0}+m_{a\,R}^{2V(6)0}\,.
\end{eqnarray}
The masses are for the off-diagonal or charged meson with two different
valence quarks but the masses can be
equal or different.

\mytt{x} should be \mytt{b} or \mytt{t} depening on whether you want to use
the finite volume integrals using bessel functions or theta functions.
\\

One valence mass, one sea mass:\\
\mytt{mass=quarkmassnf(\{Bm1,Bm4\},f0,mu,2)}\\
\mytt{double \newfunction{mv1s1nf3p4Vx}(quarkmassnf mass, double L)} returns $m_a^{2V(4)0}$\\
\mytt{double \newfunction{mv1s1nf3p6Vx}(quarkmassnf mass, Linf Liin,double L)} returns $m_{a}^{2V(6)0}$\\
\mytt{double \newfunction{mv1s1nf3p6LVx}(quarkmassnf mass, Linf Liin, double L)} returns $m_{a\,L}^{2V(6)0}$\\
\mytt{double \newfunction{mv1s1nf3p6RVx}(quarkmassnf mass, double L)} returns $m_{a\,R}^{2V(6)0}$\\


Two valence mass, one sea mass:\\
\mytt{mass=quarkmassnf(\{Bm1,Bm3,Bm4\},f0,mu,3)}\\
\mytt{double \newfunction{mv2s1nf3p4Vx}(quarkmassnf mass, double L)} returns $m_a^{2V(4)0}$\\
\mytt{double \newfunction{mv2s1nf3p6Vx}(quarkmassnf mass, Linf Liin, double L)} returns $m_{a}^{2V(6)0}$\\
\mytt{double \newfunction{mv2s1nf3p6LVx}(quarkmassnf mass, Linf Liin, double L)} returns $m_{a\,L}^{2V(6)0}$\\
\mytt{double \newfunction{mv2s1nf3p6RVx}(quarkmassnf mass, double L)} returns $m_{a\,R}^{2V(6)0}$\\

One valence mass, two sea mass:\\
\mytt{mass=quarkmassnf(\{Bm1,Bm4,Bm6\},f0,mu,3)}\\
\mytt{double \newfunction{mv1s1nf3p4Vx}(quarkmassnf mass, double L)} returns $m_a^{2V(4)0}$\\
\mytt{double \newfunction{mv1s1nf3p6Vx}(quarkmassnf mass, Linf Liin, double L)} returns $m_{a}^{2V(6)0}$\\
\mytt{double \newfunction{mv1s1nf3p6LVx}(quarkmassnf mass, Linf Liin, double L)} returns $m_{a\,L}^{2V(6)0}$\\
\mytt{double \newfunction{mv1s1nf3p6RVx}(quarkmassnf mass, double L)} returns $m_{a\,R}^{2V(6)0}$\\

Two valence mass, two sea mass:\\
\mytt{mass=quarkmassnf(\{Bm1,Bm3,Bm4,Bm6\},f0,mu,4)}\\
\mytt{double \newfunction{mv2s2nf3p4Vx}(quarkmassnf mass, double L)} returns $m_a^{2V(4)0}$\\
\mytt{double \newfunction{mv2s2nf3p6Vx}(quarkmassnf mass, Linf Liin, double L)} returns $m_{a}^{2V(6)0}$\\
\mytt{double \newfunction{mv2s2nf3p6LVx}(quarkmassnf mass, Linf Liin, double L)} returns $m_{a\,L}^{2V(6)0}$\\
\mytt{double \newfunction{mv2s2nf3p6RVx}(quarkmassnf mass, double L)} returns $m_{a\,R}^{2V(6)0}$\\

One valence mass, three sea mass:\\
\mytt{mass=quarkmassnf(\{Bm1,Bm4,Bm5,Bm6\},f0,mu,4)}\\
\mytt{double \newfunction{mv1s3nf3p4Vx}(quarkmassnf mass, double L)} returns $m_a^{2V(4)0}$\\
\mytt{double \newfunction{mv1s3nf3p6Vx}(quarkmassnf mass, Linf Liin, double L)} returns $m_{a}^{2V(6)0}$\\
\mytt{double \newfunction{mv1s3nf3p6LVx}(quarkmassnf mass, Linf Liin, double L)} returns $m_{a\,L}^{2V(6)0}$\\
\mytt{double \newfunction{mv1s3nf3p6RVx}(quarkmassnf mass, double L)} returns $m_{a\,R}^{2V(6)0}$\\

Two valence mass, three sea mass:\\
\mytt{mass=quarkmassnf(\{Bm1,Bm3,Bm4,Bm5,Bm6\},f0,mu,5)}\\
\mytt{double \newfunction{mv2s3nf3p4Vx}(quarkmassnf mass, double L)} returns $m_a^{2V(4)0}$\\
\mytt{double \newfunction{mv2s3nf3p6Vx}(quarkmassnf mass, Linf Liin, double L)} returns $m_{a}^{2V(6)0}$\\
\mytt{double \newfunction{mv2s3nf3p6LVx}(quarkmassnf mass, Linf Liin, double L)} returns $m_{a\,L}^{2V(6)0}$\\
\mytt{double \newfunction{mv2s3nf3p6RVx}(quarkmassnf mass, double L)} returns $m_{a\,R}^{2V(6)0}$\\

Defined in \newfunction{massdecayvevPQV.h}, implemented in \newfunction{massdecayvevPQV.cc} and examples of use in \newfunction{testmassdecayvevPQV.cc}.

\subsection{Decay constants at finite volume}

The expressions treated in this section have been
derived in \cite{Bijnens:2015dra}. It contains the routines for the finite volume
corrections for the decay constants of the off-diagonal or charged mesons
in partially quenched ChPT with three sea quark flavours.

A general remark is that care should be
taken to set the precision in the loop integrals sufficiently high.
For the one-loop integrals setting it very high is usually no problem.
For the sunset integrals the evaluation can become very slow. It is
strongly recommended to play around with the settings and compare the outputs
for the two ways to evaluate the integral. 
The theta and Bessel function evaluation approach the correct answer
differently.
For most cases
it is possible to have \mytt{rsacc} set smaller than \mytt{racc}.

For many applications it is useful to calculate the
very time consuming parts, those labeled \mytt{6RV}, once and store them.
They only depend nontrivially on the masses and size of the finite volume.
The decay constant dependence is very simple, an overall factor at each order,
and there is no dependence
on the NLO LECs $L_i^r$.

The results presented in this section are with periodic boundary conditions
and an infinite extension in the time direction. They are also restricted
to the case where the particle is at rest, i.e. $\vec p=0$.

The expansion are defined similar to (\ref{defdecayloV}) via
\begin{eqnarray}
\label{decayPQVnf3}
\Delta^V F_a &=& F^{V}_a-F^{\,V=\infty}_a
= F_0\left(F_a^{V(4)0}+ F_a^{V(6)0}\right)\,.
\nonumber\\
F_a^{V(6)0} &=& F_{a\,L}^{V(6)0}+F_{a\,R}^{V(6)0}\,.
\end{eqnarray}
\\
The decay constants are for the off-diagonal or charged meson with two different
valence quarks but the masses can be
equal or different.

\mytt{x} should be \mytt{b} or \mytt{t} depening on whether you want to use
the finite volume integrals using bessel functions or theta functions.
\\

One valence mass, one sea mass:\\
\mytt{mass=quarkmassnf(\{Bm1,Bm4\},f0,mu,2)}\\
\mytt{double \newfunction{fv1s1nf3p4Vx}(quarkmassnf mass, double L)} returns $F_a^{V(4)0}$\\
\mytt{double \newfunction{fv1s1nf3p6Vx}(quarkmassnf mass, Linf Liin, double L)} returns $F_{a}^{V(6)0}$\\
\mytt{double \newfunction{fv1s1nf3p6LVx}(quarkmassnf mass, Linf Liin, double L)} returns $F_{a\,L}^{V(6)0}$\\
\mytt{double \newfunction{fv1s1nf3p6RVx}(quarkmassnf mass, double L)} returns $F_{a\,R}^{V(6)0}$\\


Two valence mass, one sea mass:\\
\mytt{mass=quarkmassnf(\{Bm1,Bm3,Bm4\},f0,mu,3)}\\
\mytt{double \newfunction{fv2s1nf3p4Vx}(quarkmassnf mass, double L)} returns $F_a^{V(4)0}$\\
\mytt{double \newfunction{fv2s1nf3p6Vx}(quarkmassnf mass, Linf Liin, double L)} returns $F_{a}^{V(6)0}$\\
\mytt{double \newfunction{fv2s1nf3p6LVx}(quarkmassnf mass, Linf Liin, double L)} returns $F_{a\,L}^{V(6)0}$\\
\mytt{double \newfunction{fv2s1nf3p6RVx}(quarkmassnf mass, double L)} returns $F_{a\,R}^{V(6)0}$\\

One valence mass, two sea mass:\\
\mytt{mass=quarkmassnf(\{Bm1,Bm4,Bm6\},f0,mu,3)}\\
\mytt{double \newfunction{fv1s1nf3p4Vx}(quarkmassnf mass, double L)} returns $F_a^{V(4)0}$\\
\mytt{double \newfunction{fv1s1nf3p6Vx}(quarkmassnf mass, Linf Liin, double L)} returns $F_{a}^{V(6)0}$\\
\mytt{double \newfunction{fv1s1nf3p6LVx}(quarkmassnf mass, Linf Liin, double L)} returns $F_{a\,L}^{V(6)0}$\\
\mytt{double \newfunction{fv1s1nf3p6RVx}(quarkmassnf mass, double L)} returns $F_{a\,R}^{V(6)0}$\\

Two valence mass, two sea mass:\\
\mytt{mass=quarkmassnf(\{Bm1,Bm3,Bm4,Bm6\},f0,mu,4)}\\
\mytt{double \newfunction{fv2s2nf3p4}(quarkmassnf mass, double L)} returns $F_a^{V(4)0}$\\
\mytt{double \newfunction{fv2s2nf3p6}(quarkmassnf mass, Linf Liin, double L)} returns $F_{a}^{V(6)0}$\\
\mytt{double \newfunction{fv2s2nf3p6L}(quarkmassnf mass, Linf Liin, double L)} returns $F_{a\,L}^{V(6)0}$\\
\mytt{double \newfunction{fv2s2nf3p6R}(quarkmassnf mass, double L)} returns $F_{a\,R}^{V(6)0}$\\

One valence mass, three sea mass:\\
\mytt{mass=quarkmassnf(\{Bm1,Bm4,Bm5,Bm6\},f0,mu,4)}\\
\mytt{double \newfunction{fv1s3nf3p4Vx}(quarkmassnf mass, double L)} returns $F_a^{V(4)0}$\\
\mytt{double \newfunction{fv1s3nf3p6Vx}(quarkmassnf mass, Linf Liin, double L)} returns $F_{a}^{V(6)0}$\\
\mytt{double \newfunction{fv1s3nf3p6LVx}(quarkmassnf mass, Linf Liin, double L)} returns $F_{a\,L}^{V(6)0}$\\
\mytt{double \newfunction{fv1s3nf3p6RVx}(quarkmassnf mass, double L)} returns $F_{a\,R}^{V(6)0}$\\

Two valence mass, three sea mass:\\
\mytt{mass=quarkmassnf(\{Bm1,Bm3,Bm4,Bm5,Bm6\},f0,mu,5)}\\
\mytt{double \newfunction{fv2s3nf3p4Vx}(quarkmassnf mass, double L)} returns $F_a^{V(4)0}$\\
\mytt{double \newfunction{fv2s3nf3p6Vx}(quarkmassnf mass, Linf Liin, double L)} returns $F_{a}^{V(6)0}$\\
\mytt{double \newfunction{fv2s3nf3p6LVx}(quarkmassnf mass, Linf Liin, double L)} returns $F_{a\,L}^{V(6)0}$\\
\mytt{double \newfunction{fv2s3nf3p6RVx}(quarkmassnf mass, double L)} returns $F_{a\,R}^{V(6)0}$\\

Defined in \newfunction{massdecayvevPQV.h}, implemented in \newfunction{massdecayvevPQV.cc} and examples of use in \newfunction{testmassdecayvevPQV.cc}.

\section{QCD like theories for $N_F$ flavours}

There are other symmetry breaking patterns possible then the one
used in two and three-flavour ChPT. With $N_F$ Dirac fermions in a complex,
real or pseudoreal representation the global symmetries are
$SU(N_F)\times SU(N_F)$, $SU(2N_F)$ and $SU(2N_F)$. This is
described in \cite{Bijnens:2009qm} and references therein. The symmetry
breaking pattern in these case is down to the subgroups
the diagonal $SU(N_F)$, $SO(2N_F)$ and $Sp(2N_F)$ respectyively.
An additional case is $n_F$ Majorana fermions in a real representation. 
In this case the global symmetry group is $SU(n_F)$ which is expected to be
spontaneously broken to $SO(n_F)$, see \cite{QCDliketemp} and references
therein. The formulas for the two cases with real fermions
are identical with $n_F=2N_F$, the reason is that the two cases are
related by a $U(2N_F)$ rotation as explained in \cite{QCDliketemp}.

The number of flavours $N_F$ in this section refers to the symmetry breaking
patterns $SU(N_F)\times SU(N_F)\to SU(N_F)$, $SU(N_F)\to SO(N_F)$
and $SU(2N_F)\to Sp(2N_F)$. The number of flavours for the real representation
case thus counts the number of Majorana fermions and is in the case twice
the $N_F$ used in \cite{Bijnens:2009qm}. The extension to finite volume
and partially quenched was done in \cite{QCDliketemp}.

For all cases we only treat the simplest mass case.
It means that for the unquenched case we have a single mass given by
$B_0 m_1$ or the lowest order meson mass is $m_{LO}^2= 2 B_0 m_1$.
For the partially quenched case the sea quark mass is given by
$B_0 m_4$ or the lowest order mass for sea quark mesons
is $m_{LO}^2=2 B_0 m_4$. 

The standard \textsc{C++} conversions allow the
routines to be called using a \mytt{lomassnf} instead of a \mytt{quarkmassnf}.


\subsection{Mass, decay constant and vacuum-expectation-value: in lowest order}

The functions in this section return the corrections
to the mass-squared in terms of the quarkmassnf structure. It is assumed that
all the quarks have the same mass.
The label \mytt{XXX=SUN,SON,SPN} refers to the three possible patterns
of symmetry breaking, $SU(N_F)\times SU(N_F)\to SU(N_F)$, $SU(N_F)\to SO(N_F)$
and $SU(2N_F)\to Sp(2N_F)$ and \mytt{nf}=$N_F$ as defined in this way.

The mass is defined as in \cite{Bijnens:2009qm} as
\begin{align}
M^2_\mathrm{phys} =& M^2+M^{(4)2}+M^{(6)2}\,,
\nonumber\\
M^{(4)2} =& M^{(4)2}_L+M^{(4)2}_R\,,
\nonumber\\
M^{(6)2} =& M^{(6)2}_K+ M^{(6)2}_L+M^{(6)2}_R\,,
\end{align}
The expressions are defined in terms of lowest order masses and decay
constant.

\mytt{mass=quarkmassnf(\{Bm1\},f0,mu,1)}\\
\mytt{double \newfunction{mnfXXXp4}(int nf, quarkmassnf mass, Liinf Liin)}
 returns $M^{(4)2}$\\
\mytt{double \newfunction{mnfXXXp4L}(int nf, quarkmassnf mass, Liinf Liin)}
 returns $M^{(4)2}_L$\\
\mytt{double \newfunction{mnfXXXp4R}(int nf, quarkmassnf mass)}
 returns $M^{(4)2}_R$\\
\mytt{double \newfunction{mnfXXX6}(int nf, quarkmassnf mass, Liinf Liin, Ki Kiin)}
 returns $M^{(6)2}$\\
\mytt{double \newfunction{mnfXXXp6K}(int nf, quarkmassnf mass, Ki Kiin)}
 returns $M^{(6)2}_K$\\
\mytt{double \newfunction{mnfXXXp6L}(int nf, quarkmassnf mass, Liinf Liin)}
 returns $M^{(6)2}_L$\\
\mytt{double \newfunction{mnfXXXp6R}(int nf, quarkmassnf mass)}
 returns $M^{(6)2}_R$\\

The functions \mytt{mnfXXXp6K} for \mytt{XXX=SON,SPN} simply return 0.
The Lagrangians for these cases have not been classified at NNLO or order $p^6$.
The interface used here is also with the $n_F$-flavour ChPT data structures.
This is correct for \mytt{XXX=SUN} and the results are
independent of the subtraction scale as is needed. However, the extra
constant $L_{11}^r$ is always included even if it plays no role.
In addition the running of the LECs is not correct for \mytt{XXX=SON,SPN}.\\

The decay constant is defined as in \cite{Bijnens:2009qm} as
\begin{align}
F_\mathrm{phys} =& F_0\left(1+F^{(4)2}+F^{(6)2}\right)\,,
\nonumber\\
F^{(4)} =& F^{(4)}_L+F^{(4)}_R\,,
\nonumber\\
F^{(6)} =& F^{(6)}_K+ F^{(6)}_L+M^{(6)}_R\,.
\end{align}
The expressions are defined in terms of lowest order masses and decay
constant.

\mytt{mass=quarkmassnf(\{Bm1\},f0,mu,1)}\\
\mytt{double \newfunction{fnfXXXp4}(int nf, quarkmassnf mass, Liinf Liin)}
 returns $F^{(4)}$\\
\mytt{double \newfunction{fnfXXXp4L}(int nf, quarkmassnf mass, Liinf Liin)}
 returns $F^{(4)}_L$\\
\mytt{double \newfunction{fnfXXXp4R}(int nf, quarkmassnf mass)}
 returns $F^{(4)}_R$\\
\mytt{double \newfunction{fnfXXX6}(int nf, quarkmassnf mass, Liinf Liin, Ki Kiin)}
 returns $F^{(6)}$\\
\mytt{double \newfunction{fnfXXXp6K}(int nf, quarkmassnf mass, Ki Kiin)}
 returns $F^{(6)}_K$\\
\mytt{double \newfunction{fnfXXXp6L}(int nf, quarkmassnf mass, Liinf Liin)}
 returns $F^{(6)}_L$\\
\mytt{double \newfunction{fnfXXXp6R}(int nf, quarkmassnf mass)}
 returns $F^{(6)}_R$\\

The functions \mytt{fnfXXXp6K} for \mytt{XXX=SON,SPN} simply return 0.
The Lagrangians for these cases have not been classified at NNLO or order $p^6$.
The interface used here is also with the $n_F$-flavour ChPT data structures.
This is correct for \mytt{XXX=SUN} and the results are
independent of the subtraction scale as is needed. However, the extra
constant $L_{11}^r$ is always included even if it plays no role.
In addition the running of the LECs is not correct for \mytt{XXX=SON,SPN}.\\

The vacuum expectation value is defined for a single quark
similar to \cite{Bijnens:2009qm} as
\begin{align}
\langle \bar q q\rangle_\mathrm{phys} =& 
-B_0F_0^2\left(1+\langle \bar q q\rangle^{(4)}+\langle \bar q q\rangle^{(6)}\right)\,,
\nonumber\\
\langle \bar q q\rangle^{(4)} =& \langle \bar q q\rangle^{(4)}_L+\langle \bar q q\rangle^{(4)}_R\,,
\nonumber\\
\langle \bar q q\rangle^{(6)} =& \langle \bar q q\rangle^{(6)}_K+
 \langle \bar q q\rangle^{(6)}_L+\langle \bar q q\rangle^{(6)}_R\,.
\end{align}
The expressions are defined in terms of lowest order masses and decay
constant.

\mytt{mass=quarkmassnf(\{Bm1\},f0,mu,1)}\\
\mytt{double \newfunction{qnfXXXp4}(int nf, quarkmassnf mass, Liinf Liin)}
 returns $\langle \bar q q\rangle^{(4)}$\\
\mytt{double \newfunction{qnfXXXp4L}(int nf, quarkmassnf mass, Liinf Liin)}
 returns $\langle \bar q q\rangle^{(4)}_L$\\
\mytt{double \newfunction{qnfXXXp4R}(int nf, quarkmassnf mass)}
 returns $\langle \bar q q\rangle^{(4)}_R$\\
\mytt{double \newfunction{qnfXXX6}(int nf, quarkmassnf mass, Liinf Liin, Ki Kiin)}
 returns $\langle \bar q q\rangle^{(6)}$\\
\mytt{double \newfunction{qnfXXXp6K}(int nf, quarkmassnf mass, Ki Kiin)}
 returns $\langle \bar q q\rangle^{(6)}_K$\\
\mytt{double \newfunction{qnfXXXp6L}(int nf, quarkmassnf mass, Liinf Liin)}
 returns $\langle \bar q q\rangle^{(6)}_L$\\
\mytt{double \newfunction{qnfXXXp6R}(int nf, quarkmassnf mass)}
 returns $\langle \bar q q\rangle^{(6)}_R$\\

The functions \mytt{qnfXXXp6K} for \mytt{XXX=SON,SPN} simply return 0.
The Lagrangians for these cases have not been classified at NNLO or order $p^6$.
The interface used here is also with the $n_F$-flavour ChPT data structures.
This is correct for \mytt{XXX=SUN} and the results are
independent of the subtraction scale as is needed. However, the extra
constant $L_{11}^r$ is always included even if it plays no role.
In addition the running of the LECs is not correct for \mytt{XXX=SON,SPN}.\\


Defined in \newfunction{massdecayvevnf.h}, implemented in \newfunction{massdecayvevnf.cc} and examples of use in \newfunction{testmassdecayvevnf.cc}.

\subsection{Mass, decay constant and vacuum-expectation-value at finite volume: in lowest order}

The functions in this section return the finite volume corrections
to the mass-squared in terms of the quarkmassnf structure. It is assumed that
all the quarks have the same mass.
The label \mytt{XXX=SUN,SON,SPN} refers to the three possible patterns
of symmetry breaking, $SU(N_F)\times SU(N_F)\to SU(N_F)$, $SU(N_F)\to SO(N_F)$
and $SU(2N_F)\to Sp(2N_F)$ and \mytt{nf}=$N_F$ as defined in this way.

The last letter \mytt{x} is \mytt{b} for the finite volume integrals
evaluated using the Bessel function method or \mytt{t} when they are evaluated
using the theta function method.\\

The finite volume correction to the mass is defined as in \cite{QCDliketemp} as
\begin{align}
\Delta^V M^2_\mathrm{phys} \equiv&M^{2\,V}_\mathrm{phys} -M_\mathrm{phys}^{2\,V=\infty} = M^{V(4)2}+M^{V(6)2}\,,
\nonumber\\
M^{V(6)2} =& M^{V(6)2}_L+M^{V(6)2}_R\,,
\end{align}
The expressions are defined in terms of lowest order masses and decay
constant. \mytt{L} is the size of the spatial directions.

\mytt{mass=quarkmassnf(\{Bm1\},f0,mu,1)}\\
\mytt{double \newfunction{mnfXXXp4Vx}(int nf, quarkmassnf mass, double L)}
 returns $M^{V(4)2}$\\
\mytt{double \newfunction{mnfXXX6Vx}(int nf, quarkmassnf mass, Liinf Liin, double L)}
 returns $M^{V(6)2}$\\
\mytt{double \newfunction{mnfXXXp6LVx}(int nf, quarkmassnf mass, Liinf Liin, double L)}
 returns $M^{V(6)2}_L$\\
\mytt{double \newfunction{mnfXXXp6RVx}(int nf, quarkmassnf mass, double L)}
 returns $M^{V(6)2}_R$\\

The interface used here is with the $n_F$-flavour ChPT data structures.
This is correct for \mytt{XXX=SUN} and the results are
independent of the subtraction scale as is needed. However, the extra
constant $L_{11}^r$ is always included even if it plays no role.
In addition the running of the LECs is not correct for \mytt{XXX=SON,SPN}.\\

The finite volume correction to the decay constant is defined as in
\cite{QCDliketemp} as
\begin{align}
\Delta^V F_\mathrm{phys} \equiv&
F_\mathrm{phys}^V -F_\mathrm{phys}^{V=\infty}= F_0\left(F^{V(4)2}+F^{V(6)2}\right)\,,
\nonumber\\
F^{V(6)} =& F^{V(6)}_L+M^{V(6)}_R\,.
\end{align}
The expressions are defined in terms of lowest order masses and decay
constant.

\mytt{mass=quarkmassnf(\{Bm1\},f0,mu,1)}\\
\mytt{double \newfunction{fnfXXXp4Vx}(int nf, quarkmassnf mass, double L)}
 returns $F^{V(4)}$\\
\mytt{double \newfunction{fnfXXX6Vx}(int nf, quarkmassnf mass, Liinf Liin, double L)}
 returns $F^{V(6)}$\\
\mytt{double \newfunction{fnfXXXp6LVx}(int nf, quarkmassnf mass, Liinf Liin, double L)}
 returns $F^{V(6)}_L$\\
\mytt{double \newfunction{fnfXXXp6RVx}(int nf, quarkmassnf mass, double L)}
 returns $F^{V(6)}_R$\\

The interface used here is with the $n_F$-flavour ChPT data structures.
This is correct for \mytt{XXX=SUN} and the results are
independent of the subtraction scale as is needed. However, the extra
constant $L_{11}^r$ is always included even if it plays no role.
In addition the running of the LECs is not correct for \mytt{XXX=SON,SPN}.\\

The vacuum expectation value is defined for a single quark
similar to \cite{QCDliketemp} as
\begin{align}
\Delta^V\langle \bar q q\rangle_\mathrm{phys}\equiv&
\langle \bar q q\rangle_\mathrm{phys}^V -
\langle \bar q q\rangle_\mathrm{phys}^{V=\infty} =
-B_0F_0^2\left(\langle \bar q q\rangle^{V(4)}+\langle \bar q q\rangle^{V(6)}\right)\,,
\nonumber\\
\langle \bar q q\rangle^{(6)} =& 
 \langle \bar q q\rangle^{(6)}_L+\langle \bar q q\rangle^{(6)}_R\,.
\end{align}
The expressions are defined in terms of lowest order masses and decay
constant.

\mytt{mass=quarkmassnf(\{Bm1\},f0,mu,1)}\\
\mytt{double \newfunction{qnfXXXp4Vx}(int nf, quarkmassnf mass, double L)}
 returns $\langle \bar q q\rangle^{V(4)}$\\
\mytt{double \newfunction{qnfXXX6Vx}(int nf, quarkmassnf mass, Liinf Liin, Ki Kiin)}
 returns $\langle \bar q q\rangle^{V(6)}$\\
\mytt{double \newfunction{qnfXXXp6LVx}(int nf, quarkmassnf mass, Liinf Liin)}
 returns $\langle \bar q q\rangle^{V(6)}_L$\\
\mytt{double \newfunction{qnfXXXp6RVx}(int nf, quarkmassnf mass)}
 returns $\langle \bar q q\rangle^{V(6)}_R$\\

The interface used here is with the $n_F$-flavour ChPT data structures.
This is correct for \mytt{XXX=SUN} and the results are
independent of the subtraction scale as is needed. However, the extra
constant $L_{11}^r$ is always included even if it plays no role.
In addition the running of the LECs is not correct for \mytt{XXX=SON,SPN}.\\

Defined in \newfunction{massdecayvevnfV.h}, implemented in \newfunction{massdecayvevnfV.cc} and examples of use in \newfunction{testmassdecayvevnf.cc}.


\subsection{Partially quenched mass, decay constant and vacuum-expectation-value: in lowest order}

The functions in this section return the corrections
to the mass-squared in terms of the quarkmassnf structure. It is assumed that
all the valence quarks have the same mass and all the sea quarks have the same
mass but different from the valence quarks.

The label \mytt{XXX=SUN,SON,SPN} refers to the three possible patterns
of symmetry breaking, $SU(N_F)\times SU(N_F)\to SU(N_F)$, $SU(N_F)\to SO(N_F)$
and $SU(2N_F)\to Sp(2N_F)$ and \mytt{nf}=$N_F$ as defined in this way.

The mass is defined as in \cite{QCDliketemp} as
\begin{align}
M^2_\mathrm{phys} =& M^2+M^{(4)2}+M^{(6)2}\,,
\nonumber\\
M^{(4)2} =& M^{(4)2}_L+M^{(4)2}_R\,,
\nonumber\\
M^{(6)2} =& M^{(6)2}_K+ M^{(6)2}_L+M^{(6)2}_R\,,
\end{align}
The expressions are defined in terms of lowest order masses and decay
constant.

\mytt{mass=quarkmassnf(\{Bm1,Bm4\},f0,mu,2)}\\
\mytt{double \newfunction{mnfXXXp4PQ}(int nf, quarkmassnf mass, Liinf Liin)}
 returns $M^{(4)2}$\\
\mytt{double \newfunction{mnfXXXp4LPQ}(int nf, quarkmassnf mass, Liinf Liin)}
 returns $M^{(4)2}_L$\\
\mytt{double \newfunction{mnfXXXp4RPQ}(int nf, quarkmassnf mass)}
 returns $M^{(4)2}_R$\\
\mytt{double \newfunction{mnfXXX6PQ}(int nf, quarkmassnf mass, Liinf Liin, Ki Kiin)}
 returns $M^{(6)2}$\\
\mytt{double \newfunction{mnfXXXp6KPQ}(int nf, quarkmassnf mass, Ki Kiin)}
 returns $M^{(6)2}_K$\\
\mytt{double \newfunction{mnfXXXp6LPQ}(int nf, quarkmassnf mass, Liinf Liin)}
 returns $M^{(6)2}_L$\\
\mytt{double \newfunction{mnfXXXp6RPQ}(int nf, quarkmassnf mass)}
 returns $M^{(6)2}_R$\\

The functions \mytt{mnfXXXp6KPQ} for \mytt{XXX=SON,SPN} simply return 0.
The Lagrangians for these cases have not been classified at NNLO or order $p^6$.
The interface used here is also with the $n_F$-flavour ChPT data structures.
This is correct for \mytt{XXX=SUN} and the results are
independent of the subtraction scale as is needed. However, the extra
constant $L_{11}^r$ is always included even if it plays no role.
In addition the running of the LECs is not correct for \mytt{XXX=SON,SPN}.\\

The decay constant is defined as in \cite{QCDliketemp} as
\begin{align}
F_\mathrm{phys} =& F_0\left(1+F^{(4)2}+F^{(6)2}\right)\,,
\nonumber\\
F^{(4)} =& F^{(4)}_L+F^{(4)}_R\,,
\nonumber\\
F^{(6)} =& F^{(6)}_K+ F^{(6)}_L+M^{(6)}_R\,.
\end{align}
The expressions are defined in terms of lowest order masses and decay
constant.

\mytt{mass=quarkmassnf(\{Bm1,Bm4\},f0,mu,2)}\\
\mytt{double \newfunction{fnfXXXp4PQ}(int nf, quarkmassnf mass, Liinf Liin)}
 returns $F^{(4)}$\\
\mytt{double \newfunction{fnfXXXp4LPQ}(int nf, quarkmassnf mass, Liinf Liin)}
 returns $F^{(4)}_L$\\
\mytt{double \newfunction{fnfXXXp4RPQ}(int nf, quarkmassnf mass)}
 returns $F^{(4)}_R$\\
\mytt{double \newfunction{fnfXXX6PQ}(int nf, quarkmassnf mass, Liinf Liin, Ki Kiin)}
 returns $F^{(6)}$\\
\mytt{double \newfunction{fnfXXXp6KPQ}(int nf, quarkmassnf mass, Ki Kiin)}
 returns $F^{(6)}_K$\\
\mytt{double \newfunction{fnfXXXp6LPQ}(int nf, quarkmassnf mass, Liinf Liin)}
 returns $F^{(6)}_L$\\
\mytt{double \newfunction{fnfXXXp6RPQ}(int nf, quarkmassnf mass)}
 returns $F^{(6)}_R$\\

The functions \mytt{fnfXXXp6KPQ} for \mytt{XXX=SON,SPN} simply return 0.
The Lagrangians for these cases have not been classified at NNLO or order $p^6$.
The interface used here is also with the $n_F$-flavour ChPT data structures.
This is correct for \mytt{XXX=SUN} and the results are
independent of the subtraction scale as is needed. However, the extra
constant $L_{11}^r$ is always included even if it plays no role.
In addition the running of the LECs is not correct for \mytt{XXX=SON,SPN}.\\

The vacuum expectation value is defined for a single quark
similar to \cite{QCDliketemp} as
\begin{align}
\langle \bar q q\rangle_\mathrm{phys} =& 
-B_0F_0^2\left(1+\langle \bar q q\rangle^{(4)}+\langle \bar q q\rangle^{(6)}\right)\,,
\nonumber\\
\langle \bar q q\rangle^{(4)} =& \langle \bar q q\rangle^{(4)}_L+\langle \bar q q\rangle^{(4)}_R\,,
\nonumber\\
\langle \bar q q\rangle^{(6)} =& \langle \bar q q\rangle^{(6)}_K+
 \langle \bar q q\rangle^{(6)}_L+\langle \bar q q\rangle^{(6)}_R\,.
\end{align}
The expressions are defined in terms of lowest order masses and decay
constant.

\mytt{mass=quarkmassnf(\{Bm1,Bm4\},f0,mu,2)}\\
\mytt{double \newfunction{qnfXXXp4PQ}(int nf, quarkmassnf mass, Liinf Liin)}
 returns $\langle \bar q q\rangle^{(4)}$\\
\mytt{double \newfunction{qnfXXXp4LPQ}(int nf, quarkmassnf mass, Liinf Liin)}
 returns $\langle \bar q q\rangle^{(4)}_L$\\
\mytt{double \newfunction{qnfXXXp4RPQ}(int nf, quarkmassnf mass)}
 returns $\langle \bar q q\rangle^{(4)}_R$\\
\mytt{double \newfunction{qnfXXX6PQ}(int nf, quarkmassnf mass, Liinf Liin, Ki Kiin)}
 returns $\langle \bar q q\rangle^{(6)}$\\
\mytt{double \newfunction{qnfXXXp6KPQ}(int nf, quarkmassnf mass, Ki Kiin)}
 returns $\langle \bar q q\rangle^{(6)}_K$\\
\mytt{double \newfunction{qnfXXXp6LPQ}(int nf, quarkmassnf mass, Liinf Liin)}
 returns $\langle \bar q q\rangle^{(6)}_L$\\
\mytt{double \newfunction{qnfXXXp6RPQ}(int nf, quarkmassnf mass)}
 returns $\langle \bar q q\rangle^{(6)}_R$\\

The functions \mytt{qnfXXXp6KPQ} for \mytt{XXX=SON,SPN} simply return 0.
The Lagrangians for these cases have not been classified at NNLO or order $p^6$.
The interface used here is also with the $n_F$-flavour ChPT data structures.
This is correct for \mytt{XXX=SUN} and the results are
independent of the subtraction scale as is needed. However, the extra
constant $L_{11}^r$ is always included even if it plays no role.
In addition the running of the LECs is not correct for \mytt{XXX=SON,SPN}.\\


Defined in \newfunction{massdecayvevnfPQ.h}, implemented in \newfunction{massdecayvevnfPQ.cc} and examples of use in \newfunction{testmassdecayvevnf.cc}.

\subsection{Partially quenched mass, decay constant and vacuum-expectation-value at finite volume: in lowest order}

The functions in this section return the finite volume corrections
to the mass-squared in terms of the quarkmassnf structure for the partially
quenched case. It is assumed that
all the quarks have the same mass.
The label \mytt{XXX=SUN,SON,SPN} refers to the three possible patterns
of symmetry breaking, $SU(N_F)\times SU(N_F)\to SU(N_F)$, $SU(N_F)\to SO(N_F)$
and $SU(2N_F)\to Sp(2N_F)$ and \mytt{nf}=$N_F$ as defined in this way.

The last letter \mytt{x} is \mytt{b} for the finite volume integrals
evaluated using the Bessel function method or \mytt{t} when they are evaluated
using the theta function method.\\

The finite volume correction to the mass is defined as in \cite{QCDliketemp} as
\begin{align}
\Delta^V M^2_\mathrm{phys} \equiv&M^{2\,V}_\mathrm{phys} -M_\mathrm{phys}^{2\,V=\infty} = M^{V(4)2}+M^{V(6)2}\,,
\nonumber\\
M^{V(6)2} =& M^{V(6)2}_L+M^{V(6)2}_R\,,
\end{align}
The expressions are defined in terms of lowest order masses and decay
constant. \mytt{L} is the size of the spatial directions.

\mytt{mass=quarkmassnf(\{Bm1,Bm4\},f0,mu,2)}\\
\mytt{double \newfunction{mnfXXXp4PQVx}(int nf, quarkmassnf mass, double L)}
 returns $M^{V(4)2}$\\
\mytt{double \newfunction{mnfXXX6PQVx}(int nf, quarkmassnf mass, Liinf Liin, double L)}
 returns $M^{V(6)2}$\\
\mytt{double \newfunction{mnfXXXp6LPQVx}(int nf, quarkmassnf mass, Liinf Liin, double L)}
 returns $M^{V(6)2}_L$\\
\mytt{double \newfunction{mnfXXXp6RPQVx}(int nf, quarkmassnf mass, double L)}
 returns $M^{V(6)2}_R$\\

The interface used here is with the $n_F$-flavour ChPT data structures.
This is correct for \mytt{XXX=SUN} and the results are
independent of the subtraction scale as is needed. However, the extra
constant $L_{11}^r$ is always included even if it plays no role.
In addition the running of the LECs is not correct for \mytt{XXX=SON,SPN}.\\

The finite volume correction to the decay constant is defined as in
\cite{QCDliketemp} as
\begin{align}
\Delta^V F_\mathrm{phys} \equiv&
F_\mathrm{phys}^V -F_\mathrm{phys}^{V=\infty}= F_0\left(F^{V(4)2}+F^{V(6)2}\right)\,,
\nonumber\\
F^{V(6)} =& F^{V(6)}_L+M^{V(6)}_R\,.
\end{align}
The expressions are defined in terms of lowest order masses and decay
constant.

\mytt{mass=quarkmassnf(\{Bm1,Bm4\},f0,mu,2)}\\
\mytt{double \newfunction{fnfXXXp4PQVx}(int nf, quarkmassnf mass, double L)}
 returns $F^{V(4)}$\\
\mytt{double \newfunction{fnfXXX6PQVx}(int nf, quarkmassnf mass, Liinf Liin, double L)}
 returns $F^{V(6)}$\\
\mytt{double \newfunction{fnfXXXp6LPQVx}(int nf, quarkmassnf mass, Liinf Liin, double L)}
 returns $F^{V(6)}_L$\\
\mytt{double \newfunction{fnfXXXp6RPQVx}(int nf, quarkmassnf mass, double L)}
 returns $F^{V(6)}_R$\\

The interface used here is with the $n_F$-flavour ChPT data structures.
This is correct for \mytt{XXX=SUN} and the results are
independent of the subtraction scale as is needed. However, the extra
constant $L_{11}^r$ is always included even if it plays no role.
In addition the running of the LECs is not correct for \mytt{XXX=SON,SPN}.\\

The vacuum expectation value is defined for a single quark
similar to \cite{QCDliketemp} as
\begin{align}
\Delta^V\langle \bar q q\rangle_\mathrm{phys}\equiv&
\langle \bar q q\rangle_\mathrm{phys}^V -
\langle \bar q q\rangle_\mathrm{phys}^{V=\infty} =
-B_0F_0^2\left(\langle \bar q q\rangle^{V(4)}+\langle \bar q q\rangle^{V(6)}\right)\,,
\nonumber\\
\langle \bar q q\rangle^{(6)} =& 
 \langle \bar q q\rangle^{(6)}_L+\langle \bar q q\rangle^{(6)}_R\,.
\end{align}
The expressions are defined in terms of lowest order masses and decay
constant.

\mytt{mass=quarkmassnf(\{Bm1,Bm4\},f0,mu,2)}\\
\mytt{double \newfunction{qnfXXXp4PQVx}(int nf, quarkmassnf mass, double L)}
 returns $\langle \bar q q\rangle^{V(4)}$\\
\mytt{double \newfunction{qnfXXX6PQVx}(int nf, quarkmassnf mass, Liinf Liin, Ki Kiin)}
 returns $\langle \bar q q\rangle^{V(6)}$\\
\mytt{double \newfunction{qnfXXXp6LPQVx}(int nf, quarkmassnf mass, Liinf Liin)}
 returns $\langle \bar q q\rangle^{V(6)}_L$\\
\mytt{double \newfunction{qnfXXXp6RPQVx}(int nf, quarkmassnf mass)}
 returns $\langle \bar q q\rangle^{V(6)}_R$\\

The interface used here is with the $n_F$-flavour ChPT data structures.
This is correct for \mytt{XXX=SUN} and the results are
independent of the subtraction scale as is needed. However, the extra
constant $L_{11}^r$ is always included even if it plays no role.
In addition the running of the LECs is not correct for \mytt{XXX=SON,SPN}.\\

Defined in \newfunction{massdecayvevnfPQV.h}, implemented in \newfunction{massdecayvevnfPQV.cc} and examples of use in \newfunction{testmassdecayvevnf.cc}.


\section*{Acknowledgements}

This work is supported, in part, by
the Swedish Research Council grants 621-2011-5080 and 621-2013-4287.
I thank all my collaborators in the various applications for which my version
of the program made it into this collection and especially
Ilaria Jemos who has tested many of the earlier versions
in the course of \cite{Bijnens:2011tb}.

\appendix
\section{GNU GENERAL PUBLIC LICENSE}
\label{appGPL}
{Version 2, June 1991}


\begin{center}
{\parindent 0in

Copyright \copyright\ 1989, 1991 Free Software Foundation, Inc.

\bigskip

51 Franklin Street, Fifth Floor, Boston, MA  02110-1301, USA

\bigskip

Everyone is permitted to copy and distribute verbatim copies
of this license document, but changing it is not allowed.
}
\end{center}

\begin{center}
{\bf\large Preamble}
\end{center}


The licenses for most software are designed to take away your freedom to
share and change it.  By contrast, the GNU General Public License is
intended to guarantee your freedom to share and change free software---to
make sure the software is free for all its users.  This General Public
License applies to most of the Free Software Foundation's software and to
any other program whose authors commit to using it.  (Some other Free
Software Foundation software is covered by the GNU Library General Public
License instead.)  You can apply it to your programs, too.

When we speak of free software, we are referring to freedom, not price.
Our General Public Licenses are designed to make sure that you have the
freedom to distribute copies of free software (and charge for this service
if you wish), that you receive source code or can get it if you want it,
that you can change the software or use pieces of it in new free programs;
and that you know you can do these things.

To protect your rights, we need to make restrictions that forbid anyone to
deny you these rights or to ask you to surrender the rights.  These
restrictions translate to certain responsibilities for you if you
distribute copies of the software, or if you modify it.

For example, if you distribute copies of such a program, whether gratis or
for a fee, you must give the recipients all the rights that you have.  You
must make sure that they, too, receive or can get the source code.  And
you must show them these terms so they know their rights.

We protect your rights with two steps: (1) copyright the software, and (2)
offer you this license which gives you legal permission to copy,
distribute and/or modify the software.

Also, for each author's protection and ours, we want to make certain that
everyone understands that there is no warranty for this free software.  If
the software is modified by someone else and passed on, we want its
recipients to know that what they have is not the original, so that any
problems introduced by others will not reflect on the original authors'
reputations.

Finally, any free program is threatened constantly by software patents.
We wish to avoid the danger that redistributors of a free program will
individually obtain patent licenses, in effect making the program
proprietary.  To prevent this, we have made it clear that any patent must
be licensed for everyone's free use or not licensed at all.

The precise terms and conditions for copying, distribution and
modification follow.

\begin{center}
{\Large \sc Terms and Conditions For Copying, Distribution and
  Modification}
\end{center}


%\renewcommand{\theenumi}{\alpha{enumi}}
\begin{enumerate}

\addtocounter{enumi}{-1}

\item 

This License applies to any program or other work which contains a notice
placed by the copyright holder saying it may be distributed under the
terms of this General Public License.  The ``Program'', below, refers to
any such program or work, and a ``work based on the Program'' means either
the Program or any derivative work under copyright law: that is to say, a
work containing the Program or a portion of it, either verbatim or with
modifications and/or translated into another language.  (Hereinafter,
translation is included without limitation in the term ``modification''.)
Each licensee is addressed as ``you''.

Activities other than copying, distribution and modification are not
covered by this License; they are outside its scope.  The act of
running the Program is not restricted, and the output from the Program
is covered only if its contents constitute a work based on the
Program (independent of having been made by running the Program).
Whether that is true depends on what the Program does.

\item You may copy and distribute verbatim copies of the Program's source
  code as you receive it, in any medium, provided that you conspicuously
  and appropriately publish on each copy an appropriate copyright notice
  and disclaimer of warranty; keep intact all the notices that refer to
  this License and to the absence of any warranty; and give any other
  recipients of the Program a copy of this License along with the Program.

You may charge a fee for the physical act of transferring a copy, and you
may at your option offer warranty protection in exchange for a fee.

\item

You may modify your copy or copies of the Program or any portion
of it, thus forming a work based on the Program, and copy and
distribute such modifications or work under the terms of Section 1
above, provided that you also meet all of these conditions:

\begin{enumerate}

\item 

You must cause the modified files to carry prominent notices stating that
you changed the files and the date of any change.

\item

You must cause any work that you distribute or publish, that in
whole or in part contains or is derived from the Program or any
part thereof, to be licensed as a whole at no charge to all third
parties under the terms of this License.

\item
If the modified program normally reads commands interactively
when run, you must cause it, when started running for such
interactive use in the most ordinary way, to print or display an
announcement including an appropriate copyright notice and a
notice that there is no warranty (or else, saying that you provide
a warranty) and that users may redistribute the program under
these conditions, and telling the user how to view a copy of this
License.  (Exception: if the Program itself is interactive but
does not normally print such an announcement, your work based on
the Program is not required to print an announcement.)

\end{enumerate}


These requirements apply to the modified work as a whole.  If
identifiable sections of that work are not derived from the Program,
and can be reasonably considered independent and separate works in
themselves, then this License, and its terms, do not apply to those
sections when you distribute them as separate works.  But when you
distribute the same sections as part of a whole which is a work based
on the Program, the distribution of the whole must be on the terms of
this License, whose permissions for other licensees extend to the
entire whole, and thus to each and every part regardless of who wrote it.

Thus, it is not the intent of this section to claim rights or contest
your rights to work written entirely by you; rather, the intent is to
exercise the right to control the distribution of derivative or
collective works based on the Program.

In addition, mere aggregation of another work not based on the Program
with the Program (or with a work based on the Program) on a volume of
a storage or distribution medium does not bring the other work under
the scope of this License.

\item
You may copy and distribute the Program (or a work based on it,
under Section 2) in object code or executable form under the terms of
Sections 1 and 2 above provided that you also do one of the following:

\begin{enumerate}

\item

Accompany it with the complete corresponding machine-readable
source code, which must be distributed under the terms of Sections
1 and 2 above on a medium customarily used for software interchange; or,

\item

Accompany it with a written offer, valid for at least three
years, to give any third party, for a charge no more than your
cost of physically performing source distribution, a complete
machine-readable copy of the corresponding source code, to be
distributed under the terms of Sections 1 and 2 above on a medium
customarily used for software interchange; or,

\item

Accompany it with the information you received as to the offer
to distribute corresponding source code.  (This alternative is
allowed only for noncommercial distribution and only if you
received the program in object code or executable form with such
an offer, in accord with Subsection b above.)

\end{enumerate}


The source code for a work means the preferred form of the work for
making modifications to it.  For an executable work, complete source
code means all the source code for all modules it contains, plus any
associated interface definition files, plus the scripts used to
control compilation and installation of the executable.  However, as a
special exception, the source code distributed need not include
anything that is normally distributed (in either source or binary
form) with the major components (compiler, kernel, and so on) of the
operating system on which the executable runs, unless that component
itself accompanies the executable.

If distribution of executable or object code is made by offering
access to copy from a designated place, then offering equivalent
access to copy the source code from the same place counts as
distribution of the source code, even though third parties are not
compelled to copy the source along with the object code.

\item
You may not copy, modify, sublicense, or distribute the Program
except as expressly provided under this License.  Any attempt
otherwise to copy, modify, sublicense or distribute the Program is
void, and will automatically terminate your rights under this License.
However, parties who have received copies, or rights, from you under
this License will not have their licenses terminated so long as such
parties remain in full compliance.

\item
You are not required to accept this License, since you have not
signed it.  However, nothing else grants you permission to modify or
distribute the Program or its derivative works.  These actions are
prohibited by law if you do not accept this License.  Therefore, by
modifying or distributing the Program (or any work based on the
Program), you indicate your acceptance of this License to do so, and
all its terms and conditions for copying, distributing or modifying
the Program or works based on it.

\item
Each time you redistribute the Program (or any work based on the
Program), the recipient automatically receives a license from the
original licensor to copy, distribute or modify the Program subject to
these terms and conditions.  You may not impose any further
restrictions on the recipients' exercise of the rights granted herein.
You are not responsible for enforcing compliance by third parties to
this License.

\item
If, as a consequence of a court judgment or allegation of patent
infringement or for any other reason (not limited to patent issues),
conditions are imposed on you (whether by court order, agreement or
otherwise) that contradict the conditions of this License, they do not
excuse you from the conditions of this License.  If you cannot
distribute so as to satisfy simultaneously your obligations under this
License and any other pertinent obligations, then as a consequence you
may not distribute the Program at all.  For example, if a patent
license would not permit royalty-free redistribution of the Program by
all those who receive copies directly or indirectly through you, then
the only way you could satisfy both it and this License would be to
refrain entirely from distribution of the Program.

If any portion of this section is held invalid or unenforceable under
any particular circumstance, the balance of the section is intended to
apply and the section as a whole is intended to apply in other
circumstances.

It is not the purpose of this section to induce you to infringe any
patents or other property right claims or to contest validity of any
such claims; this section has the sole purpose of protecting the
integrity of the free software distribution system, which is
implemented by public license practices.  Many people have made
generous contributions to the wide range of software distributed
through that system in reliance on consistent application of that
system; it is up to the author/donor to decide if he or she is willing
to distribute software through any other system and a licensee cannot
impose that choice.

This section is intended to make thoroughly clear what is believed to
be a consequence of the rest of this License.

\item
If the distribution and/or use of the Program is restricted in
certain countries either by patents or by copyrighted interfaces, the
original copyright holder who places the Program under this License
may add an explicit geographical distribution limitation excluding
those countries, so that distribution is permitted only in or among
countries not thus excluded.  In such case, this License incorporates
the limitation as if written in the body of this License.

\item
The Free Software Foundation may publish revised and/or new versions
of the General Public License from time to time.  Such new versions will
be similar in spirit to the present version, but may differ in detail to
address new problems or concerns.

Each version is given a distinguishing version number.  If the Program
specifies a version number of this License which applies to it and ``any
later version'', you have the option of following the terms and conditions
either of that version or of any later version published by the Free
Software Foundation.  If the Program does not specify a version number of
this License, you may choose any version ever published by the Free Software
Foundation.

\item
If you wish to incorporate parts of the Program into other free
programs whose distribution conditions are different, write to the author
to ask for permission.  For software which is copyrighted by the Free
Software Foundation, write to the Free Software Foundation; we sometimes
make exceptions for this.  Our decision will be guided by the two goals
of preserving the free status of all derivatives of our free software and
of promoting the sharing and reuse of software generally.

\begin{center}
{\Large\sc
No Warranty
}
\end{center}

\item
{\sc Because the program is licensed free of charge, there is no warranty
for the program, to the extent permitted by applicable law.  Except when
otherwise stated in writing the copyright holders and/or other parties
provide the program ``as is'' without warranty of any kind, either expressed
or implied, including, but not limited to, the implied warranties of
merchantability and fitness for a particular purpose.  The entire risk as
to the quality and performance of the program is with you.  Should the
program prove defective, you assume the cost of all necessary servicing,
repair or correction.}

\item
{\sc In no event unless required by applicable law or agreed to in writing
will any copyright holder, or any other party who may modify and/or
redistribute the program as permitted above, be liable to you for damages,
including any general, special, incidental or consequential damages arising
out of the use or inability to use the program (including but not limited
to loss of data or data being rendered inaccurate or losses sustained by
you or third parties or a failure of the program to operate with any other
programs), even if such holder or other party has been advised of the
possibility of such damages.}

\end{enumerate}


\begin{center}
{\Large\sc End of Terms and Conditions}
\end{center}


\pagebreak[2]

\section*{Appendix: How to Apply These Terms to Your New Programs}

If you develop a new program, and you want it to be of the greatest
possible use to the public, the best way to achieve this is to make it
free software which everyone can redistribute and change under these
terms.

  To do so, attach the following notices to the program.  It is safest to
  attach them to the start of each source file to most effectively convey
  the exclusion of warranty; and each file should have at least the
  ``copyright'' line and a pointer to where the full notice is found.

\begin{quote}
one line to give the program's name and a brief idea of what it does. \\
Copyright (C) yyyy  name of author \\

This program is free software; you can redistribute it and/or modify
it under the terms of the GNU General Public License as published by
the Free Software Foundation; either version 2 of the License, or
(at your option) any later version.

This program is distributed in the hope that it will be useful,
but WITHOUT ANY WARRANTY; without even the implied warranty of
MERCHANTABILITY or FITNESS FOR A PARTICULAR PURPOSE.  See the
GNU General Public License for more details.

You should have received a copy of the GNU General Public License
along with this program; if not, write to the Free Software
Foundation, Inc., 51 Franklin Street, Fifth Floor, Boston, MA  02110-1301, USA.
\end{quote}

Also add information on how to contact you by electronic and paper mail.

If the program is interactive, make it output a short notice like this
when it starts in an interactive mode:

\begin{quote}
Gnomovision version 69, Copyright (C) yyyy  name of author \\
Gnomovision comes with ABSOLUTELY NO WARRANTY; for details type `show w'. \\
This is free software, and you are welcome to redistribute it
under certain conditions; type `show c' for details.
\end{quote}


The hypothetical commands {\tt show w} and {\tt show c} should show the
appropriate parts of the General Public License.  Of course, the commands
you use may be called something other than {\tt show w} and {\tt show c};
they could even be mouse-clicks or menu items---whatever suits your
program.

You should also get your employer (if you work as a programmer) or your
school, if any, to sign a ``copyright disclaimer'' for the program, if
necessary.  Here is a sample; alter the names:

\begin{quote}
Yoyodyne, Inc., hereby disclaims all copyright interest in the program \\
`Gnomovision' (which makes passes at compilers) written by James Hacker. \\

signature of Ty Coon, 1 April 1989 \\
Ty Coon, President of Vice
\end{quote}


This General Public License does not permit incorporating your program
into proprietary programs.  If your program is a subroutine library, you
may consider it more useful to permit linking proprietary applications
with the library.  If this is what you want to do, use the GNU Library
General Public License instead of this License.


\section{Creative Commons Attribution 4.0 International Public License}
\label{appCCBY4}
\setlength{\parindent}{0cm}
\setlength{\parskip}{0cm}
\setlength{\itemsep}{0cm}
\small

By exercising the Licensed Rights (defined below), You accept and agree to be bound by the terms and conditions of this Creative Commons Attribution 4.0 International Public License ("Public License"). To the extent this Public License may be interpreted as a contract, You are granted the Licensed Rights in consideration of Your acceptance of these terms and conditions, and the Licensor grants You such rights in consideration of benefits the Licensor receives from making the Licensed Material available under these terms and conditions.

\subsection*{Section 1 – Definitions.}

\begin{description}
\setlength{\itemsep}{0cm}
\item[a] {\bf Adapted Material} means material subject to Copyright and Similar Rights that is derived from or based upon the Licensed Material and in which the Licensed Material is translated, altered, arranged, transformed, or otherwise modified in a manner requiring permission under the Copyright and Similar Rights held by the Licensor. For purposes of this Public License, where the Licensed Material is a musical work, performance, or sound recording, Adapted Material is always produced where the Licensed Material is synched in timed relation with a moving image.
\item[b] {\bf Adapter's License} means the license You apply to Your Copyright and Similar Rights in Your contributions to Adapted Material in accordance with the terms and conditions of this Public License.
\item[c] {\bf Copyright and Similar Rights} means copyright and/or similar rights closely related to copyright including, without limitation, performance, broadcast, sound recording, and Sui Generis Database Rights, without regard to how the rights are labeled or categorized. For purposes of this Public License, the rights specified in Section 2(b)(1)-(2) are not Copyright and Similar Rights.
\item[d] {\bf Effective Technological Measures} means those measures that, in the absence of proper authority, may not be circumvented under laws fulfilling obligations under Article 11 of the WIPO Copyright Treaty adopted on December 20, 1996, and/or similar international agreements.
\item[e] {\bf Exceptions and Limitations} means fair use, fair dealing, and/or any other exception or limitation to Copyright and Similar Rights that applies to Your use of the Licensed Material.
\item[f] {\bf Licensed Material} means the artistic or literary work, database, or other material to which the Licensor applied this Public License.
\item[g] {\bf Licensed Rights} means the rights granted to You subject to the terms and conditions of this Public License, which are limited to all Copyright and Similar Rights that apply to Your use of the Licensed Material and that the Licensor has authority to license.
\item[h] {\bf Licensor} means the individual(s) or entity(ies) granting rights under this Public License.
\item[i] {\bf Share} means to provide material to the public by any means or process that requires permission under the Licensed Rights, such as reproduction, public display, public performance, distribution, dissemination, communication, or importation, and to make material available to the public including in ways that members of the public may access the material from a place and at a time individually chosen by them.
\item[j] {\bf Sui Generis Database Rights} means rights other than copyright resulting from Directive 96/9/EC of the European Parliament and of the Council of 11 March 1996 on the legal protection of databases, as amended and/or succeeded, as well as other essentially equivalent rights anywhere in the world.
\item[j] {\bf You} means the individual or entity exercising the Licensed Rights under this Public License. {\bf Your} has a corresponding meaning.
\end{description}
  
\subsection*{Section 2 – Scope.}

\subsubsection*{a License grant.}
\begin{enumerate}
\setlength{\itemsep}{0cm}
\item        Subject to the terms and conditions of this Public License, the Licensor hereby grants You a worldwide, royalty-free, non-sublicensable, non-exclusive, irrevocable license to exercise the Licensed Rights in the Licensed Material to:
\begin{description}
\setlength{\itemsep}{0cm}
\item[A.]            reproduce and Share the Licensed Material, in whole or in part; and
\item[B.]          produce, reproduce, and Share Adapted Material.
\end{description}
\item \underline{Exceptions and Limitations.} For the avoidance of doubt, where Exceptions and Limitations apply to Your use, this Public License does not apply, and You do not need to comply with its terms and conditions.
\item \underline{Term.} The term of this Public License is specified in Section 6(a).
\item \underline{Media and formats; technical modifications allowed.} The Licensor authorizes You to exercise the Licensed Rights in all media and formats whether now known or hereafter created, and to make technical modifications necessary to do so. The Licensor waives and/or agrees not to assert any right or authority to forbid You from making technical modifications necessary to exercise the Licensed Rights, including technical modifications necessary to circumvent Effective Technological Measures. For purposes of this Public License, simply making modifications authorized by this Section 2(a)(4) never produces Adapted Material.
\item \underline{Downstream recipients.}
\begin{description}
\item[A.] \underline{Offer from the Licensor – Licensed Material.} Every recipient of the Licensed Material automatically receives an offer from the Licensor to exercise the Licensed Rights under the terms and conditions of this Public License.
\item[B.] \underline{No downstream restrictions.} You may not offer or impose any additional or different terms or conditions on, or apply any Effective Technological Measures to, the Licensed Material if doing so restricts exercise of the Licensed Rights by any recipient of the Licensed Material.
\end{description}
\item \underline{No endorsement.} Nothing in this Public License constitutes or may be construed as permission to assert or imply that You are, or that Your use of the Licensed Material is, connected with, or sponsored, endorsed, or granted official status by, the Licensor or others designated to receive attribution as provided in Section 3(a)(1)(A)(i).
\end{enumerate}
      
\subsubsection*{b Other rights.}

\begin{enumerate}
\setlength{\itemsep}{0cm}
\item Moral rights, such as the right of integrity, are not licensed under this Public License, nor are publicity, privacy, and/or other similar personality rights; however, to the extent possible, the Licensor waives and/or agrees not to assert any such rights held by the Licensor to the limited extent necessary to allow You to exercise the Licensed Rights, but not otherwise.
\item        Patent and trademark rights are not licensed under this Public License.
\item        To the extent possible, the Licensor waives any right to collect royalties from You for the exercise of the Licensed Rights, whether directly or through a collecting society under any voluntary or waivable statutory or compulsory licensing scheme. In all other cases the Licensor expressly reserves any right to collect such royalties.
\end{enumerate}

\subsection*{Section 3 – License Conditions.}

Your exercise of the Licensed Rights is expressly made subject to the following conditions.

\subsubsection*{a. Attribution.}

\begin{enumerate}
\setlength{\itemsep}{0cm}
\item If You Share the Licensed Material (including in modified form), You must:
\begin{description}
\item[A.]            retain the following if it is supplied by the Licensor with the Licensed Material:
\begin{description}
\item[i.]                identification of the creator(s) of the Licensed Material and any others designated to receive attribution, in any reasonable manner requested by the Licensor (including by pseudonym if designated);
\item[ii.]                a copyright notice;
\item[iii.]                a notice that refers to this Public License;
\item[iv.]                a notice that refers to the disclaimer of warranties;
\item[v.]                a URI or hyperlink to the Licensed Material to the extent reasonably practicable;
\end{description}
\item[B.] indicate if You modified the Licensed Material and retain an indication of any previous modifications; and
\item[C.] indicate the Licensed Material is licensed under this Public License, and include the text of, or the URI or hyperlink to, this Public License.
\end{description}
\item You may satisfy the conditions in Section 3(a)(1) in any reasonable manner based on the medium, means, and context in which You Share the Licensed Material. For example, it may be reasonable to satisfy the conditions by providing a URI or hyperlink to a resource that includes the required information.
\item        If requested by the Licensor, You must remove any of the information required by Section 3(a)(1)(A) to the extent reasonably practicable.
\item        If You Share Adapted Material You produce, the Adapter's License You apply must not prevent recipients of the Adapted Material from complying with this Public License.
\end{enumerate}

\subsection*{Section 4 – Sui Generis Database Rights.}

Where the Licensed Rights include Sui Generis Database Rights that apply to Your use of the Licensed Material:
\begin{description}
\setlength{\itemsep}{0cm}
\item [a.]
    for the avoidance of doubt, Section 2(a)(1) grants You the right to extract, reuse, reproduce, and Share all or a substantial portion of the contents of the database;
\item[b.]
    if You include all or a substantial portion of the database contents in a database in which You have Sui Generis Database Rights, then the database in which You have Sui Generis Database Rights (but not its individual contents) is Adapted Material; and
\item[c.]
    You must comply with the conditions in Section 3(a) if You Share all or a substantial portion of the contents of the database.
  \end{description}
  For the avoidance of doubt, this Section 4 supplements and does not replace Your obligations under this Public License where the Licensed Rights include other Copyright and Similar Rights.

\subsection*{Section 5 – Disclaimer of Warranties and Limitation of Liability.}
\begin{description}
\setlength{\itemsep}{0cm}
\item[a.] \bf
    Unless otherwise separately undertaken by the Licensor, to the extent possible, the Licensor offers the Licensed Material as-is and as-available, and makes no representations or warranties of any kind concerning the Licensed Material, whether express, implied, statutory, or other. This includes, without limitation, warranties of title, merchantability, fitness for a particular purpose, non-infringement, absence of latent or other defects, accuracy, or the presence or absence of errors, whether or not known or discoverable. Where disclaimers of warranties are not allowed in full or in part, this disclaimer may not apply to You.
\item[b.] \bf    To the extent possible, in no event will the Licensor be liable to You on any legal theory (including, without limitation, negligence) or otherwise for any direct, special, indirect, incidental, consequential, punitive, exemplary, or other losses, costs, expenses, or damages arising out of this Public License or use of the Licensed Material, even if the Licensor has been advised of the possibility of such losses, costs, expenses, or damages. Where a limitation of liability is not allowed in full or in part, this limitation may not apply to You.
\item[c.]
    The disclaimer of warranties and limitation of liability provided above shall be interpreted in a manner that, to the extent possible, most closely approximates an absolute disclaimer and waiver of all liability.
\end{description}

\subsection*{Section 6 – Term and Termination.}

\begin{description}
\setlength{\itemsep}{0cm}
\item[a.]
    This Public License applies for the term of the Copyright and Similar Rights licensed here. However, if You fail to comply with this Public License, then Your rights under this Public License terminate automatically.
\item[b.]
    Where Your right to use the Licensed Material has terminated under Section 6(a), it reinstates:
\begin{enumerate}
\setlength{\itemsep}{0cm}
\item         automatically as of the date the violation is cured, provided it is cured within 30 days of Your discovery of the violation; or
\item        upon express reinstatement by the Licensor.
\end{enumerate}
For the avoidance of doubt, this Section 6(b) does not affect any right the Licensor may have to seek remedies for Your violations of this Public License.
\item[c.]    For the avoidance of doubt, the Licensor may also offer the Licensed Material under separate terms or conditions or stop distributing the Licensed Material at any time; however, doing so will not terminate this Public License.
\item[d.]    Sections 1, 5, 6, 7, and 8 survive termination of this Public License.
\end{description}

\subsection*{Section 7 – Other Terms and Conditions.}

\begin{description}
\setlength{\itemsep}{0cm}
\item[a.]
    The Licensor shall not be bound by any additional or different terms or conditions communicated by You unless expressly agreed.
\item[b.]   Any arrangements, understandings, or agreements regarding the Licensed Material not stated herein are separate from and independent of the terms and conditions of this Public License.
\end{description}

\subsection*{Section 8 – Interpretation.}

\begin{description}
\setlength{\itemsep}{0cm}
\item[a.]
    For the avoidance of doubt, this Public License does not, and shall not be interpreted to, reduce, limit, restrict, or impose conditions on any use of the Licensed Material that could lawfully be made without permission under this Public License.
\item[b.] To the extent possible, if any provision of this Public License is deemed unenforceable, it shall be automatically reformed to the minimum extent necessary to make it enforceable. If the provision cannot be reformed, it shall be severed from this Public License without affecting the enforceability of the remaining terms and conditions.
\item[c.]    No term or condition of this Public License will be waived and no failure to comply consented to unless expressly agreed to by the Licensor.
\item[d.]    Nothing in this Public License constitutes or may be interpreted as a limitation upon, or waiver of, any privileges and immunities that apply to the Licensor or You, including from the legal processes of any jurisdiction or authority.
\end{description}

\begin{thebibliography}{99}
\phantomsection
\addcontentsline{toc}{section}{References}

\bibitem{Bijnens:2014gsa}
 J.~Bijnens,
  %``CHIRON: a package for ChPT numerical results at two loops,''
  Eur.\ Phys.\ J.\ C {\bf 75} (2015) 1,  27
  [arXiv:1412.0887 [hep-ph]].
  %%CITATION = ARXIV:1412.0887;%%

\bibitem{CClicense} \url{http://creativecommons.org/licenses/by/4.0/}.


\bibitem{chiron} \href{http://en.wikipedia.org/wiki/Chiron}{\tt http://en.wikipedia.org/wiki/Chiron}

\bibitem{GPLv2} \url{http://www.gnu.org/licenses/gpl-2.0.html}{\tt http://www.gnu.org/licenses/gpl-2.0.html}

\bibitem{chironsite} \href{http://www.thep.lu.se/\%7Ebijnens/chiron/}{\tt http://www.thep.lu.se/\textasciitilde{}bijnens/chiron/}

%\cite{'tHooft:1978xw}
\bibitem{'tHooft:1978xw}
  G.~'t Hooft, M.~J.~G.~Veltman,
  %``Scalar One Loop Integrals,''
  Nucl.\ Phys.\  {\bf B153 } (1979)  365-401.

\bibitem{cernlib} \href{http://cernlib.web.cern.ch}{\tt http://cernlib.web.cern.ch}

%\cite{Bijnens:2013doa}
\bibitem{Bijnens:2013doa}
  J.~Bijnens, E.~Bostr\"om and T.~A.~L\"ahde,
  %``Two-loop Sunset Integrals at Finite Volume,''
  JHEP {\bf 1401} (2014) 019
  [arXiv:1311.3531 [hep-lat]].
  %%CITATION = ARXIV:1311.3531;%%

\bibitem{radmulpaper} 
A. van Doren and L. de Ridder, 
%"An adaptive algorithm for numerical integration over an n-dimensional cube,"
J. Comput. Appl. Math. 2 (1976) 207-217. 

%\cite{Weinberg:1978kz}
\bibitem{Weinberg:1978kz}
  S.~Weinberg,
  %\emph{Phenomenological Lagrangians,}
  Physica A {\bf 96} (1979) 327.
  %%CITATION = PHYSA,A96,327;%%

%\cite{Gasser:1983yg}
\bibitem{Gasser:1983yg}
  J.~Gasser and H.~Leutwyler,
  %\emph{Chiral Perturbation Theory to One Loop,}
  Annals Phys.\  {\bf 158} (1984) 142.
  %%CITATION = APNYA,158,142;%%

 %\cite{Gasser:1984gg}
\bibitem{Gasser:1984gg}
  J.~Gasser and H.~Leutwyler,
  %\emph{Chiral Perturbation Theory: Expansions in the Mass of the Strange Quark,}
  Nucl.\ Phys.\ B {\bf 250} (1985) 465.
  %%CITATION = NUPHA,B250,465;%%

\bibitem{webpage} \href{http://www.thep.lu.se/~bijnens/chpt/}{\tt http://www.thep.lu.se/$\sim$bijnens/chpt/}

%\cite{Bijnens:2006zp}
\bibitem{Bijnens:2006zp}
  J.~Bijnens,
  %\emph{Chiral perturbation theory beyond one loop,}
  Prog.\ Part.\ Nucl.\ Phys.\  {\bf 58} (2007) 521
  [hep-ph/0604043].
  %%CITATION = HEP-PH/0604043;%%

%\cite{Bijnens:2011tb}
\bibitem{Bijnens:2011tb}
  J.~Bijnens and I.~Jemos,
  %``A new global fit of the $L^r_i$ at next-to-next-to-leading order in Chiral Perturbation Theory,''
  Nucl.\ Phys.\ B {\bf 854} (2012) 631
  [arXiv:1103.5945 [hep-ph]].
  %%CITATION = ARXIV:1103.5945;%%

%\cite{Bijnens:2014lea}
\bibitem{Bijnens:2014lea}
  J.~Bijnens and G.~Ecker,
  %``Mesonic low-energy constants,''
  arXiv:1405.6488 [hep-ph].
  %%CITATION = ARXIV:1405.6488;%%
  XXX add published XXX

%\cite{Bijnens:1999sh}
\bibitem{Bijnens:1999sh}
  J.~Bijnens, G.~Colangelo and G.~Ecker,
  %``The Mesonic chiral Lagrangian of order p**6,''
  JHEP {\bf 9902} (1999) 020
  [hep-ph/9902437].
  %%CITATION = HEP-PH/9902437;%%

%\cite{Bijnens:1999hw}
\bibitem{Bijnens:1999hw}
  J.~Bijnens, G.~Colangelo and G.~Ecker,
  %``Renormalization of chiral perturbation theory to order p**6,''
  Annals Phys.\  {\bf 280} (2000) 100
  [hep-ph/9907333].
  %%CITATION = HEP-PH/9907333;%%

%\cite{Amoros:1999dp}
\bibitem{Amoros:1999dp}
  G.~Amor\'os, J.~Bijnens and P.~Talavera,
  %``Two point functions at two loops in three flavor chiral perturbation theory,''
  Nucl.\ Phys.\ B {\bf 568} (2000) 319
  [hep-ph/9907264].
  %%CITATION = HEP-PH/9907264;%%

%\cite{Bijnens:2002hp}
\bibitem{Bijnens:2002hp}
  J.~Bijnens and P.~Talavera,
  %``Pion and kaon electromagnetic form-factors,''
  JHEP {\bf 0203} (2002) 046
  [hep-ph/0203049].
  %%CITATION = HEP-PH/0203049;%%

%\cite{Passarino:1978jh}
\bibitem{Passarino:1978jh}
  G.~Passarino and M.~J.~G.~Veltman,
  %``One Loop Corrections for e+ e- Annihilation Into mu+ mu- in the Weinberg Model,''
  Nucl.\ Phys.\ B {\bf 160} (1979) 151.
  %%CITATION = NUPHA,B160,151;%%

%\cite{Bijnens:2006jv}
\bibitem{Bijnens:2006jv}
  J.~Bijnens, N.~Danielsson and T.~A.~L\"ahde,
  %``Three-flavor partially quenched chiral perturbation theory at NNLO for meson masses and decay constants,''
  Phys.\ Rev.\ D {\bf 73} (2006) 074509
  [hep-lat/0602003].
  %%CITATION = HEP-LAT/0602003;%%
  %38 citations counted in INSPIRE as of 21 Jan 2015

%\cite{Bijnens:2005pa}
\bibitem{Bijnens:2005pa}
  J.~Bijnens and T.~A.~L\"ahde,
  %``Masses and decay constants of pseudoscalar mesons to two loops in two-flavor partially quenched chiral perturbation theory,''
  Phys.\ Rev.\ D {\bf 72} (2005) 074502
  [hep-lat/0506004].
  %%CITATION = HEP-LAT/0506004;%%
  %27 citations counted in INSPIRE as of 21 Jan 2015

%\cite{Bijnens:2005ae}
\bibitem{Bijnens:2005ae}
  J.~Bijnens and T.~A.~L\"ahde,
  %``Decay constants of pseudoscalar mesons to two loops in three-flavor partially quenched (chi)PT,''
  Phys.\ Rev.\ D {\bf 71} (2005) 094502
  [hep-lat/0501014].
  %%CITATION = HEP-LAT/0501014;%%
  %25 citations counted in INSPIRE as of 21 Jan 2015

%\cite{Bijnens:2004hk}
\bibitem{Bijnens:2004hk}
  J.~Bijnens, N.~Danielsson and T.~A.~L\"ahde,
  %``The Pseudoscalar meson mass to two loops in three-flavor partially quenched chiPT,''
  Phys.\ Rev.\ D {\bf 70} (2004) 111503
  [hep-lat/0406017].
  %%CITATION = HEP-LAT/0406017;%%
  %31 citations counted in INSPIRE as of 21 Jan 2015

%\cite{Bijnens:2014dea}
\bibitem{Bijnens:2014dea}
  J.~Bijnens and T.~Rössler,
  %``Finite Volume at Two-loops in Chiral Perturbation Theory,''
  JHEP {\bf 1501} (2015) 034
  [arXiv:1411.6384 [hep-lat]].
  %%CITATION = ARXIV:1411.6384;%%

%\cite{Bijnens:2006ve}
\bibitem{Bijnens:2006ve}
  J.~Bijnens and K.~Ghorbani,
  %``Finite volume dependence of the quark-antiquark vacuum expectation value,''
  Phys.\ Lett.\ B {\bf 636} (2006) 51
  [hep-lat/0602019].
  %%CITATION = HEP-LAT/0602019;%%

%\cite{Golowich:1997zs}
\bibitem{Golowich:1997zs}
  E.~Golowich and J.~Kambor,
  %``Two loop analysis of axial vector current propagators in chiral perturbation theory,''
  Phys.\ Rev.\ D {\bf 58} (1998) 036004
  [hep-ph/9710214].
  %%CITATION = HEP-PH/9710214;%%

%\cite{Amoros:2000mc}
\bibitem{Amoros:2000mc}
  G.~Amor\'os, J.~Bijnens and P.~Talavera,
  %``K(lepton 4) form-factors and pi pi scattering,''
  Nucl.\ Phys.\ B {\bf 585} (2000) 293
   [Erratum-ibid.\ B {\bf 598} (2001) 665]
  [hep-ph/0003258].
  %%CITATION = HEP-PH/0003258;%%

  %\cite{Bijnens:2015dra}
\bibitem{Bijnens:2015dra}
  J.~Bijnens and T.~Rössler,
  %``Finite Volume for Three-Flavour Partially Quenched Chiral Perturbation Theory through NNLO in the Meson Sector,''
  arXiv:1508.07238 [hep-lat].
  %%CITATION = ARXIV:1508.07238;%%

\bibitem{Bijnens:2009qm}
  J.~Bijnens and J.~Lu,
  %``Technicolor and other QCD-like theories at next-to-next-to-leading order,''
  JHEP {\bf 0911} (2009) 116
  [arXiv:0910.5424 [hep-ph]].
  %%CITATION = ARXIV:0910.5424;%%

\bibitem{QCDliketemp}
  J.~Bijnens and T.~Rössler, to be published

\end{thebibliography}

\newpage
\phantomsection
\addcontentsline{toc}{section}{Index}\printindex
\end{document}